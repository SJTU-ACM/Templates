\documentclass[a4paper,10pt]{book}
\usepackage{amsmath}
\usepackage{amssymb}
\usepackage{fontspec}
\usepackage{listings} 
\usepackage{harpoon}
\usepackage[left=1.5cm, right=1.5cm]{geometry}
\usepackage[BoldFont]{xeCJK}
\newcommand{\ud}{\mathrm{d}}
\oddsidemargin -0.1 true cm
\if@twoside
	\evensidemargin -0.1 true cm
\fi
\setlength{\parindent}{0em}

%\setCJKmainfont[BoldFont=Songti SC Black]{Songti SC Regular}          % 设置缺省中文字体
\setCJKmainfont{Hiragino Sans GB}          % 设置缺省中文字体
%\setCJKmonofont{Songti SC Black}                                      % 设置等宽字体
\setmainfont{Courier}                                                  % 英文衬线字体
%\setmonofont{Monaco}                                                  % 英文等宽字体
%\setsansfont{Trebuchet MS}                                            % 英文无衬线字体
%\setCJKmainfont{SimSong}
\lstset{
	language=C++,
	numbers=left,
	tabsize=4,
	breaklines=tr,
	extendedchars=false
}
\title{\LARGE{Standard Code Library}}
\author{Tempest}
\date{October, 2014}
\begin{document}
\maketitle
\tableofcontents
\newpage
\chapter{数论算法}
	\section{$O(m^2\log n)$求线性递推数列第n项}
		已知$a_0, a_1, ..., a_{m - 1}$\\
	$a_n = c_0 * a_{n - m} + ... + c_{m - 1} * a_{n - 1}$\\
	求$a_n = v_0 * a_0 + v_1 * a_1 + ... + v_{m - 1} * a_{m - 1}$\\
\begin{lstlisting}
void linear_recurrence(long long n, int m, int a[], int c[], int p) {
	long long v[M] = {1 % p}, u[M << 1], msk = !!n;
	for(long long i(n); i > 1; i >>= 1) msk <<= 1;
	for(long long x(0); msk; msk >>= 1, x <<= 1) {
		fill_n(u, m << 1, 0);
		int b(!!(n & msk));
		x |= b;
		if(x < m) u[x] = 1 % p;
		else {
			for(int i(0); i < m; i++)
				for(int j(0), t(i + b); j < m; j++, t++)
					u[t] = (u[t] + v[i] * v[j]) % p;
			for(int i((m << 1) - 1); i >= m; i--)
				for(int j(0), t(i - m); j < m; j++, t++)
					u[t] = (u[t] + c[j] * u[i]) % p;
		}
		copy(u, u + m, v);
	}
	//a[n] = v[0] * a[0] + v[1] * a[1] + ... + v[m - 1] * a[m - 1].
	for(int i(m); i < 2 * m; i++) {
		a[i] = 0;
		for(int j(0); j < m; j++)
			a[i] = (a[i] + (long long)c[j] * a[i + j - m]) % p;
	}
	for(int j(0); j < m; j++) {
		b[j] = 0;
		for(int i(0); i < m; i++)
			b[j] = (b[j] + v[i] * a[i + j]) % p;
	}
	for(int j(0); j < m; j++)
		a[j] = b[j];
}
\end{lstlisting}

	\section{NTT}
		\begin{lstlisting}
const int modulo(786433);
const int G(10);//原根
int pw[999999];
void FFT(int P[], int n, int oper) {
	for(int i(1), j(0); i < n - 1; i++) {
		for(int s(n); j ^= s >>= 1, ~j & s;);
		if (i < j) 
			swap(P[i], P[j]);
	}
	int unit_p0;
	for(int d(0); (1 << d) < n; d++) {
		int m(1 << d), m2(m * 2);
		unit_p0 = oper == 1?pw[(modulo - 1) / m2]:pw[modulo - 1 - (modulo - 1) / m2];
		for(int i = 0; i < n; i += m2) {
			int unit(1);
			for(int j(0); j < m; j++) {
				int &P1 = P[i + j + m], &P2 = P[i + j];
				int t = (long long)unit * P1 % modulo;
				P1 = (P2 - t + modulo) % modulo;
				P2 = (P2 + t) % modulo;
				unit = (long long)unit * unit_p0 % modulo;
			}
		}
	}
}

int nn;
int A[N], B[N], C[N];
//A * B = C;
//len = nn
void multiply() {
	FFT(A, nn, 1);
	FFT(B, nn, 1);
	for(int i(0); i < nn; i++) {
		C[i] = (long long)A[i] * B[i] % modulo;
	}
	FFT(C,  nn, -1);
}

int main() {
	pw[0] = 1;
	for(int i(1); i < modulo; i++) {
		pw[i] = (long long)pw[i - 1] * G % modulo;
	}
}
\end{lstlisting}

	\section{中国剩余定理}
		包括扩展欧几里得,求逆元,和保证除数互质条件下的CRT
\begin{lstlisting}
LL x, y;
void exGcd(LL a, LL b)
{
	if (b == 0) {
		x = 1;
		y = 0;
		return;
	}
	exGcd(b, a % b);
	LL k = y;
	y = x - a / b * y;
	x = k;
}

LL inversion(LL a, LL b)
{
	exGcd(a, b);
	return (x % b + b) % b;
}

LL CRT(vector<LL> m, vector<LL> a)
{
	int N = m.size();
	LL M = 1, ret = 0;
	for(int i = 0; i < N; ++ i)
		M *= m[i];
	
	for(int i = 0; i < N; ++ i) {
		ret = (ret + (M / m[i]) * a[i] % M * inversion(M / m[i], m[i])) % M;
	}
	return ret;
}
\end{lstlisting}


	\section{中国剩余定理(可不互质)}
		\begin{lstlisting}
namespace number_theory_basic {
	inline void euclid(const long long &a, const long long &b, long long &x, long long &y) {
		if (b == 0) {
			x = 1;
			y = 0;
		} else {
			euclid(b, a % b, x, y);
			x -= a / b * y;
			swap(x, y);
		}
	}
}
namespace chinese_remainder_theorem {
	inline bool crt(int n, long long r[], long long m[], long long &remainder, long long &modular) {
		remainder = modular = 1;
		for (int i = 1; i <= n; ++i) {
			long long x, y;
			euclid(modular, m[i], x, y);
			long long divisor = gcd(modular, m[i]);
			if ((r[i] - remainder) % divisor) {
				return false;
			}
			x *= (r[i] - remainder) / divisor;
			remainder += modular * x;
			modular *= m[i] / divisor;
			((remainder %= modular) += modular) %= modular;
		}
		return true;
	}
}
\end{lstlisting}

	\section{Miller Rabin}	
		\begin{lstlisting}
int const n = 3; int const base[] = {2, 7, 61};
int const n = 9; int const base[] = {2, 3, 5, 7, 11, 13, 17, 19, 23};
inline long long power(const long long &x, const long long &k, const long long &modular) {
	long long ans = 1, num = x % modular;
	for (long long i = k; i > 0; i >>= 1) {
		if (i & 1) ans = multiply(ans, num, modular);
		num = multiply(num, num, modular);
	} return ans;
}
inline bool check(const long long &p, const long long &base) {
	long long n = p - 1; for (; !(n & 1); n >>= 1);
	long long m = power(base, n, p);
	for (; n != p - 1 && m != 1 && m != p - 1; )
		m = multiply(m, m, p), n <<= 1;
	return m == p - 1 || (n & 1) == 1;
}
inline bool prime(const long long &p) {
	for (int i = 0; i < n; ++i) if (base[i] == p) return true;
	if (p == 1 || !(p & 1)) return false;
	for (int i = 0; i < n; ++i) if (!check(p, base[i])) return false;
	return true;
}
\end{lstlisting}
	\section{Pollard Rho}
		\begin{lstlisting}
inline long long pollard_rho(const long long &n, const long long &c) {
	long long x = rand() % (n - 1) + 1, y = x;
	for (int head = 1, tail = 2; true; ) {
		x = multiply(x, x, n);
		if ((x += c) >= n) x -= n;
		if (x == y) return n;
		long long d = __gcd(abs(x - y), n);
		if (d > 1 && d < n) return d;
		if ((++head) == tail) y = x, tail <<= 1;
	}
}
inline vector<long long> mergy(const vector<long long> &a, const vector<long long> &b) {
	vector<long long> vec;
	for (int i = 0; i < (int)a.size(); ++i) vec.push_back(a[i]);
	for (int i = 0; i < (int)b.size(); ++i) vec.push_back(b[i]);
	return vec;
}
inline vector<long long> factor(const long long &n) {
	if (n <= 1) return vector<long long>();
	if (miller_rabin::prime(n)) return vector<long long>(1, n);
	long long p = n; for (; p >= n; p = pollard_rho(n, rand() % (n - 1) + 1));
	return mergy(factor(n / p), factor(p));
}
\end{lstlisting}
	\section{离散对数}
		\begin{lstlisting}
#include <iostream>
#include <cstdio>
#include <cstdlib>
#include <algorithm>
#include <cmath>
#include <map>
#include <cstring>

using namespace std;

typedef long long int64;

struct hash_table {
	static const int MAXN = 100003;
	int first[MAXN], key[MAXN], value[MAXN], next[MAXN], tot;
	hash_table() : tot(0) {
		memset(first, 255, sizeof first);
	}
	void clear() {
		memset(first, 255, sizeof first);
		tot = 0;
	}
	int &operator[] (const int &o) {
		int pos = o % MAXN;
		for (int i = first[pos]; i != -1; i = next[i])
			if (key[i] == o)
				return value[i];
		next[tot] = first[pos];
		first[pos] = tot;
		key[tot] = o;
		return value[tot++];
	}
	bool has_key(const int &o) {
		int pos = o % MAXN;
		for (int i = first[pos]; i != -1; i = next[i])
			if (key[i] == o)
				return true;
		return false;
	}
};

int discrete_log(int base, int n, int mod) {
	int block = int(sqrt(mod)) + 1;
	int val = 1;
	hash_table dict;
	for (int i = 0; i < block; ++i) {
		if (dict.has_key(val) == 0)
			dict[val] = i;
		val = (int64)val * base % mod;
	}
	int inv = inverse(val, mod);
	val = 1;
	for (int i = 0; i < block; ++i) {
		if (dict.has_key((int64)val * n % mod))
			return dict[(int64)val * n % mod] + i * block;
		val = (int64)val * inv % mod;
	}
	return -1;
}

int main() {
	int base, n, p;
	while (scanf("%d %d %d", &p, &base, &n) == 3) {
		int ans = discrete_log(base, n, p);
		if (ans == -1)
			puts("no solution");
		else
			printf("%d\n", ans);
	}
}
\end{lstlisting}

	\section{原根}
		\begin{lstlisting}
int primitive_root(int p) {
	int n = p - 1;
	while (true) {
		int root = rand() % (p - 1) + 1, m = n;
		bool found = true;
		for (int i = 0; i < (int)prim.size(); ++i) {
			int cur = prim[i];
			if (m / cur < cur)
				break;
			if (m % cur == 0) {
				if (pow_mod(root, n / cur, p) == 1) {
					found = false;
					break;
				}
				while (m % cur == 0)
					m /= cur;
			}
		}
		if (m > 1)
			if (pow_mod(root, n / m, p) == 1)
				found = false;
		if (found)
			return root;
	}
}

vector<int> discrete_root(int expo, int n, int mod) {
	if (n == 0)
		return vector<int>(1, 0);
	int g = primitive_root(mod);
	int e = discrete_log(g, n, mod);
	int64 u, v;
	int d = extend_euclid(expo, mod - 1, u, v);
	if (e % d != 0)
		return vector<int>();
	int64 delta = (mod - 1) / d;
	u = u * e / d % delta;
	if (u < 0)
		u += delta;
	vector<int> ret;
	while (u < mod - 1) {
		ret.push_back(pow_mod(g, u, mod));
		u += delta;
	}
	return ret;
}
\end{lstlisting}

	\section{离散二次方根}
		\begin{lstlisting}
inline bool quad_resi(int x, int p) {
	return pow_mod(x, (p - 1) / 2, p) == 1;
}

struct quad_poly {
	int zero, one, val, mod;

	quad_poly(int zero, int one, int val, int mod) : zero(zero), one(one), val
			(val), mod(mod) {}

	quad_poly multiply(quad_poly o) {
		int z0 = (zero * o.zero + one * o.one % mod * val % mod) % mod;
		int z1 = (zero * o.one + one * o.zero) % mod;
		return quad_poly(z0, z1, val, mod);
	}

	quad_poly pow(int x) {
		if (x == 1)
			return *this;
		quad_poly ret = this->pow(x / 2);
		ret = ret.multiply(ret);
		if (x & 1)
			ret = ret.multiply(*this);
		return ret;
	}
};

inline int calc(int a, int p) {
	a %= p;
	if (a < 2)
		return a;
	if (!quad_resi(a, p))
		return p;			// no solution
	if (p % 4 == 3)
		return pow_mod(a, (p + 1) / 4, p);
	int b = 0;
	while (quad_resi((my_sqr(b) - a + p) % p, p))
		b = rand() % p;
	quad_poly ret = quad_poly(b, 1, (my_sqr(b) - a + p) % p, p);
	ret = ret.pow((p + 1) / 2);
	return ret.zero;
}
\end{lstlisting}

	\section{牛顿迭代求平方根}
		\begin{lstlisting}
//use newton-method to solve f(x) = 0
//init x0
//xi -> x(i + 1) = xi - f(xi) / f'(xi)
//O(N^2logN)
int64 square_root(int64 x) {
	if (x <= 0)
		return 0;
	int64 last_root = -1, root = 1 << (bit_length(x) / 2);
	while (true) {
		int64 next_root = (root + x / root) >> 1;
		if (next_root == last_root)
			return min(next_root, root);
		last_root = root;
		root = next_root;
	}
}
\end{lstlisting}

	\section{Pell方程求根}
		$x^2-n*y^2=1$
\begin{lstlisting}
pair<int64, int64> solve_pell64(int64 n) {
	const static int MAXC = 111;
	int64 p[MAXC], q[MAXC], a[MAXC], g[MAXC], h[MAXC];
	p[1] = 1; p[0] = 0;
	q[1] = 0; q[0] = 1;
	a[2] = square_root(n);
	g[1] = 0; h[1] = 1;
	for (int i = 2; ; ++i) {
		g[i] = -g[i - 1] + a[i] * h[i - 1];
		h[i] = (n - g[i] * g[i]) / h[i - 1];
		a[i + 1] = (g[i] + a[2]) / h[i];
		p[i] = a[i] * p[i - 1] + p[i - 2];
		q[i] = a[i] * q[i - 1] + q[i - 2];
		if (p[i] * p[i] - n * q[i] * q[i] == 1)
			return make_pair(p[i], q[i]);
	}
}
\end{lstlisting}

	\section{直线下整点个数}
		求$\displaystyle\sum_{i=0}^{n-1} \lfloor\frac{a+bi}{m}\rfloor$.
	\begin{lstlisting}
LL count(LL n, LL a, LL b, LL m) {
	if (b == 0) return n * (a / m);
	if (a >= m) return n * (a / m) + count(n, a % m, b, m);
	if (b >= m) return (n - 1) * n / 2 * (b / m) + count(n, a, b % m, m);
	return count((a + b * n) / m, (a + b * n) % m, m, b);
}
	\end{lstlisting}

	
\chapter{数值算法}
	\section{FFT}
		\begin{lstlisting}
void FFT(Complex P[], int n, int oper) {
	for (int i(1), j(0); i < n - 1; i++) {
		for (int s(n); j ^= s >>= 1, ~j & s;);
		if (i < j)
			swap(P[i], P[j]);
	}
	Complex unit_p0;
	for (int d(0); (1 << d) < n; d++) {
		int m(1 << d), m2(m * 2);
		double p0(pi / m * oper);
		unit_p0.imag(sin(p0));
		unit_p0.real(cos(p0));
		for (int i(0); i < n; i += m2) {
			Complex unit = 1;
			for (int j = 0; j < m; j++) {
				Complex &P1 = P[i + j + m], &P2 = P[i + j];
				Complex t = unit * P1;
				P1 = P2 - t;
				P2 = P2 + t;
				unit = unit * unit_p0;
			}
		}
	}
}
void multiply() {
	FFT(a, n, 1);
	FFT(b, n, 1);
	for(int i(0); i < n; i++) {
		c[i] = a[i] * b[i];
	}
	FFT(c, n, -1);
	for(int i(0); i < n; i++) {
		ans[i] += (int)(c[i].real() / n + 0.5);
	}
}
\end{lstlisting}
	
	\section{解一元三次方程+求三阶二次型的标准型}
		\begin{lstlisting}
double sqr(const double & x) {
    return x * x;
}
double eps(1e-8);
int main() {
    double A, B, C, D, E, F;
    for(;6 == scanf("%lf%lf%lf%lf%lf%lf", &A, &B, &C, &D, &E, &F);) {
    	D /= 2; E /= 2; F /= 2;
        complex<double> a(1), b(-A - B - C)
        	, c(A * B + B * C + C * A - sqr(D) - sqr(E) - sqr(F))
        	, d(-A * B * C - 2 * D * E * F + A * sqr(D) + B * sqr(E) + C * sqr(F));
        complex<double> delta(pow(pow(b * c / 6. / a / a - b * b * b / 27. / a / a / a - d / 2. / a, 2) + pow(c / 3. / a - b * b / 9. / a / a, 3), 0.5));
        complex<double> p(pow(b * c / 6. / a / a - b * b * b / 27. / a / a / a - d / 2. / a + delta, 1. / 3));
        complex<double> q(pow(b * c / 6. / a / a - b * b * b / 27. / a / a / a - d / 2. / a - delta, 1. / 3));
        complex<double> omega1(-0.5, 0.5 * sqrt(3.)), omega2(-0.5, -0.5 * sqrt(3.));
        complex<double> x1(-b / 3. / a + p + q), x2(-b / 3. / a + omega1 * p + omega2 * q), x3(-b / 3. / a + omega2 * p + omega1 * q);
        printf("%.10f\n", min(min(sqrt(1 / x1.real()), sqrt(1 / x2.real())), sqrt(1 / x3.real())));
    }
}
\end{lstlisting}

	\section{高斯消元}
		\begin{lstlisting}
vector<double> operator* (const vector<double> &a, double b) {
	vector<double> ret;
	for (int i = 0; i < (int)a.size(); ++i)
		ret.push_back(a[i] * b);
	return ret;
}

vector<double> operator+ (const vector<double> &a, const vector<double> &b) {
	vector<double> ret;
	for (int i = 0; i < (int)a.size(); ++i)
		ret.push_back(a[i] + b[i]);
	return ret;
}

vector<double> operator- (const vector<double> &a, const vector<double> &b) {
	vector<double> ret;
	for (int i = 0; i < (int)a.size(); ++i)
		ret.push_back(a[i] - b[i]);
	return ret;
}

struct solution {
	int size, dimension;
	vector<vector<double> > null_space;
	vector<double> special;
	solution(int size = 0, int dimension = 0) : size(size), dimension(dimension)
			 {
		special = vector<double>(size, 0);
		null_space = vector<vector<double> >(size, vector<double>(dimension,
				 0));
	}
};

solution gauss_elimination(vector<vector<double> > a, vector<double> b) {
	int n = (int)a.size(), m = (int)a[0].size();
	static const int MAX_SIZE = 211;
	int index[MAX_SIZE], row = 0;
	bool pivot[MAX_SIZE];
	fill(index, index + n, -1);
	fill(pivot, pivot + m, false);

	for (int col = 0; row < n && col < m; ++col) {
		int best = row;
		for (int i = row + 1; i < n; ++i)
			if (fabs(a[i][col]) > fabs(a[best][col]))
				best = i;
		swap(a[best], a[row]);
		swap(b[best], b[row]);
		if (fabs(a[row][col]) < EPS)
			continue;
		pivot[col] = true;
		index[row] = col;
		double coef = a[row][col];
		a[row] = a[row] * (1. / coef);
		b[row] = b[row] * (1. / coef);
		for (int i = 0; i < n; ++i)
			if (i != row && fabs(a[i][col]) > EPS) {
				double coef = a[i][col];
				a[i] = a[i] - a[row] * coef;
				b[i] = b[i] - b[row] * coef;
			}
		++row;
	}

	for (int i = row; i < n; ++i)
		if (fabs(b[i]) > EPS)
			return solution(0, 0);					//no solution

	solution ret(m, m - row);
	for (int i = 0; i < row; ++i)
		ret.special[index[i]] = b[i];

	int cnt = 0;
	for (int i = 0; i < m; ++i)
		if (!pivot[i]) {
			for (int j = 0; j < row; ++j)
				ret.null_space[index[j]][cnt] = a[j][i];
			ret.null_space[i][cnt++] = -1;
		}
	return ret;
}
\end{lstlisting}

	\section{最小二乘法}
		\begin{lstlisting}
// calculate argmin ||AX - B||
solution least_squares(vector<vector<double> > a, vector<double> b) {
	int n = (int)a.size(), m = (int)a[0].size();
	vector<vector<double> > p(m, vector<double>(m, 0));
	vector<double> q(m, 0);
	for (int i = 0; i < m; ++i)
		for (int j = 0; j < m; ++j)
			for (int k = 0; k < n; ++k)
				p[i][j] += a[k][i] * a[k][j];
	for (int i = 0; i < m; ++i)
		for (int j = 0; j < n; ++j)
			q[i] += a[j][i] * b[j];
	return gauss_elimination(p, q);
}
\end{lstlisting}

	\section{多项式求根}
		\begin{lstlisting}
const double eps=1e-12;
double a[10][10];
typedef vector<double> vd;
int sgn(double x) { return x < -eps ? -1 : x > eps; }
double mypow(double x,int num){
	double ans=1.0;
	for(int i=1;i<=num;++i)ans*=x;
	return ans;
}
double f(int n,double x){
	double ans=0;
	for(int i=n;i>=0;--i)ans+=a[n][i]*mypow(x,i);
	return ans;
}
double getRoot(int n,double l,double r){
	if(sgn(f(n,l))==0)return l;
	if(sgn(f(n,r))==0)return r;
	double temp;
	if(sgn(f(n,l))>0)temp=-1;else temp=1;
	double m;
	for(int i=1;i<=10000;++i){
		m=(l+r)/2;
		double mid=f(n,m);
		if(sgn(mid)==0){
			return m;
		}
		if(mid*temp<0)l=m;else r=m;
	}
	return (l+r)/2;
}
vd did(int n){
	vd ret;
	if(n==1){
		ret.push_back(-1e10);
		ret.push_back(-a[n][0]/a[n][1]);
		ret.push_back(1e10);
		return ret;
	}
	vd mid=did(n-1);
	ret.push_back(-1e10);
	for(int i=0;i+1<mid.size();++i){
		int t1=sgn(f(n,mid[i])),t2=sgn(f(n,mid[i+1]));
		if(t1*t2>0)continue;
		ret.push_back(getRoot(n,mid[i],mid[i+1]));
	}
	ret.push_back(1e10);
	return ret;
}
int main(){
	int n; scanf("%d",&n);
	for(int i=n;i>=0;--i){
		scanf("%lf",&a[n][i]);
	}
	for(int i=n-1;i>=0;--i)
		for(int j=0;j<=i;++j)a[i][j]=a[i+1][j+1]*(j+1);
	vd ans=did(n);
	sort(ans.begin(),ans.end());
	for(int i=1;i+1<ans.size();++i)printf("%.10f\n",ans[i]);
	return 0;
}
\end{lstlisting}

	\section{自适应辛普森}
	 	\begin{lstlisting}
namespace adaptive_simpson {
	template<typename function>
	inline double area(function f, const double &left, const double &right) {
		double mid = (left + right) / 2;
		return (right - left) * (f(left) + 4 * f(mid) + f(right)) / 6;
	}
	
	template<typename function>
	inline double simpson(function f, const double &left, const double &right, const double &eps, const double &area_sum) {
		double mid = (left + right) / 2;
		double area_left = area(f, left, mid);
		double area_right = area(f, mid, right);
		double area_total = area_left + area_right;
		if (fabs(area_total - area_sum) <= 15 * eps) {
			return area_total + (area_total - area_sum) / 15;
		}
		return simpson(f, left, right, eps / 2, area_left) + simpson(f, mid, right, eps / 2, area_right);
	}
	
	template<typename function>
	inline double simpson(function f, const double &left, const double &right, const double &eps) {
		return simpson(f, left, right, eps, area(f, left, right));
	}
}
\end{lstlisting}

\chapter{计算几何}
	\section{圆与多边形交}
		\begin{lstlisting}
const double eps = 5e-7;
const int N = 2222;
const double pi = acos(-1.0);
int sign(double x) { return x < -eps ? -1 : x > eps; }
double sqr(double x) { return x * x; }
struct Point {
	double x, y;
	Point (double x = 0, double y = 0) : x(x), y(y) {}
	friend inline Point operator +(const Point &a, const Point &b) {
		return Point(a.x + b.x, a.y + b.y);
	}
	friend inline Point operator -(const Point &a, const Point &b) {
		return Point(a.x - b.x, a.y - b.y);
	}
	friend inline Point operator *(const Point &a, double k) {
		return Point(a.x * k, a.y * k);
	}
	friend inline Point operator /(const Point &a, double k) {
		return Point(a.x / k, a.y / k);
	}
	double dist() const { return hypot(x, y); }
	double dist2() const { return x * x + y * y; }
	double ang() const { return atan2(y, x); }
};
vector<Point> convex; 
int n;
double radius;
Point points[N][2];
Point target;
double det(Point a, Point b, Point c) { return (b.x - a.x) * (c.y - a.y) - (c.x - a.x) * (b.y - a.y); }
double dot(Point a, Point b, Point c) { return (b.x - a.x) * (c.x - a.x) + (b.y - a.y) * (c.y - a.y); }
double det(Point a, Point b) { return a.x * b.y - b.x * a.y; }
double dot(Point a, Point b) { return a.x * b.x + a.y * b.y; }
inline bool point_on_line(const Point &a, const Point &b, const Point &c) {
	return sign(det(Point(0, 0), a - b, c - b)) == 0 && dot(Point(0, 0), b - a, c - a) < eps;
}
double point_to_line(const Point &a, const Point &b, const Point &c) {
	return fabs(det(Point(0, 0), c - b, a - b)) / (b - c).dist();
}
Point project_to_line(const Point &p, const Point &a, const Point &b) {
	return a + (b - a) * dot(Point(0, 0), p - a, b - a) / sqr((b - a).dist());
}
Point intersect(Point a, Point b, Point c, Point d) {
	double s1 = det(a, b, c); double s2 = det(a, b, d); return (c * s2 - d * s1) / (s2 - s1);
}
inline Point line_to_circle(const Point &a, const Point &b) {
	double x = sqrt(sqr(radius) - sqr(point_to_line(Point(0, 0), a, b)));
	return project_to_line(Point(0, 0), a, b) - (b - a) / (b - a).dist() * x;
}
inline double area_tri(Point a, Point b) { return det(Point(0, 0), a, b) / 2; }
inline double area_cir(Point a, Point b, double radius) {
	if (sign(det(Point(0, 0), a, b)) == 0) return 0;
	a = a / a.dist() * radius; b = b / b.dist() * radius;
	double d = atan2(det(Point(0, 0), a, b), dot(Point(0, 0), a, b));
	return sqr(radius) * d / 2;
}
int intersect(const Point &a, const Point &b, Point &u, Point &v, double radius) {
	if (point_to_line(Point(0, 0), a, b) + eps > radius) return 0;
	u = line_to_circle(a, b); v = line_to_circle(b, a);
	return point_on_line(u, a, b) + point_on_line(v, a, b); 
}
vector<Point> calc(vector<Point> vec, Point a, Point b) {
	vector<Point> result;
	for(int i = 0; i < (int)vec.size(); i++) {
		Point c = vec[i], d = vec[(i + 1) % (int)vec.size()];
		if (det(a, b, c) > -eps)
			result.push_back(c);
		if (sign(det(a, b, c)) * sign(det(a, b, d)) == -1)
			result.push_back(intersect(a, b, c, d));
	} return result;
}
double areaCT(double R, Point pa, Point pb) {
	if (pa.dist() < pb.dist()) swap(pa, pb);
	if (pb.dist() < eps) return 0;
	Point pc = pb - pa;
	double a = pb.dist(), b = pa.dist(), c = pc.dist();
	double cosB = dot(pb, pc) / a / c, B = acos(cosB);
	double cosC = dot(pa, pb) / a / b, C = acos(cosC);
	double S, h, theta;
	if (a > R) {
		S = C * 0.5 * R * R;
		h = a * b * sin(C) / c;
		if (h < R && B < pi * 0.5) 
			S -= acos(h / R) * R * R - h * sqrt(max(0.0, R * R - h * h));
	} else if (b > R) {
		theta = pi - B - asin(sin(B) / R * a);
		S = 0.5 * a * R * sin(theta) + (C - theta) * 0.5 * R * R;
	} else S = 0.5 * sin(C) * a * b;
	return S;
}
void solve() {
	scanf("%lf%d", &radius, &n);
	convex.clear();
	convex.push_back(Point(-radius, -radius));
	convex.push_back(Point(radius, -radius));
	convex.push_back(Point(radius, radius));
	convex.push_back(Point(-radius, radius));
	for(int i = 1; i <= n; i++)
		scanf("%lf%lf%lf%lf", &points[i][0].x, &points[i][0].y, &points[i][1].x, &points[i][1].y);
	scanf("%lf %lf", &target.x, &target.y);
	for(int i = 1; i <= n; i++) {
		if (det(points[i][0], points[i][1], target) < -eps)
			swap(points[i][0], points[i][1]);
		convex = calc(convex, points[i][0], points[i][1]);
	}
	double ans = 0;
	for(int i = 0; i < (int)convex.size(); i++)
		ans += areaCT(radius, convex[i], convex[(i + 1) % (int)convex.size()]) * sign(det(convex[i], convex[(i + 1) % (int)convex.size()]));
	printf("%.5f", max(0., fabs(ans) / (pi * radius * radius) * 100));
	puts("%");
}
\end{lstlisting}

	\section{动态凸包}
		\begin{lstlisting}
#define x first
#define y second
typedef map<int, int> mii;
typedef map<int, int>::iterator mit;
struct point { // something omitted
	point(const mit &p): x(p->first), y(p->second) {}
};
inline bool checkInside(mii &a, const point &p) { // `border inclusive`
	int x = p.x, y = p.y;
	mit p1 = a.lower_bound(x);
	if (p1 == a.end()) return false;
	if (p1->x == x)	return y <= p1->y;
	if (p1 == a.begin()) return false;
	mit p2(p1--);
	return sign(det(p - point(p1), point(p2) - p)) >= 0;
}
inline void addPoint(mii &a, const point &p) { // `no collinear points`
	int x = p.x, y = p.y;
	mit pnt = a.insert(make_pair(x, y)).first, p1, p2;
	for (pnt->y = y; ; a.erase(p2)) {
		p1 = pnt;
		if (++p1 == a.end())
			break;
		p2 = p1;
		if (++p1 == a.end())
			break;
		if (det(point(p2) - p, point(p1) - p) < 0)
			break;
	}
	for ( ; ; a.erase(p2)) {
		if ((p1 = pnt) == a.begin())
			break;
		if (--p1 == a.begin())
			break;
		p2 = p1--;
		if (det(point(p2) - p, point(p1) - p) > 0)
			break;
	}
}
\end{lstlisting}
`upperHull $\leftarrow (x, y)$`
`lowerHull $\leftarrow (x, -y)$`

	\section{farmland}
		\begin{lstlisting}

const int N = 11111, M = 111111 * 4;

struct eglist {
	int other[M], succ[M], last[M], sum;
	void clear() {
		memset(last, -1, sizeof(last));
		sum = 0;
	}
	void addEdge(int a, int b) {
		other[sum] = b, succ[sum] = last[a], last[a] = sum++;
		other[sum] = a, succ[sum] = last[b], last[b] = sum++;
	}
}e;

int n, m;
struct point {
	int x, y;
	point(int x, int y) : x(x), y(y) {}
	point() {}
	friend point operator -(point a, point b) {
		return point(a.x - b.x, a.y - b.y);
	}
	double arg() {
		return atan2(y, x);
	}
}points[N];

vector<pair<int, double> > vecs;
vector<int> ee[M];
vector<pair<double, pair<int, int> > > edges;
double length[M];
int tot, father[M], next[M], visit[M];

int find(int x) {
	return father[x] == x ? x : father[x] = find(father[x]);
}

long long det(point a, point b) {
	return 1LL * a.x * b.y - 1LL * b.x * a.y;
}

double dist(point a, point b) {
	return sqrt(1.0 * (a.x - b.x) * (a.x - b.x) + 1.0 * (a.y - b.y) * (a.y - b.y));
}

int main() {
	scanf("%d %d", &n, &m);
	e.clear();
	for(int i = 1; i <= n; i++) {
		scanf("%d %d", &points[i].x, &points[i].y);
	}
	for(int i = 1; i <= m; i++) {
		int a, b;
		scanf("%d %d", &a, &b);
		e.addEdge(a, b);
	}	
	for(int x = 1; x <= n; x++) {
		vector<pair<double, int> > pairs;
		for(int i = e.last[x]; ~i; i = e.succ[i]) {
			int y = e.other[i];
			pairs.push_back(make_pair((points[y] - points[x]).arg(), i));
		}
		sort(pairs.begin(), pairs.end());
		for(int i = 0; i < (int)pairs.size(); i++) {
			next[pairs[(i + 1) % (int)pairs.size()].second ^ 1] = pairs[i].second;
		}
	}
	memset(visit, 0, sizeof(visit));
	tot = 0;
	for(int start = 0; start < e.sum; start++) {
		if (visit[start])
			continue;
		long long total = 0;
		int now = start;
		vecs.clear();
		while(!visit[now]) {
			visit[now] = 1;
			total += det(points[e.other[now ^ 1]], points[e.other[now]]);
			vecs.push_back(make_pair(now / 2, dist(points[e.other[now ^ 1]], points[e.other[now]])));
			now = next[now];
		}
		if (now == start && total > 0) {
			++tot;
			for(int i = 0; i < (int)vecs.size(); i++) {
				ee[vecs[i].first].push_back(tot);
			}
		}
	}
	
	for(int i = 0; i < e.sum / 2; i++) {
		int a = 0, b = 0;
		if (ee[i].size() == 0)
			continue;
		else if (ee[i].size() == 1) {
			a = ee[i][0];
		} else if (ee[i].size() == 2) {
			a = ee[i][0], b = ee[i][1];
		}
		edges.push_back(make_pair(dist(points[e.other[i * 2]], points[e.other[i * 2 + 1]]), make_pair(a, b)));
	}
	sort(edges.begin(), edges.end());
	for(int i = 0; i <= tot; i++)
		father[i] = i;
	double ans = 0;
	for(int i = 0; i < (int)edges.size(); i++) {
		int a = edges[i].second.first, b = edges[i].second.second;
		double v = edges[i].first;
		if (find(a) != find(b)) {
			ans += v;
			father[father[a]] = father[b];
		}
	}
	printf("%.5f\n", ans);
}

scanf("%lf %lf %d", &W, &H, &n);
for (int i = 0; i < n; i++) {
	scanf("%lf %lf %lf %lf", &segments[i][0].x, &segments[i][0].y, &segments[i][1].x, &segments[i][1].y);
}
addSegment(Point(0, 0), Point(W, 0));
addSegment(Point(W, 0), Point(W, H));
addSegment(Point(W, H), Point(0, H));
addSegment(Point(0, H), Point(0, 0));

for (int i = 0; i < n; i++) {
	Points.push_back(segments[i][0]);
	Points.push_back(segments[i][1]);
	for (int j = 0; j < i; j++) {
		if (!parallel(segments[i][0], segments[i][1], segments[j][0], segments[j][1])) {
			Point p = intersect(segments[i][0], segments[i][1], segments[j][0], segments[j][1]);
			if (p.on(segments[i][0], segments[i][1]) && p.on(segments[j][0], segments[j][1])) {
				Points.push_back(p);
			}
		}
	}
}
sort(Points.begin(), Points.end());
Points.erase(unique(Points.begin(), Points.end()), Points.end());

e.clear();
for (int i = 0; i < n; i++) {
	vector<pair<double, int> > pairs;
	for (int j = 0; j < Points.size(); j++) {
		if (Points[j].on(segments[i][0], segments[i][1]))
			pairs.push_back(make_pair((Points[j] - segments[i][0]).norm(), j));
	}
	sort(pairs.begin(), pairs.end());
	for (int i = 1; i < pairs.size(); i++) {
		e.addEdge(pairs[i - 1].second, pairs[i].second);
		e.addEdge(pairs[i].second, pairs[i - 1].second);
	}
}

\end{lstlisting}

	\section{farmland完全体}
		\begin{lstlisting}

const int MAXN = 200;
const int MAXV = MAXN * MAXN;
const int MAXE = MAXV * 6;
const double eps = 1e-8;

int sign(double x) {
	return x < -eps ? -1 : x > eps;
}

struct Point {
	double x, y;
	
	Point(int x, int y) : x(x), y(y) {}
	Point() {}
	
	Point &operator +=(const Point &o) {
		x += o.x;
		y += o.y;
		return *this;
	}
	
	Point &operator -=(const Point &o) {
		x -= o.x;
		y -= o.y;
		return *this;
	}
	
	Point &operator *=(double k) {
		x *= k;
		y *= k;
		return *this;
	}
	
	Point &operator /=(double k) {
		x /= k;
		y /= k;
		return *this;
	}
	
	double norm2() const {
		return x * x + y * y;
	}
	
	double norm() const {
		return sqrt(norm2());
	}
	
	double arg() const {
		return atan2(y, x);
	}
	
	bool on(const Point &, const Point &) const;
	bool in(const vector<Point> &) const;
};

bool operator <(const Point &a, const Point &b) {
	return sign(a.x - b.x) < 0 || sign(a.x - b.x) == 0 && sign(a.y - b.y) < 0;
}

bool operator ==(const Point &a, const Point &b) {
	return sign(a.x - b.x) == 0 && sign(a.y - b.y) == 0;
}

Point operator +(Point a, const Point &b) {
	return a += b;
}

Point operator -(Point a, const Point &b) {
	return a -= b;
}

Point operator /(Point a, double k) {
	return a /= k;
}

Point operator *(Point a, double k) {
	return a *= k;
}

Point operator *(double k, Point a) {
	return a *= k;
}

double det(const Point &a, const Point &b) {
	return a.x * b.y - b.x * a.y;
}

double dot(const Point &a, const Point &b) {
	return a.x * b.x + a.y * b.y;
}

bool parallel(const Point &a, const Point &b, const Point &c, const Point &d) {
	return sign(det(b - a, d - c)) == 0;
}

Point intersect(const Point &a, const Point &b, const Point &c, const Point &d) {
	double s1 = det(b - a, c - a);
	double s2 = det(b - a, d - a);
	return (c * s2 - d * s1) / (s2 - s1);
}

bool Point::on(const Point &a, const Point &b) const {
	const Point &p = *this;
	return sign(det(p - a, p - b)) == 0 && sign(dot(p - a, p - b)) <= 0;
} 

bool Point::in(const vector<Point> &polygon) const {
	const Point &p = *this;
	int n = polygon.size();
	int count = 0;
	for (int i = 0; i < n; ++ i) {
		const Point &a = polygon[i];
		const Point &b = polygon[(i + 1) % n];
		if (p.on(a, b)){
			return false;
		}
		int t0 = sign(det(a - p, b - p));
		int t1 = sign(a.y - p.y);
		int t2 = sign(b.y - p.y);
		count += t0 > 0 && t1 <= 0 && t2 > 0;
		count -= t0 < 0 && t2 <= 0 && t1 > 0;
	}
	return count != 0;
}

struct eglist {
	int other[MAXE], succ[MAXE], last[MAXE], sum;
	set<pair<int, int> > Edges; 
	void clear() {
		memset(last, -1, sizeof(last));
		sum = 0;
		Edges.clear();
	}
	void addEdge(int a, int b) {
		if (Edges.count(make_pair(a, b)))
			return;
		Edges.insert(make_pair(a, b));
		other[sum] = b, succ[sum] = last[a], last[a] = sum;
		sum++;
	}
	void _addEdge(int a, int b) {
		addEdge(a, b);
		addEdge(b, a);
	}
}e, topo;

vector<Point> Points;

Point segments[MAXE][2];
double W, H;
int n, next[MAXE];
vector<double> areas, allAreas;
vector<vector<Point> > regions;

void addSegment(Point a, Point b) {
	segments[n][0] = a;
	segments[n][1] = b;
	n++;
}

int getPointID(const Point &p) {
	return lower_bound(Points.begin(), Points.end(), p) - Points.begin();
}

const int VERTEX = 0;
const int EDGE = 1;
const int REGION = 2;

int getID(int type, int id) {
	if (type == VERTEX) {
		return id;
	}
	if (type == EDGE) {
		return id + Points.size();
	}
	if (type == REGION) {
		return id + Points.size() + e.sum / 2;
	}
	assert(false);
}

double getArea(int id) {
	id -= Points.size() + e.sum / 2;
	return id < 0 ? 0 : areas[id];
}

int locate(const Point &p) {
	for (int i = 0; i < e.sum; i += 2) {
		if (p.on(Points[e.other[i]], Points[e.other[i ^ 1]])) {
			return getID(EDGE, i >> 1);
		}
	}
	int best = -1;
	for (int i = 0; i < regions.size(); ++i) {
		if (p.in(regions[i]) && (best == -1 || allAreas[best] > allAreas[i])) {
			best = i;
		}
	}
	return getID(REGION, best);
}

vector<string> colorNames;
map<string, int> colorIDs;

int getColorID(const char *color) {
	if (!colorIDs.count(color)) {
		colorNames.push_back(color);
		int newID = colorIDs.size();
		colorIDs[color] = newID;
	}
	return colorIDs[color];
}

int color[MAXV * 10];

void paint(const Point &p, const char * c) {
	int start = locate(p);
	int old = color[start];
	int cid = getColorID(c);
	if (old == cid)
		return;
	queue<int> q;
	q.push(start);
	color[start] = cid;
	while(!q.empty()) {
		int x = q.front();
		q.pop();
		for (int i = topo.last[x]; ~i; i = topo.succ[i]) {
			int y = topo.other[i];
			if (color[y] == old) {
				color[y] = cid;
				q.push(y);
			}
		}
	}
}

int main() {
	freopen("input.txt", "r", stdin);
	//freopen("output.txt", "w", stdout);
	scanf("%lf %lf %d", &W, &H, &n);
	for (int i = 0; i < n; i++) {
		scanf("%lf %lf %lf %lf", &segments[i][0].x, &segments[i][0].y, &segments[i][1].x, &segments[i][1].y);
	}
	addSegment(Point(0, 0), Point(W, 0));
	addSegment(Point(W, 0), Point(W, H));
	addSegment(Point(W, H), Point(0, H));
	addSegment(Point(0, H), Point(0, 0));
	
	for (int i = 0; i < n; i++) {
		Points.push_back(segments[i][0]);
		Points.push_back(segments[i][1]);
		for (int j = 0; j < i; j++) {
			if (!parallel(segments[i][0], segments[i][1], segments[j][0], segments[j][1])) {
				Point p = intersect(segments[i][0], segments[i][1], segments[j][0], segments[j][1]);
				if (p.on(segments[i][0], segments[i][1]) && p.on(segments[j][0], segments[j][1])) {
					Points.push_back(p);
				}
			}
		}
	}
	sort(Points.begin(), Points.end());
	Points.erase(unique(Points.begin(), Points.end()), Points.end());
	
	e.clear();
	for (int i = 0; i < n; i++) {
		vector<pair<double, int> > pairs;
		for (int j = 0; j < Points.size(); j++) {
			if (Points[j].on(segments[i][0], segments[i][1]))
				pairs.push_back(make_pair((Points[j] - segments[i][0]).norm(), j));
		}
		sort(pairs.begin(), pairs.end());
		for (int i = 1; i < pairs.size(); i++) {
			e.addEdge(pairs[i - 1].second, pairs[i].second);
			e.addEdge(pairs[i].second, pairs[i - 1].second);
		}
	}
	
	for (int u = 0; u < Points.size(); u++) {
		vector<pair<double, int> > pairs;
		for (int iter = e.last[u]; ~iter; iter = e.succ[iter]) {
			pairs.push_back(make_pair((Points[e.other[iter]] - Points[u]).arg(), iter));
		}
		sort(pairs.begin(), pairs.end());
		for(int i = 0; i < pairs.size(); i++) {
			next[pairs[(i + 1) % pairs.size()].second ^ 1] = pairs[i].second;
		}
	}
	
	vector<pair<Point, double> > waits;
	static bool visit[MAXV];
	memset(visit, 0, sizeof(visit));
	for (int start = 0; start < e.sum; ++start) {
		if (!visit[start]) {
			int v = start;
			double totalArea = 0;
			vector <Point> region;
			for(; !visit[v]; v = next[v]) {
				visit[v] = true;
				totalArea += det(Points[e.other[v ^ 1]], Points[e.other[v]]);
				region.push_back(Points[e.other[v]]);
			}
			
			if (sign(totalArea) > 0) {
				regions.push_back(region);
				areas.push_back(totalArea);
				allAreas.push_back(totalArea);
			} else {
				waits.push_back(make_pair(region.front(), -totalArea));
			}
		}
	}
	
	//build
	topo.clear();
	for (int i = 0; i < e.sum; i++) {
		topo._addEdge(getID(EDGE, i >> 1), getID(VERTEX, e.other[i]));
	}
	for (int i = 0; i < regions.size(); i++) {
		topo._addEdge(getID(REGION, i), getID(VERTEX, getPointID(regions[i].front())));
	}
	for (int iter = 0; iter < waits.size(); iter++) {
		const Point &p = waits[iter].first;
		int best = -1;
		for (int i = 0; i < regions.size(); i++) {
			if (p.in(regions[i]) && (best == -1 || allAreas[best] > allAreas[i])) {
				best = i;
			}
		}
		if (best != -1) {
			areas[best] -= waits[iter].second;
			topo._addEdge(getID(REGION, best), getID(VERTEX, getPointID(p)));
		}
	}
	
	
	getColorID("white");
	getColorID("blake");
	getColorID("__COLOR__");
	
	for (int i = 0; i < regions.size(); i++) {
		color[getID(REGION, i)] = getColorID("white");
	}
	for (int i = 0; i < Points.size(); i++) {
		color[getID(VERTEX, i)] = getColorID("black");
	}
	for(int i = 0; i < e.sum / 2; i++) {
		color[getID(EDGE, i)] = getColorID("black");
	}
	paint(Point(0, 0), "__COLOR__");
	int m;
	scanf("%d", &m);
	while (m --) {
		Point p;
		char buffer[16];
		scanf("%lf %lf %s", &p.x, &p.y, buffer);
		paint(p, buffer);
	}
	
	map<string, double> answer;
	for (int i = 0; i < Points.size() + (e.sum >> 1) + regions.size(); ++i) {
		const string &name = colorNames[color[i]];
		if (name != "__COLOR__") {
			answer[name] += getArea(i);
		}
	}
	for (map<string, double> :: iterator iter = answer.begin(); iter != answer.end(); ++ iter) {
		printf("%s %.8lf\n", iter->first.c_str(), 0.5 * iter->second);
	}
}
	\end{lstlisting}

	\section{半平面交}
		\begin{lstlisting}
struct Border {
	point p1, p2; double alpha;
	Border() : p1(), p2(), alpha(0.0) {}
	Border(const point &a, const point &b): p1(a), p2(b), alpha( atan2(p2.y - p1.y, p2.x - p1.x) ) {}
	bool operator == (const Border &b) const { 
		return sign(alpha - b.alpha) == 0; 
	}
	bool operator < (const Border &b) const {
		int c = sign(alpha - b.alpha); if (c != 0) return c > 0;
		return sign(det(b.p2 - b.p1, p1 - b.p1)) >= 0;
	}
};
point isBorder(const Border &a, const Border &b) { // a and b should not be parallel
	point is; 
	lineIntersect(a.p1, a.p2, b.p1, b.p2, is); 
	return is;
}
bool checkBorder(const Border &a, const Border &b, const Border &me) {
	point is; 
	lineIntersect(a.p1, a.p2, b.p1, b.p2, is);
	return sign(det(me.p2 - me.p1, is - me.p1)) > 0;
}
double HPI(int N, Border border[]) {
	static Border que[MAXN * 2 + 1]; static point ps[MAXN];
	int head = 0, tail = 0, cnt = 0; // [head, tail)
	sort(border, border + N); 
	N = unique(border, border + N) - border;
	for (int i = 0; i < N; ++i) {
		Border &cur = border[i];
		while (head + 1 < tail && !checkBorder(que[tail - 2], que[tail - 1], cur)) 
			--tail;
		while (head + 1 < tail && !checkBorder(que[head], que[head + 1], cur)) 
			++head;
		que[tail++] = cur;
	}
	while (head + 1 < tail && !checkBorder(que[tail - 2], que[tail - 1], que[head])) 
		--tail;
	while (head + 1 < tail && !checkBorder(que[head], que[head + 1], que[tail - 1])) 
		++head;
	if (tail - head <= 2) 
		return 0.0;
	//Foru(i, a, b) : a <= i < b
	Foru(i, head, tail) 
		ps[cnt++] = isBorder(que[i], que[(i + 1 == tail) ? (head) : (i + 1)]);
	double area = 0; 
	Foru(i, 0, cnt) 
		area += det(ps[i], ps[(i + 1) % cnt]);
	return fabs(area * 0.5); // or (-area * 0.5)
}
\end{lstlisting}

	\section{三维绕轴旋转}
		\begin{lstlisting}
const double pi = acos(-1.0);
int n, m; char ch1; bool flag;
double a[4][4], s1, s2, x, y, z, w, b[4][4], c[4][4];
double sqr(double x)
{
	return x*x;
}
int main()
{
	scanf("%d\n", &n);
	memset(b, 0, sizeof(b));
	b[0][0] = b[1][1] = b[2][2] = b[3][3] = 1;//initial matrix
	for(int i = 1; i <= n; i++)
	{
		scanf("%c", &ch1);
		if(ch1 == 'T')
		{
			scanf("%lf %lf %lf\n", &x, &y, &z);//plus each coordinate by a number (x, y, z)
			memset(a, 0, sizeof(a));
			a[0][0] = 1; a[3][0] = x;
			a[1][1] = 1; a[3][1] = y;
			a[2][2] = 1; a[3][2] = z;
			a[3][3] = 1;
		}else if(ch1 == 'S')
		{
			scanf("%lf %lf %lf\n", &x, &y, &z);//multiply each coordinate by a number (x, y, z)
			memset(a, 0, sizeof(a));
			a[0][0] = x;
			a[1][1] = y;
			a[2][2] = z;
			a[3][3] = 1;
		}else
		{
			scanf("%lf %lf %lf %lf\n", &x, &y, &z, &w);
			//大拇指指向x轴正方向时, 4指弯曲由y轴正方向指向z轴正方向
			//大拇指沿着原点到点(x, y, z)的向量, 4指弯曲方向旋转w度
			w = w*pi/180;
			memset(a, 0, sizeof(a));
			s1 = x*x+y*y+z*z;
			a[3][3] = 1;
			a[0][0] = ((y*y+z*z)*cos(w)+x*x)/s1;			a[0][1] = x*y*(1-cos(w))/s1+z*sin(w)/sqrt(s1);	a[0][2] = x*z*(1-cos(w))/s1-y*sin(w)/sqrt(s1);
			a[1][0] = x*y*(1-cos(w))/s1-z*sin(w)/sqrt(s1);	a[1][1] = ((x*x+z*z)*cos(w)+y*y)/s1;			a[1][2] = y*z*(1-cos(w))/s1+x*sin(w)/sqrt(s1);
			a[2][0] = x*z*(1-cos(w))/s1+y*sin(w)/sqrt(s1);	a[2][1] = y*z*(1-cos(w))/s1-x*sin(w)/sqrt(s1);	a[2][2] = ((x*x+y*y)*cos(w)+z*z)/s1;
		}
		memset(c, 0, sizeof(c));
		for(int i = 0; i < 4; i++)
			for(int j = 0; j < 4; j++)
				for(int k = 0; k < 4; k++)
					c[i][j] += b[i][k]*a[k][j];
		memcpy(b, c, sizeof(c));
	}
	scanf("%d", &m);
	for(int i = 1; i <= m; i++)
	{
		scanf("%lf%lf%lf", &x, &y, &z);//initial vector
		printf("%lf %lf %lf\n", x*b[0][0]+y*b[1][0]+z*b[2][0]+b[3][0], x*b[0][1]+y*b[1][1]+z*b[2][1]+b[3][1], x*b[0][2]+y*b[1][2]+z*b[2][2]+b[3][2]);
	}
	return 0;
}
\end{lstlisting}

	\section{点到凸包切线}???
	%!!!
	\section{直线凸包交点}
		\begin{lstlisting}
int n;
double eps(1e-8);
int sign(const double & x) {
	return (x > eps) - (x + eps < 0);
}
struct Point {
	double x, y;
	void scan() {
		scanf("%lf%lf", &x, &y);
	}
	void print() {
		printf("%lf %lf\n", x, y);
	}
	Point() {
	}
	Point(const double & x, const double & y) : x(x), y(y) {
	}
};
Point operator + (const Point & a, const Point & b) {
	return Point(a.x + b.x, a.y + b.y);
}
Point operator - (const Point & a, const Point & b) {
	return Point(a.x - b.x, a.y - b.y);
}
Point operator * (const double & a, const Point & b) {
	return Point(a * b.x, a * b.y);
}
double operator * (const Point & a, const Point & b) {
	return a.x * b.y - a.y * b.x;
}
bool isUpper(const Point & a) {
	return sign(a.x) < 0 or sign(a.x) == 0 and sign(a.y > 0);
}
Point crs(const Point & as, const Point & at, const Point & bs, const Point & bt) {
	if(sign((at - as) * (bt - bs)) == 0) {
		return bs;
	}
	double lambda((bs - as) * (bt - bs) / ((at - as) * (bt - bs)));
	return as + lambda * (at - as);
}
struct reca {
	Point a[50000];
	double s[50000];
	Point & operator [] (int x) {
		assert(x % n < 50000);
		return a[x % n];
	}
	void init() {
		s[0] = a[0] * a[1];
		for(int i(1); i < n; i++) {
			s[i] = s[i - 1] + a[i] * (i == n - 1?a[0]:a[i + 1]);
		}
	}

	double getS(int le, int ri) {
		if(le > ri)
			return 0;
		le %= n;
		ri %= n;
		if(le <= ri) {
			return s[ri] - (le?s[le - 1]:0);
		}else {
			return getS(le, n - 1) + getS(0, ri);
		}
	}
} a;

int lowerBound(int le, int ri, const Point & dir) {
	while(le < ri) {
		int mid((le + ri) / 2);
		if(sign((a[mid + 1] - a[mid]) * dir) >= 0) {
			le = mid + 1;
		}else {
			ri = mid;
		}
	}
	return le;
}
int boundLower(int le, int ri, const Point & s, const Point & t) {
	while(le < ri) {
		int mid((le + ri + 1) / 2);
		if(sign((a[mid] - s) * (t - s)) >= 0) {
			le = mid;
		}else {
			ri = mid - 1;
		}
	}
	return le;
}
bool check(const Point & a, const Point & b, const Point & c, const Point & d) {
	return sign((a - c) * (d - c)) * sign((b - c) * (d - c)) <= 0;
}
bool f[55555];
int main() {
	scanf("%d", &n);
	for(int i(0); i < n; i++) {
		//printf("%d\n", n);
		a[i].scan();
		//return 0;
	}
	//return 0;
	for(int i(0); i < n; i++) {
		int d(sign((a[i + 1] - a[i]) * (a[i + 2] - a[i + 1])));
		if(d) {
			if(d < 0) {
				reverse(a.a, a.a + n);
			}
			break;
		}
	}
	for(int i(0); i < n; i++) {
		if(!sign(a[i].x - a[i + 1].x) and !sign(a[i].y - a[i + 1].y)) {
			f[i] = false;
		}else {
			f[i] = true;
		}
	}
	int n1(0);
	for(int i(0); i < n; i++) {
		if(f[i]) {
			a[n1++] = a[i];
		}
	}
	n = n1;
	//现在a必须是严格逆时针凸包
	a.init();
	int i1, j1;
	for(int i(0); i < n; i++) {
		if(isUpper(a[i + 1] - a[i])) {
			for(int j(i + 1); j != i; ++j %= n) {
				if(!isUpper(a[j + 1] - a[j])) {
					i1 = i; j1 = j;
					break;
				}
			}
			break;
		}
	}
	if(i1 > j1) {
		j1 += n;
	}
	int m;
	scanf("%d", &m);
	for(int i(0); i < m; i++) {
		Point s, t;
		s.scan(); t.scan();
		if(!isUpper(t - s)) {
			swap(t, s);
		}
		int i3(lowerBound(i1, j1, t - s));
		int j3(lowerBound(j1, i1 + n, s - t));
		int i4(boundLower(i3, j3, s, t));
		int j4(boundLower(j3, i3 + n, t, s));
		if(check(a[i4], a[i4 + 1], s, t)) {
			Point p1(crs(s, t, a[i4], a[i4 + 1]));
			Point p2(crs(s, t, a[j4], a[j4 + 1]));
			if(sign(p1.x - p2.x) or sign(p1.y - p2.y)) {
				assert(i4 % n != j4 % n);
				double area1(p1 * a[i4 + 1] + a.getS(i4 + 1, j4 - 1) + a[j4] * p2 + p2 * p1);
				double area2(p2 * a[j4 + 1] + a.getS(j4 + 1, i4 + n - 1) + a[i4] * p1 + p1 * p2);
				printf("%.6f\n", min(fabs(area1), fabs(area2)) / 2);
			}else {
				printf("0.000000\n");
			}
		}else {
			printf("0.000000\n");
		}
	}
}
\end{lstlisting}

	\section{exhausted\_robot 凸多边形卡壳+凸多边形交}
		\begin{lstlisting}
double eps(1e-8);
int sign(const double & x) {
	return (x > eps) - (x + eps < 0);
}
bool equal(const double & x, const double & y) {
	return x + eps > y and y + eps > x;
}
struct Point {
	double x, y;
	Point () {
	}
	Point(const double & x, const double & y) : x(x), y(y) {
	}
	void scan() {
		scanf("%lf%lf", &x, &y);
	}
	double sqrlen() const {
		return x * x + y * y;
	}
	double len() const  {
		return sqrt(sqrlen());
	}
	Point zoom(const double & l) const {
		double lambda(l / len());
		return Point(lambda * x, lambda * y);
	}
	Point rev() const {
		return Point(-y, x);
	}
	void print() const {
		printf("(%f %f)\n", x, y);
	}
};

vector<Point> blocks[22], denied[22], robot;

vector<pair<double, int> > vec;

bool f[111];

Point operator - (const Point & a, const Point & b) {
	return Point(a.x - b.x, a.y - b.y);
}
Point operator + (const Point & a, const Point & b) {
	return Point(a.x + b.x, a.y + b.y);
}
Point operator * (const double & a, const Point & b) {
	return Point(a * b.x, a * b.y);
}
double operator * (const Point & a, const Point & b) {
	return a.x * b.y - a.y * b.x;
}
double operator % (const Point & a, const Point & b) {
	return a.x * b.x + a.y * b.y;
}

bool operator < (const Point & a, const Point & b) {
	if(!equal(a.x, b.x))
		return a.x < b.x;
	else if(!equal(a.y, b.y));
		return a.y < b.y;
	return false;
}
bool operator == (const Point & a, const Point & b) {
	return equal(a.x, b.x) and equal(a.y, b.y);
}

void scan(vector<Point> & vec) {
	vec.clear();
	int x;
	scanf("%d", &x);
	for(int i(0); i < x; i++) {
		Point tmp;
		tmp.scan();
		vec.push_back(tmp);
	}
}

Point intersect(const Point & as, const Point & ad, const Point & bs, const Point & bd) {
	double lambda((bs - as) * bd / (ad * bd));
	return as + lambda * ad;
}

void cut(vector<Point> & vec, const Point & s, const Point & d) {
	vector<Point> vec1;
	for(int i(0); i < (int)vec.size(); i++) {
		if(sign((vec[i] - s) * d) <= 0) {
			vec1.push_back(vec[i]);
		}
		if(sign((vec[i] - s) * d) * sign((vec[(i + 1) % (int)vec.size()] - s) * d) < 0) {
			vec1.push_back(intersect(s, d, vec[i], vec[(i + 1) % (int)vec.size()] - vec[i]));
		}
	}
	vec = vec1;
}

int mi;

Point getMax(const Point & norm) {
	Point res(robot[0]);
	mi = 0;
	for(int i(0); i < (int)robot.size(); i++) {
		if(sign(robot[i] % norm - res % norm) > 0) {
			res = robot[i];
			mi = i;
		}
	}
	return res;
}

bool vecCmp(const pair<double, int> & a, const pair<double, int> & b) {
	if(!equal(a.first, b.first))
		return a.first < b.first;
	else
		return a.second > b.second;
}

bool vecEql(const pair<double, int> & a, const pair<double, int> & b) {
	return equal(a.first, b.first) and a.second == b.second;
}

void print(const vector<Point> & vec) {
	printf("print:\n");
	for(int i(0); i < (int)vec.size(); i++) {
		vec[i].print();
	}
	printf("endprint\n");
}

void getConvex(vector<Point> & vec) {
	sort(vec.begin(), vec.end());
	vector<Point> vec1;
	for(int i(0); i < (int)vec.size(); i++) {
		while(vec1.size() >= 2 and sign((vec1.back() - vec1[(int)vec1.size() - 2]) * (vec[i] - vec1.back())) <= 0)
			vec1.pop_back();
		vec1.push_back(vec[i]);
	}
	vector<Point> vec2;
	for(int i((int)vec.size() - 1); i >= 0; i--) {
		while(vec2.size() >= 2 and sign((vec2.back() - vec2[(int)vec2.size() - 2]) * (vec[i] - vec2.back())) <= 0)
			vec2.pop_back();
		vec2.push_back(vec[i]);
	}
	vec.clear();
	for(int i(0); i + 1 < (int)vec1.size(); i++)
		vec.push_back(vec1[i]);
	for(int i(0); i + 1 < (int)vec2.size(); i++)
		vec.push_back(vec2[i]);
}

int main() {
	int tst;
	scanf("%d", &tst);
	for(int qq(1); qq <= tst; qq++) {
		int n;
		scanf("%d", &n);
		for(int i(0); i < n; i++)
			scan(blocks[i]);
		scan(robot);
		double x1, y1, x2, y2;
		scanf("%lf%lf%lf%lf", &x1, &y1, &x2, &y2);
		x1 += robot[0].x - getMax(Point(-1, 0)).x;
		y1 += robot[0].y - getMax(Point(0, -1)).y;
		x2 -= getMax(Point(1, 0)).x - robot[0].x;
		y2 -= getMax(Point(0, 1)).y - robot[0].y;
		double ans((x2 - x1) * (y2 - y1));
		for(int i(0); i < n; i++) {
			int siz(blocks[i].size());
			denied[i].clear();
			int p1, p2;
			p1 = 0;
			getMax((blocks[i][1] - blocks[i][0]).rev());
			p2 = mi;
			denied[i].push_back(blocks[i][0] + robot[0] - robot[mi]);
			for(int j1(1), j2(mi); j1 != p1 or j2 != p2; ) {
				denied[i].push_back(blocks[i][j1] + robot[0] - robot[j2]);
				Point dir((blocks[i][(j1 + 1) % (int)blocks[i].size()] - blocks[i][j1]).rev());
				getMax(dir);
				if(equal(robot[j2] % dir, robot[mi] % dir))
					++j1 %= (int)blocks[i].size();
				else
					++j2 %= (int)robot.size();
			}
		}
		for(int i(0); i < n; i++) {
			cut(denied[i], Point(x1, y1), Point(x2 - x1, 0));
			cut(denied[i], Point(x2, y1), Point(0, y2 - y1));
			cut(denied[i], Point(x2, y2), Point(x1 - x2, 0));
			cut(denied[i], Point(x1, y2), Point(0, y1 - y2));
			for(int j(0); j < (int)denied[i].size(); j++) {
				f[j] = !(denied[i][j] == denied[i][(j + 1) % (int)denied[i].size()]);
			}
			getConvex(denied[i]);
			denied[i].push_back(denied[i].front());
		}
		for(int i(0); i < n; i++) {
			for(int j(0); j + 1 < (int)denied[i].size(); j++) {
				vec.clear();
				vec.push_back(make_pair(0., 0));
				vec.push_back(make_pair(1., 0));
				Point norm(denied[i][j + 1] - denied[i][j]);
				Point a(denied[i][j]), b(denied[i][j + 1]);
				norm = norm.zoom(1 / norm.len());
				for(int k(0); k < n; k++) if(k != i) {
					int sz(vec.size());
					for(int l(0); l + 1 < (int)denied[k].size(); l++) {
						Point c(denied[k][l]), d(denied[k][l + 1]);
						int s1(sign((c - a) * norm));
						int s2(sign((d - a) * norm));
						if(!s1 and !s2 and k < i and sign((d - c) % norm) > 0) {
							vec.push_back(make_pair((c - a) % norm, 1));
							vec.push_back(make_pair((d - a) % norm, -1));
						} else if(s1 <= 0 and s2 > 0 or s1 > 0 and s2 <= 0) {
							double a1((d - c) * (a - c));
							double a2((d - c) * (b - c));
							vec.push_back(make_pair(a1 / (a1 - a2), (s1 < 0 or s2 > 0)?1:-1));
						}
					}
				}
				sort(vec.begin(), vec.end(), vecCmp);
				int cnt(0);
				double tot(0);
				for(int k(0); k + 1 < (int)vec.size(); k++) {
					cnt += vec[k].second;
					if(cnt == 0 and sign(vec[k].first) >= 0 and sign(vec[k + 1].first - 1) <= 0) {
						tot += vec[k + 1].first - vec[k].first;
					}
				}
				ans -= tot * (denied[i][j] * denied[i][j + 1]) / 2;
			}
		}
		printf("Case #%d: %.3f\n", qq, ans);
	}
}
\end{lstlisting}

	\section{判断圆存在交集O(nlogk)}
		传入n个圆,圆心存在cir中,半径存在radius中,nlogk判断是否存在交集
\begin{lstlisting}
int n;
double sx, sy, d;
vector<Point> cir;
vector<double> radius;

int isIntersectCircleToCircle(Point c1, double r1, Point c2, double r2)
{
	double dis = c1.distTo(c2);
	return sign(dis - (r1 + r2)) <= 0;
}

void getRange(double x, Point &c, double r, double &retl, double &retr)
{
	double tmp = sqrt(max(r * r - (c.x - x) * (c.x - x), 0.0));
	retl = c.y - tmp; retr = c.y + tmp;
}

int checkInLine(double x)
{
	double minR = INF, maxL = -INF;
	double tmpl, tmpr;
	for(int i = 0; i < n; ++ i) {
		if (sign(cir[i].x + radius[i] - x) < 0 || sign(cir[i].x - radius[i] - x) > 0) 
			return false;
		getRange(x, cir[i], radius[i], tmpl, tmpr);
		maxL = max(tmpl, maxL);
		minR = min(tmpr, minR);
		if (maxL > minR) return false;
	}
	return true;
}

int shouldGoLeft(double x)
{
	if (checkInLine(x)) return 2;
	int onL = 0, onR = 0;
	for(int i = 0; i < n; ++ i) {
		if (sign(cir[i].x + radius[i] - x) < 0) onL = 1;
		if (sign(cir[i].x - radius[i] - x) > 0) onR = 1;
	}
	if (onL && onR) return -1;
	if (onL) return 1;
	if (onR) return 0;

	double minR = INF, maxL = -INF, tmpl, tmpr;
	int idMinR, idMaxL;

	for(int i = 0; i < n; ++ i) {
		getRange(x, cir[i], radius[i], tmpl, tmpr);
		if (tmpr < minR) {
			minR = tmpr;
			idMinR = i;
		}
		if (tmpl > maxL) {
			maxL = tmpl;
			idMaxL = i;
		}
	}
	if (! isIntersectCircleToCircle(cir[idMinR], radius[idMinR], cir[idMaxL], radius[idMaxL])) 
		return -1;
	Point p1, p2;
	intersectionCircleToCircle(cir[idMinR], radius[idMinR], cir[idMaxL], radius[idMaxL], p1, p2); 
	return (p1.x < x);
}

int hasIntersectionCircles()
{
	double l = -INF, r = INF, mid;
	for(int i = 0; i < 100; ++ i) {
		mid = (l + r) * 0.5;
		int tmp = shouldGoLeft(mid);
		if (tmp < 0) return 0;
		if (tmp == 2) return 1;
		if (tmp) r = mid;
		else l = mid;
	}
	mid = (l + r) * 0.5;
	return checkInLine(mid);
}
\end{lstlisting}

	\section{最小覆盖球}
		\begin{lstlisting}
double eps(1e-8);
int sign(const double & x) {
	return (x > eps) - (x + eps < 0);
}
bool equal(const double & x, const double & y) {
	return x + eps > y and y + eps > x;
}
struct Point {
	double x, y, z;
	Point() {
	}
	Point(const double & x, const double & y, const double & z) : x(x), y(y), z(z){
	}
	void scan() {
		scanf("%lf%lf%lf", &x, &y, &z);
	}
	double sqrlen() const {
		return x * x + y * y + z * z;
	}
	double len() const {
		return sqrt(sqrlen());
	}
	void print() const {
		printf("(%lf %lf %lf)\n", x, y, z);
	}
} a[33];
Point operator + (const Point & a, const Point & b) {
	return Point(a.x + b.x, a.y + b.y, a.z + b.z);
}
Point operator - (const Point & a, const Point & b) {
	return Point(a.x - b.x, a.y - b.y, a.z - b.z);
}
Point operator * (const double & x, const Point & a) {
	return Point(x * a.x, x * a.y, x * a.z);
}
double operator % (const Point & a, const Point & b) {
	return a.x * b.x + a.y * b.y + a.z * b.z;
}
Point operator * (const Point & a, const Point & b) {
	return Point(a.y * b.z - a.z * b.y, a.z * b.x - a.x * b.z, a.x * b.y - a.y * b.x);
}
struct Circle {
	double r;
	Point o;
	Circle() {
		o.x = o.y = o.z = r = 0;
	}
	Circle(const Point & o, const double & r) : o(o), r(r) {
	}
	void scan() {
		o.scan();
		scanf("%lf", &r);
	}
	void print() const {
		o.print();
		printf("%lf\n", r);
	}
};
struct Plane {
	Point nor;
	double m;
	Plane(const Point & nor, const Point & a) : nor(nor){
		m = nor % a;
	}
};
Point intersect(const Plane & a, const Plane & b, const Plane & c) {
	Point c1(a.nor.x, b.nor.x, c.nor.x), c2(a.nor.y, b.nor.y, c.nor.y), c3(a.nor.z, b.nor.z, c.nor.z), c4(a.m, b.m, c.m);
	return 1 / ((c1 * c2) % c3) * Point((c4 * c2) % c3, (c1 * c4) % c3, (c1 * c2) % c4);
}
bool in(const Point & a, const Circle & b) {
	return sign((a - b.o).len() - b.r) <= 0; 
}	
bool operator < (const Point & a, const Point & b) {
	if(!equal(a.x, b.x)) {
		return a.x < b.x;
	}
	if(!equal(a.y, b.y)) {
		return a.y < b.y;
	}
	if(!equal(a.z, b.z)) {
		return a.z < b.z;
	}
	return false;
}
bool operator == (const Point & a, const Point & b) {
	return equal(a.x, b.x) and equal(a.y, b.y) and equal(a.z, b.z);
}
vector<Point> vec;
Circle calc() {
	if(vec.empty()) {
		return Circle(Point(0, 0, 0), 0);
	}else if(1 == (int)vec.size()) {
		return Circle(vec[0], 0);
	}else if(2 == (int)vec.size()) {
		return Circle(0.5 * (vec[0] + vec[1]), 0.5 * (vec[0] - vec[1]).len());
	}else if(3 == (int)vec.size()) {
		double r((vec[0] - vec[1]).len() * (vec[1] - vec[2]).len() * (vec[2] - vec[0]).len() / 2 / fabs(((vec[0] - vec[2]) * (vec[1] - vec[2])).len()));
		return Circle(intersect(Plane(vec[1] - vec[0], 0.5 * (vec[1] + vec[0])),
				       	Plane(vec[2] - vec[1], 0.5 * (vec[2] + vec[1])),
					Plane((vec[1] - vec[0]) * (vec[2] - vec[0]), vec[0])), r);
	}else {
		Point o(intersect(Plane(vec[1] - vec[0], 0.5 * (vec[1] + vec[0])),
				  Plane(vec[2] - vec[0], 0.5 * (vec[2] + vec[0])),
				  Plane(vec[3] - vec[0], 0.5 * (vec[3] + vec[0]))));
		return Circle(o, (o - vec[0]).len());
	}
}
Circle miniBall(int n) {
	Circle res(calc());
	for(int i(0); i < n; i++) {
		if(!in(a[i], res)) {
			vec.push_back(a[i]);
			res = miniBall(i);
			vec.pop_back();
			if(i) {
				Point tmp(a[i]);
				memmove(a + 1, a, sizeof(Point) * i);
				a[0] = tmp;
			}
		}
	}
	return res;
}
int main() {
	int n;
	for(;;) {
		scanf("%d", &n);
		if(!n) {
			break;
		}
		for(int i(0); i < n; i++) {
			a[i].scan();
		}
		sort(a, a + n);
		n = unique(a, a + n) - a;
		vec.clear();
		printf("%.10f\n", miniBall(n).r);
	}
}
\end{lstlisting}

	\section{最小覆盖圆}
		\begin{lstlisting}
#include<cmath>
#include<cstdio>
#include<algorithm>
using namespace std;
const double eps=1e-6;
struct couple
{
	double x, y;
	couple(){}
	couple(const double &xx, const double &yy)
	{
		x = xx; y = yy;
	}
} a[100001];
int n;
bool operator < (const couple & a, const couple & b)
{
	return a.x < b.x - eps or (abs(a.x - b.x) < eps and a.y < b.y - eps);
}
bool operator == (const couple & a, const couple & b)
{
	return !(a < b) and !(b < a);
}
inline couple operator - (const couple &a, const couple &b)
{	
	return couple(a.x-b.x, a.y-b.y);
}
inline couple operator + (const couple &a, const couple &b)
{
	return couple(a.x+b.x, a.y+b.y);
}
inline couple operator * (const couple &a, const double &b)
{
	return couple(a.x*b, a.y*b);
}
inline couple operator / (const couple &a, const double &b)
{
	return a*(1/b);
}
inline double operator * (const couple &a, const couple &b)
{
	return a.x*b.y-a.y*b.x;
}
inline double len(const couple &a)
{
	return a.x*a.x+a.y*a.y;
}
inline double di2(const couple &a, const couple &b)
{
	return (a.x-b.x)*(a.x-b.x)+(a.y-b.y)*(a.y-b.y);
}
inline double dis(const couple &a, const couple &b)
{
	return sqrt((a.x-b.x)*(a.x-b.x)+(a.y-b.y)*(a.y-b.y));
}
struct circle
{
	double r; couple c;
} cir;
inline bool inside(const couple & x)
{
	return di2(x, cir.c) < cir.r*cir.r+eps;
}
inline void p2c(int x, int y)
{
	cir.c.x = (a[x].x+a[y].x)/2;
	cir.c.y = (a[x].y+a[y].y)/2;
	cir.r = dis(cir.c, a[x]);
}
inline void p3c(int i, int j, int k)
{
	couple x = a[i], y = a[j], z = a[k];
	cir.r = sqrt(di2(x,y)*di2(y,z)*di2(z,x))/fabs(x*y+y*z+z*x)/2;
	couple t1((x-y).x, (y-z).x), t2((x-y).y, (y-z).y), t3((len(x)-len(y))/2, (len(y)-len(z))/2);
	cir.c = couple(t3*t2, t1*t3)/(t1*t2);
}
inline circle mi()
{
	sort(a + 1, a + 1 + n);
	n = unique(a + 1, a + 1 + n) - a - 1;
	if(n == 1)
	{
		cir.c = a[1];
		cir.r = 0;
		return cir;
	}
	random_shuffle(a + 1, a + 1 + n);
	p2c(1, 2);
	for(int i = 3; i <= n; i++)
		if(!inside(a[i]))
		{
			p2c(1, i);
			for(int j = 2; j < i; j++)
				if(!inside(a[j]))
				{
					p2c(i, j);
					for(int k = 1; k < j; k++)
						if(!inside(a[k]))
							p3c(i,j, k);
				}
		}
	return cir;
}
\end{lstlisting}

	\section{圆交$O(n^2\log n)$计算面积和重心}
		\begin{lstlisting}
double pi = acos(-1.0), eps = 1e-12;
double sqr(const double & x) {
	return x * x;
}
double ans[2001];
int sign(const double & x) {
	return x < -eps?-1:x > eps;
}
struct Point {
	double x, y;
	Point(){}
	Point(const double & x, const double & y) : x(x), y(y) {}
	void scan() {scanf("%lf%lf", &x, &y);}
	double sqrlen() {return sqr(x) + sqr(y);}
	double len() {return sqrt(sqrlen());}
	Point rev() {return Point(y, -x);}
	void print() {printf("%f %f\n", x, y);}
	Point zoom(const double & d) {double lambda = d / len(); return Point(lambda * x, lambda * y);}
} dvd, a[2001];
Point centre[2001];
double atan2(const Point & x) {
	return atan2(x.y, x.x);
}
Point operator - (const Point & a, const Point & b) {
	return Point(a.x - b.x, a.y - b.y);
}
Point operator + (const Point & a, const Point & b) {
	return Point(a.x + b.x, a.y + b.y);
}
double operator * (const Point & a, const Point & b) {
	return a.x * b.y - a.y * b.x;
}
Point operator * (const double & a, const Point & b) {
	return Point(a * b.x, a * b.y);
}
double operator % (const Point & a, const Point & b) {
	return a.x * b.x + a.y * b.y;
}
struct circle {
	double r; Point o;
	circle() {}
	void scan() {
		o.scan();
		scanf("%lf", &r);
	}
} cir[2001];
struct arc {
	double theta;
	int delta;
	Point p;
	arc() {};
	arc(const double & theta, const Point & p, int d) : theta(theta), p(p), delta(d) {}
} vec[4444];
int nV;
inline bool operator < (const arc & a, const arc & b) {
	return a.theta + eps < b.theta;
}
int cnt;
inline void psh(const double t1, const Point p1, const double t2, const Point p2) {
	if(t2 + eps < t1) 
		cnt++;
	vec[nV++] = arc(t1, p1, 1);
	vec[nV++] = arc(t2, p2, -1);
}
inline double cub(const double & x) {
	return x * x * x;
}
inline void combine(int d, const double & area, const Point & o) {
	if(sign(area) == 0) return;
	centre[d] = 1 / (ans[d] + area) * (ans[d] * centre[d] + area * o);
	ans[d] += area;
}
bool equal(const double & x, const double & y) {
	return x + eps>  y and y + eps > x;
}
bool equal(const Point & a, const Point & b) {
	return equal(a.x, b.x) and equal(a.y, b.y);
}
bool equal(const circle & a, const circle & b) {
	return equal(a.o, b.o) and equal(a.r, b.r);
}
bool f[2001];
int main() {
	//freopen("hdu4895.in", "r", stdin);
	int n, m, index;
	while(EOF != scanf("%d%d%d", &m, &n, &index)) {
		index--;
		for(int i(0); i < m; i++) {
			a[i].scan();
		}
		for(int i(0); i < n; i++) {
			cir[i].scan();//n个圆
		}
		for(int i(0); i < n; i++) {//这一段在去重圆 能加速 删掉不会错
			f[i] = true;
			for(int j(0); j < n; j++) if(i != j) {
				if(equal(cir[i], cir[j]) and i < j or !equal(cir[i], cir[j]) and cir[i].r < cir[j].r + eps and (cir[i].o - cir[j].o).sqrlen() < sqr(cir[i].r - cir[j].r) + eps) {
					f[i] = false;
					break;
				}
			}
		}
		int n1(0);
		for(int i(0); i < n; i++)
			if(f[i])
				cir[n1++] = cir[i];
		n = n1;//去重圆结束
		fill(ans, ans + n + 1, 0);//ans[i]表示被圆覆盖至少i次的面积
		fill(centre, centre + n + 1, Point(0, 0));//centre[i]表示上面ans[i]部分的重心
		for(int i(0); i < m; i++) 
			combine(0, a[i] * a[(i + 1) % m] * 0.5, 1. / 3 * (a[i] + a[(i + 1) % m]));
		for(int i(0); i < n; i++) {
			dvd = cir[i].o - Point(cir[i].r, 0);
			nV = 0;
			vec[nV++] = arc(-pi, dvd, 1);
			cnt = 0;
			for(int j(0); j < n; j++) if(j != i) {
				double d = (cir[j].o - cir[i].o).sqrlen();
				if(d < sqr(cir[j].r - cir[i].r) + eps) {
					if(cir[i].r + i * eps < cir[j].r + j * eps)
						psh(-pi, dvd, pi, dvd);
				}else if(d + eps < sqr(cir[j].r + cir[i].r)) {
					double lambda = 0.5 * (1 + (sqr(cir[i].r) - sqr(cir[j].r)) / d);
					Point cp(cir[i].o + lambda * (cir[j].o - cir[i].o));
					Point nor((cir[j].o - cir[i].o).rev().zoom(sqrt(sqr(cir[i].r) - (cp - cir[i].o).sqrlen())));
					Point frm(cp + nor);
					Point to(cp - nor);
					psh(atan2(frm - cir[i].o), frm, atan2(to - cir[i].o), to);
				}
			}
			sort(vec + 1, vec + nV);
			vec[nV++] = arc(pi, dvd, -1);
			for(int j = 0; j + 1 < nV; j++) {
				cnt += vec[j].delta;
				//if(cnt == 1) {//如果只算ans[1]和centre[1], 可以加这个if加速.
					double theta(vec[j + 1].theta - vec[j].theta);
					double area(sqr(cir[i].r) * theta * 0.5);
					combine(cnt, area, cir[i].o + 1. / area / 3 * cub(cir[i].r) * Point(sin(vec[j + 1].theta) - sin(vec[j].theta), cos(vec[j].theta) - cos(vec[j + 1].theta)));
					combine(cnt, -sqr(cir[i].r) * sin(theta) * 0.5, 1. / 3 * (cir[i].o + vec[j].p + vec[j + 1].p));
					combine(cnt, vec[j].p * vec[j + 1].p * 0.5, 1. / 3 * (vec[j].p + vec[j + 1].p));
				//}
			}
		}//板子部分结束 下面是题目
		combine(0, -ans[1], centre[1]);
		for(int i = 0; i < m; i++) {
			if(i != index)
				(a[index] - Point((a[i] - a[index]) * (centre[0] - a[index]), (a[i] - a[index]) % (centre[0] - a[index])).zoom((a[i] - a[index]).len())).print();
			else
				a[i].print();
		}
 
	}
	fclose(stdin);
	return 0;
}
\end{lstlisting}

	\section{三维跨立实验+点到线段的垂足在线段上+分数类}
		\begin{lstlisting}
long long gcd(long long a, long long b) {
	return b?gcd(b, a % b):a;
}
struct frac {
	long long x, y;
	frac() {}
	frac(const long long & xx, const long long & yy) : x(xx), y(yy) {
		long long d(gcd(x, y));
		x /= d; y /= d;
		if(y < 0) 
			y = -y, x = -x;
	}
	void print() const {
		printf("(%lld/%lld)\n", x, y);
	}
};
frac operator + (const frac & a, const frac & b) {
	//long long y = a.y / gcd(a.y, b.y) * b.y;
	//return frac(y / a.y * a.x + y / b.y * b.x, y);//这里可以减小中间结果, 以避免爆long long.
	return frac(a.x * b.y + b.x * a.y, a.y * b.y);
}
frac operator - (const frac & a, const frac & b) {
	//long long y = a.y / gcd(a.y, b.y) * b.y;
	//return frac(y / a.y * a.x - y / b.y * b.x, y);
	return frac(a.x * b.y - b.x * a.y, a.y * b.y);
}
frac operator * (const frac & a, const frac & b) {
	//long long v(gcd(a.x, b.y)), w(gcd(a.y, b.x));
	//return frac((a.x / v) * (b.x / w), (a.y / w) * (b.y / v));
	return frac(a.x * b.x, a.y * b.y);
}
frac operator / (const frac & a, const frac & b) {
	//long long v(gcd(a.x, b.x)), w(gcd(a.y, b.y));
	//return frac((a.x / v) * (b.y / w), (a.y / w) * (b.x / v));
	return frac(a.x * b.y, a.y * b.x);
}
bool operator < (const frac & a, const frac & b) {
	return a.x * b.y < b.x * a.y;
}
bool operator == (const frac & a, const frac & b) {
	return a.x * b.y == b.x * a.y;
}
bool operator <= (const frac & a, const frac & b) {
	return a.x * b.y <= b.x * a.y;
}
 
frac sqr(const frac & a) {
	return a * a;
}
struct Point {
	frac x, y, z;
	Point () {}
	void scan() {cin >> x.x >> y.x >> z.x; x.y = y.y = z.y = 1;}
	Point(const frac & x, const frac & y, const frac & z) :x(x), y(y), z(z) {}
	frac sqrlen() {return x * x + y * y + z * z;}
	void print() const {printf("{");x.print(); y.print(); z.print();printf("}\n");}
} a, b, c, d;
Point operator - (const Point & a, const Point & b) {
	return Point(a.x - b.x, a.y - b.y, a.z - b.z);
}
Point operator + (const Point & a, const Point & b) {
	return Point(a.x + b.x, a.y + b.y, a.z + b.z);
}
Point operator * (const frac & a, const Point & b) {
	return Point(a * b.x, a * b.y, a * b.z);
}
frac operator % (const Point & a, const Point & b) {
	return a.x * b.x + a.y * b.y + a.z * b.z;
}
Point operator * (const Point & a, const Point & b) {
	return Point(a.y * b.z - a.z * b.y, a.z * b.x - a.x * b.z, a.x * b.y - a.y * b.x);
}
bool _ (const Point & a) {
	return a.x == frac(0, 1) and a.y == frac(0, 1) and a.z == frac(0, 1);
}
void check(frac & ans, const Point & a, const Point & s, const Point & t) {
	if(sign((a - s) % (t - s)) * sign((a - t) % (t - s)) <= 0) {//点到线段的垂足在线段上(端点含)
		ans = min(ans, ((a - s) * (t - s)).sqrlen() / (t - s).sqrlen());//点到直线距离
	}
}
int sign(const frac & a) {
	return a.x < 0?-1:a.x > 0;
}
int main() {
	int tst;
	scanf("%d", &tst);
	for(int qq = 1; qq <= tst; qq++) {
		a.scan(); b.scan();
		c.scan(); d.scan();//线段(a->b), (c->d)
		frac ans = (a - c).sqrlen();
		ans = min(ans, (a - d).sqrlen());
		ans = min(ans, (b - c).sqrlen());
		ans = min(ans, (b - d).sqrlen());
		Point nor;
		if(!_(nor = (b - a) * (d - c)))//线段平行
			if(sign((c - a) * (d - a) % nor) * sign((c - b) * (d - b) % nor) <= 0 and sign((a - c) * (b - c) % nor) * sign((a - d) * (b - d) % nor) <= 0)//三维跨立实验
				ans = min(ans, sqr(nor % (c - a)) / nor.sqrlen());
		check(ans, a, c, d);
		check(ans, b, c, d);
		check(ans, c, a, b);
		check(ans, d, a, b);
		cout << ans.x << ' ' << ans.y << endl;
	}
	return 0;
}
\end{lstlisting}

	\section{平面图形的转动惯量计算}
		\begin{lstlisting}
int n, m;
double eps = 1e-8;
int sign(const double & x) {
	return x < -eps?-1:x > eps;
}
struct Point {
	double x, y;
	void scan() {
		scanf("%lf%lf", &x, &y);
	}
	void print() {
		printf("(%f %f)\n", x, y);
	}
	Point(const double & x, const double & y) : x(x), y(y) {}
	Point() {}
	double len() {return sqrt(x * x + y * y);}
	Point rev() {return Point(-y, x);}
} a[222], b[222];
Point operator + (const Point & a, const Point & b) {
	return Point(a.x + b.x, a.y + b.y);
}
Point operator - (const Point & a, const Point & b) {
	return Point(a.x - b.x, a.y - b.y);
}
Point operator * (const double & a, const Point & b) {
	return Point(a * b.x, a * b.y);
}
double operator % (const Point & a, const Point & b) {
	return a.x * b.x + a.y * b.y;
}
double operator * (const Point & a, const Point & b) {
	return a.x * b.y - a.y * b.x;
}
double sqr(const double & x) {
	return x * x;
}
double cub(const double & x) {
	return x * x * x;
}
double calc(const double & Y, const double & c0, const double & c1, const double & c2, const double & c3) {
	return Y * c0 + 0.5 * Y * Y * c1 + Y * Y * Y * c2 / 3 + Y * Y * Y * Y * c3 / 4;
}
int main() {
	scanf("%d%d", &n, &m);
	for(int i = 1; i <= n; i++) {
		a[i].scan();
	}
	a[0] = a[n];
	double area(0);
	for(int i = 1; i <= n; i++) {
		area += (a[i - 1] * a[i]);
	}
	for(int i = 1; i <= m; i++) {
		b[i].scan();
	}
	double ans(0);
	for(int i = 1; i <= m; i++) {
		vector<Point> vec(a + 1, a + 1 + n);
		for(int j = 1; j <= m; j++) if(j != i) {
			vector<Point> vec1;
			Point mid(0.5 * (b[i] + b[j])), dir((b[j] - b[i]).rev());
			for(int k = 0; k < (int)vec.size(); k++) {
				if(sign((vec[k] - mid) * dir) <= 0)
					vec1.push_back(vec[k]);
				Point dir1(vec[(k + 1) % (int)vec.size()] - vec[k]);
				if(sign((vec[k] - mid) * dir) * sign((vec[(k + 1) % (int)vec.size()] - mid) * dir) < 0) {
					double lambda((mid - vec[k]) * dir / (dir1 * dir));
					vec1.push_back(vec[k] + lambda * dir1);
				}
			}
			vec = vec1;
		}
		for(int j = 0; j < (int)vec.size(); j++)
			vec[j] = vec[j] - b[i];
		for(int j = 0; j < (int)vec.size(); j++){
			double X1(vec[j].len()), X(vec[(j + 1) % (int)vec.size()] % vec[j] / vec[j].len()), Y(vec[j] * vec[(j + 1) % (int)vec.size()] / vec[j].len());
			//若是vec[j].len()为0 或者Y为0 则转动惯量为0
			//旋转中心在原点 三角形((0, 0), vec[j], vec[j + 1])的转动惯量, 其中若vec[j] * vec[j + 1] < 0求出来的是转动惯量的相反数.
			ans += calc(Y, cub(X1) / 3, sqr(X1) * (X - X1) / Y, X1 * sqr((X - X1) / Y), (cub((X - X1) / Y) - cub(X / Y)) / 3);
			ans += calc(Y, 0, 0, X1, -X1 / Y);
		}
 
	}
 
	printf("%.10f\n", ans / area * 2);
	fclose(stdin);
	return 0;
}
\end{lstlisting}

	\section{凸多边形内的最大圆$O(n\log n)$}
		\begin{lstlisting}
double eps(1e-8);
int sign(const double & x) {
	return x < -eps?-1:x > eps;
}
struct Point {
	double x, y;
	Point() {
	}
	Point(const double & x, const double & y) : x(x), y(y) {
	}
	double sqrlen() const {
		return x * x + y * y;
	}
	double len() const {
		return sqrt(sqrlen());
	}
	void scan() {
		scanf("%lf%lf", &x, &y);
	}
	void print() const {
		printf("(%f %f)\n", x, y);
	}
};
Point operator + (const Point & a, const Point & b) {
	return Point(a.x + b.x, a.y + b.y);
}
Point operator - (const Point & a, const Point & b) {
	return Point(a.x - b.x, a.y - b.y);
}
Point operator * (const double & a, const Point & b) {
	return Point(a * b.x, a * b.y);
}
double operator * (const Point & a, const Point & b) {
	return a.x * b.y - a.y * b.x;
}
struct Line {
	Point s, d;
	Line() {
	}
	Line(const Point & s, const Point & d) : s(s), d(d) {
	}
};
Point crs(const Line & a, const Line & b) {
	double lambda((b.s - a.s) * b.d / (a.d * b.d));
	return a.s + lambda * a.d;
}
struct reca {
	Point a, b;
	int prv, nxt;
	Point d() const {
		return b - a;
	}
	double calc();
} a[11111];
reca (&c)[11111](a);
double reca::calc() {
	if(sign(d() * c[prv].d()) and sign(d() * c[nxt].d())) {
		double len1(c[prv].d().len()), len2(d().len()), len3(c[nxt].d().len());
		Point cp(crs(Line(a, 1 / (len1 + len2) * (len2 * c[prv].a + len1 * b) - a), Line(b, 1 / (len2 + len3) * (len3 * a + len2 * c[nxt].b) - b)));
		return fabs((cp - a) * d() / d().len());
	}else
		return 1e100;
 
}
double val[11111];
bool f[11111];	
int main() {
	int n;
	scanf("%d", &n);
	for(int i(0); i < n; i++) {
		a[i].a.scan();
	}
	for(int i(0); i < n; i++) {
		a[i].b = a[(i + 1) % n].a;
		a[i].prv = (i + n - 1) % n;
		a[i].nxt = (i + 1) % n;
	}
	priority_queue<pair<double, int>, vector<pair<double, int> >, greater<pair<double, int> > > hp;
	for(int i(0); i < n; i++) {
		hp.push(make_pair(val[i] = a[i].calc(), i));
	}
	for(int i(1); i <= n - 3; i++) {
		int prv(a[hp.top().second].prv), nxt(a[hp.top().second].nxt);
		a[prv].nxt = nxt;
		a[nxt].prv = prv;
		if(sign(a[prv].d() * a[nxt].d()))
			a[prv].b = a[nxt].a = crs(Line(a[prv].a, a[prv].d()), Line(a[nxt].a, a[nxt].d()));
		f[hp.top().second] = true;
		hp.pop();
		hp.push(make_pair(val[prv] = a[prv].calc(), prv));
		hp.push(make_pair(val[nxt] = a[nxt].calc(), nxt));
		while(f[hp.top().second] or val[hp.top().second] != hp.top().first)
			hp.pop();
	}
	int y(hp.top().second);
	printf("%f\n", min(min(val[a[y].prv], val[a[y].nxt]), val[y]));
}
\end{lstlisting}

	\section{三维凸包}
		\begin{lstlisting}
const double eps = 1e-8;
int mark[1005][1005];
Point info[1005];
int n, cnt;
double mix(const Point &a, const Point &b, const Point &c) {
	return a.dot(b.cross(c));}
double area(int a, int b, int c) {
	return ((info[b] - info[a]).cross(info[c] - info[a])).length();}
double volume(int a, int b, int c, int d) {
	return mix(info[b] - info[a], info[c] - info[a], info[d] - info[a]);}
struct Face {
	int a, b, c;
	Face() {}
	Face(int a, int b, int c): a(a), b(b), c(c) {}
	int &operator [](int k) { return k==0?a:k==1?b:c; }
};
vector <Face> face;
inline void insert(int a, int b, int c) { face.push_back(Face(a, b, c));}
void add(int v) {
	vector <Face> tmp;
	int a, b, c;
	cnt++;
	for (int i = 0; i < SIZE(face); i++) {
		a = face[i][0];  b = face[i][1];  c = face[i][2];
		if (Sign(volume(v, a, b, c)) < 0)
			mark[a][b] = mark[b][a] = mark[b][c] = mark[c][b] = mark[c][a] =
					 mark[a][c] = cnt;
		else tmp.push_back(face[i]);
	}
	face = tmp;
	for (int i = 0; i < SIZE(tmp); i++) {
		a = face[i][0]; b = face[i][1]; c = face[i][2];
		if (mark[a][b] == cnt) insert(b, a, v);
		if (mark[b][c] == cnt) insert(c, b, v);
		if (mark[c][a] == cnt) insert(a, c, v);
	}
}
int Find() {
	for (int i = 2; i < n; i++) {
		Point ndir = (info[0] - info[i]).cross(info[1] - info[i]);
		if (ndir == Point()) continue;
		swap(info[i], info[2]);
		for (int j = i + 1; j < n; j++)
			if (Sign(volume(0, 1, 2, j)) != 0) {
				swap(info[j], info[3]);
				insert(0, 1, 2);  insert(0, 2, 1);
				return 1;
			}
	}
	return 0;
}
int main() {
	for (; scanf("%d", &n) == 1; ) {
		for (int i = 0; i < n; i++)
			info[i].Input();
		sort(info, info + n);
		n = unique(info, info + n) - info;
		face.clear();
		random_shuffle(info, info + n);
		if (Find()) {
			memset(mark, 0, sizeof(mark));
			cnt = 0;
			for (int i = 3; i < n; i++) add(i);
			vector<Point> Ndir;
			for (int i = 0; i < SIZE(face); ++i) {
				Point p = (info[face[i][0]] - info[face[i][1]]).cross
						(info[face[i][2]] - info[face[i][1]]);
				p = p / p.length();
				Ndir.push_back(p);
			}
			sort(Ndir.begin(), Ndir.end());
			int ans = unique(Ndir.begin(), Ndir.end()) - Ndir.begin();
			printf("%d\n", ans);
		} else {
			printf("1\n");
		}
	}
}
\end{lstlisting}


	\section{点在多边形内}
		\begin{lstlisting}
bool in_polygon(const point &p, const vector<point> &poly) {
	int n = (int)poly.size();
	int counter = 0;
	for (int i = 0; i < n; ++i) {
		point a = poly[i], b = poly[(i + 1) % n];
		if (point_on_line(p, line(a, b)))
			return false; // bounded excluded
		int x = sign(det(p - a, b - a));
		int y = sign(a.y - p.y);
		int z = sign(b.y - p.y);
		if (x > 0 && y <= 0 && z > 0)
			counter++;
		if (x < 0 && z <= 0 && y > 0)
			counter--;
	}
	return counter != 0;
}
\end{lstlisting}


	\section{三角形的内心}
		\begin{lstlisting}
point incenter(const point &a, const point &b, const point &c) {
	double p = (a - b).length() + (b - c).length() + (c - a).length();
	return (a * (b - c).length() + b * (c - a).length() + c * (a - b).length()) / p;
}
\end{lstlisting}
	

	\section{三角形的外心}
		\begin{lstlisting}
point circumcenter(const point &a, const point &b, const point &c) {
	point p = b - a, q = c - a, s(dot(p, p) / 2, dot(q, q) / 2);
	double d = det(p, q);
	return a + point(det(s, point(p.y, q.y)), det(point(p.x, q.x), s)) / d;
}
\end{lstlisting}


	\section{三角形的垂心}
		\begin{lstlisting}
point orthocenter(const point &a, const point &b, const point &c) {
	return a + b + c - circumcenter(a, b, c) * 2.0;
}
\end{lstlisting}


	\section{V图}
		\input{V图.tex}
\chapter{数据结构}
	\section{KD树}
		\input{KD树.tex}
	\section{树链剖分}
		\begin{lstlisting}
int const N = ;
int n;
vector<int> adj[N];
int father[N], height[N], size[N], son[N], top[N], idx[N], num[N];
inline void prepare() {
	vector<int> queue; father[1] = height[1] = 0; queue.push_back(1);
	for (int head = 0; head < (int)queue.size(); ++head) {
		int x = queue[head];
		for (int i = 0; i < (int)adj[x].size(); ++i) {
			int y = adj[x][i];
			if (y != father[x])
				father[y] = x, height[y] = height[x] + 1, queue.push_back(y);
		}
	}
	for (int i = n - 1; i >= 0; --i) {
		int x = queue[i]; size[x] = 1; son[x] = -1;
		for (int j = 0; j < (int)adj[x].size(); ++j) {
			int y = adj[x][j];
			if (y != father[x]) {
				size[x] += size[y];
				if (son[x] == -1 || size[son[x]] < size[y]) son[x] = y;
			}
		}
	} int tot = 0; fill(top + 1, top + n + 1, 0);
	for (int i = 0; i < n; ++i) {
		int x = queue[i];
		if (top[x] == 0)
			for (int y = x; y != -1; y = son[y])
				top[y] = x, idx[y] = ++tot, num[tot] = //data[y];
	} build(1, 1, n);
}
inline void handle(int x, int y) {
	for (; true; ) {
		if (top[x] == top[y]) {
			if (x == y) handle(1, 1, n, idx[x], idx[x]);
			else {
				if (height[x] < height[y]) handle(1, 1, n, idx[x], idx[y]);
				else handle(1, 1, n, idx[y], idx[x]);
			} break;
		}
		if (height[top[x]] > height[top[y]])
			handle(1, 1, n, idx[top[x]], idx[x]), x = father[top[x]];
		else handle(1, 1, n, idx[top[y]], idx[y]), y = father[top[y]];
	}
}
\end{lstlisting}
	\section{可持久化左偏树}
		\begin{lstlisting}
Node * persiMerge(Node * a, Node * b) {
	if(!a) return b;
	if(!b) return a;
	Node * res;
	if(a->v < b->v) {
		res = new Node(*a);
		res->s[1] = persiMerge(b, res->s[1]);
	}else {
		res = new Node(*b);
		res->s[1] = persiMerge(a, res->s[1]);
	}
	if(!res->s[0] or res->s[1] and res->s[0]->l < res->s[1]->l)
		swap(res->s[0], res->s[1]);
	res->l = res->s[1]?res->s[1]->l + 1:0;
	return res;
}
\end{lstlisting}

	\section{treap}
		\begin{lstlisting}
struct Node {
	Node *child[2]; int key; int size, count, aux;
	inline Node(int _aux) {
		child[0] = child[1] = 0; key = size = count = 0; aux = _aux;
	}
	inline void update() { size = count + child[0]->size + child[1]->size; }
};
Node *null;
inline void print(Node *&x) {
	if (x == null) return; print(x->child[0]); printf("%d ", x->key);
	print(x->child[1]);
}
inline Node* create(int key)
	Node *x = new Node(rand() % INT_MAX); x->key = key; x->count = x->size = 1;
	x->child[0] = x->child[1] = null; return x;
}
inline void rotate(Node *&x, int dir) {
	Node *y = x->child[!dir]; x->child[!dir] = y->child[dir]; y->child[dir] = x;
	x->update(); y->update(); x = y;
}
inline void insert(Node *&x, int key) {
	if (x == null) { x = create(key); return; }
	if (x->key == key) x->count++;
	else if (x->key > key) {
		insert(x->child[0], key); if (x->child[0]->aux < x->aux) rotate(x, 1);
	} else {
		insert(x->child[1], key); if (x->child[1]->aux < x->aux) rotate(x, 0);
	} x->update();
}
inline void erase(Node *&x, int key) {
	if (x == null) return;
	if (x->key == key) {
		if (x->count > 1) x->count--;
		else if (x->child[0] == null && x->child[1] == null) {
			delete(x); x = null; return;
		} else if (x->child[0]->aux < x->child[1]->aux)
			rotate(x, 1), erase(x->child[1], key);
		else rotate(x, 0), erase(x->child[0], key);
	} else if (x->key > key) erase(x->child[0], key);
	else erase(x->child[1], key);
	x->update();
}
inline void prepare() { null = new Node(INT_MAX); }
\end{lstlisting}
	\section{functional\_treap}
		\begin{lstlisting}
namespace functional_treap {
	struct node {
		int size;
		node *left, *right;
		inline node(node *_left, node *_right) {
			left = _left;
			right = _right;
		}
		inline node* update() {
			size = left->size + 1 + right->size;
			return this;
		}
		inline pair<node*, node*> split(int);
	};
 
	node* null;
 
	inline bool random(int x, int y) {
		return rand() % (x + y) < x;
	}
 
	inline node* mergy(node* x, node* y) {
		if (x == null) {
			return y;
		}
		if (y == null) {
			return x;
		}
		if (random(x->size, y->size)) {
			x->right = mergy(x->right, y);
			return x->update();
		}
		y->left = mergy(x, y->left);
		return y->update();
	}
 
	inline pair<node*, node*> node::split(int n) {
		if (this == null) {
			return make_pair(null, null);
		}
		if (n <= left->size) {
			pair<node*, node*> ret = left->split(n);
			left = null;
			return make_pair(ret.first, mergy(ret.second, this->update()));
		}
		pair<node*, node*> ret = right->split(n - left->size);
		right = null;
		return make_pair(mergy(this->update(), ret.first), ret.second);
	}
 
	inline void prepare() {
		null = new node(null, null);
		null->left = null->right = null;
	}
}
\end{lstlisting}

	\section{LCT}
		\begin{lstlisting}
namespace link_cut_tree {
	struct node {
		node *child[2], *father;
		bool head, rev;
		int val, sum, size;
		inline node() {
			head = rev = val = sum = size = 0;
		}
		inline void set(node *temp, int dir) {
			child[dir] = temp;
			temp->father = this;
		}
		inline int which() {
			return father->child[1] == this;
		}
		inline void update() {
			sum = val + child[0]->sum + child[1]->sum;
			size = 1 + child[0]->size + child[1]->size;
		}
		inline void release() {
			if (rev) {
				child[0]->reverse();
				child[1]->reverse();
				rev = 0;
			}
		}
		inline void reverse() {
			if (size == 0) {
				return;
			}
			rev ^= 1;
			swap(child[0], child[1]);
		}
	};
 
	node *null, *tree[N];
 
	inline node* create(int val) {
		node *temp = new node();
		temp->val = temp->sum = val;
		temp->size = 1;
		temp->child[0] = temp->child[1] = temp->father = null;
		temp->head = true;
		return temp;
	}
 
	inline void rotate(node *root) {
		node *father = root->father;
		father->release();
		root->release();
		int dir = root->which();
		father->set(root->child[!dir], dir);
		if (father->head) {
			father->head = false;
			root->head = true;
			root->father = father->father;
		} else {
			father->father->set(root, father->which());
		}
		root->set(father, !dir);
		father->update();	
	}
 
	inline void splay(node *root) {
		for (root->release(); !root->head; ) {
			if (root->father->head) {
				rotate(root);
			} else {
				root->which() == root->father->which() ? (rotate(root->father), rotate(root)) : (rotate(root), rotate(root));
			}
		}
		root->update();
	}
 
	inline void access(node *root) {
		for (node *temp = null; root != null; temp = root, root = root->father) {
			splay(root);
			root->child[1]->head = true;
			root->child[1] = temp;
			root->child[1]->head = false;
			root->update();
		}
	}
 
	inline void link(int son, int father) {
		access(tree[son]);
		splay(tree[son]);
		tree[son]->father = tree[father];
		tree[son]->reverse();
	}
 
	inline void cut(int x, int y) {
		access(tree[y]);
		splay(tree[x]);
		if (tree[x]->father == tree[y]) {
			tree[x]->father = null;
		} else {
			access(tree[x]);
			splay(tree[y]);
			if (tree[y]->father == tree[x]) {
				tree[y]->father = null;
			}
		}
	}
 
	inline void handle(int x, int y) {
		access(tree[x]);
		node *root = tree[y];
		for (node *temp = null; root != null; temp = root, root = root->father) {
			splay(root);
			if (root->father == null) {
 
			}
			root->child[1]->head = true;
			root->child[1] = temp;
			root->child[1]->head = false;
			root->update();
		}
	}
 
	inline void init(int n, int val[]) {
		for (int i = 1; i <= n; ++i) {
			tree[i] = create(val[i]);
		}
	}
 
	inline void prepare() {
		null = new node();
		null->child[0] = null->child[1] = null->father = null;
	}
}
\end{lstlisting}

	\section{Splay}
		\begin{lstlisting}
namespace splay {
struct Node {
	Node *child[2], *father; int val, sum, size;
	inline Node() { val = sum = size = 0; }
	inline int which() { return father->child[1] == this; }
	inline void set(Node *temp, int dir) {
		child[dir] = temp; temp->father = this;
	}
	inline void update() {
		size = 1 + child[0]->size + child[1]->size;
		sum = val + child[0]->sum + child[1]->sum;
	}
	inline void release() {}
};
Node *null, *head;
inline void print(Node *root) {
	if (root == null) return; print(root->child[0]); printf("%d ", root->val);
	print(root->child[1]);
}
inline Node* create(int val = 0) {
	Node *temp = new Node(); temp->val = val;
	temp->child[0] = temp->child[1] = temp->father = null; return temp;
}
inline void rotate(Node *root) {
	Node *father = root->father; int dir = root->which(); father->release();
	root->release; father->set(root->child[!dir], dir);
	father->father->set(root, father->which()); root->set(father, !dir);
	if (father == head) head = root;
	father->update();
}
inline void splay(Node *root, Node *target) {
	for (root->release(); root->father != target; )
		if (root->father->father == target) rotate(root);
		else root->which() == root->father->which() ? (rotate(root->father), rotate(root)) : (rotate(root), rotate(root));
	root->update();
}
inline int rank(Node *root) {
	splay(root, null); return root->child[0]->size + 1;
}
inline Node* find(int rank) {
	Node *now = head;
	for (; now->child[0]->size + 1 != rank; ) {
		now->release();
		if (now->child[0]->size + 1 > rank) now = now->child[0];
		else { rank -= now->child[0]->size + 1; now = now->child[1]; }
	} return now;
}
inline void splay(int left, int right) {
	splay(find(right), null); splay(find(left), head);
}
inline Node* insert(int pos, int val) {
	splay(pos, pos + 1); Node *now = head->child[0]; Node *cur = create(val);
	now->set(cur, 1); splay(cur, null); return head;
}
inline void insert(int pos, int n, int val[]) {
	splay(pos, pos + 1); Node *now = head->child[0];
	for (int i = 1; i <= n; ++i) {
		Node *cur = create(val[i]); now->set(cur, 1); now = cur;
	} splay(now, null);
}
inline void erase(Node *root) {
	int pos = rank(root); splay(pos - 1, pos + 1);
	head->child[0]->child[1] = null; head->child[0]->update(); head->update();
}
inline int query(int left, int right) {
	splay(left - 1, right + 1); return head->child[0]->child[1]->sum;
}
inline void prepare() {
	null = new Node(); head = create(); Node *tail = create();
	head->set(tail, 1); splay(tail, null);
}
\end{lstlisting}
\chapter{图论}
	\section{Gabow算法求点双连通分量(非递归)}
		边(u, v)属于min(color[u], color[v])这个点双连通分量.
		\begin{lstlisting}
int color[222222], siz[222222], cnt[222222];
long long ans[222222];
vector<int> edges[222222];
vector<pair<int, int> > st0, st2;
vector<int> st1;
void psh(int v) {
	st0.push_back(make_pair(v, 0));
	color[v] = st1.size();
	st1.push_back(v);
}
int main() {
	freopen("travel.in", "r", stdin);
	freopen("travel.out", "w", stdout);
	int n, m;
	scanf("%d%d", &n, &m);
	for(int i(1); i <= m; i++) {
		int x, y;
		scanf("%d%d", &x, &y);
		edges[x].push_back(y);
		edges[y].push_back(x);
	}
	int c(n);
	fill(color + 1, color + 1 + n, 0);
	fill(ans + 1, ans + 1 + n, 0);
	fill(cnt + 1, cnt + 1 + n, 0);
	fill(siz + 1, siz + 1 + n, 0);
	for(int i(1); i <= n; i++) if(!color[i]) {
		psh(i);
		while(!st0.empty()) {
 
			int v(st0.back().first), p(st0.back().second++);
			if(p != (int)edges[v].size()) {
				int y(edges[v][p]);
				if(!color[y]) {
					psh(y);
					st2.push_back(make_pair(color[v], color[y]));
				}else
					while(!st2.empty() and st2.back().first > color[y])
						st2.pop_back();
			}else {
				st0.pop_back();
				siz[v]++;
				if(color[v] == 1) 
					color[v] = c;
				else {
					int fa(st0.back().first);
					if(st2.back().second == color[v]) {
						st2.pop_back();
						color[v] = ++c;
						while(st1.back() != v) {
							color[st1.back()] = c;
							st1.pop_back();
						}
						st1.pop_back();
						ans[fa] += (long long)cnt[fa] * siz[v];
						cnt[fa] += siz[v];
					}
					siz[fa] += siz[v];
				}
				ans[v] += (long long)(n - cnt[v]) * cnt[v] + n - cnt[v] - 1;
			}
		}
	}
	for(int i(1); i <= n; i++) {
		cout << ans[i] << endl;//ans[i]:删去点i后, 无法连通的{a, b}数, 其中a, b为图中不同节点且无序.
	}
	fclose(stdin);
	fclose(stdout);
	return 0;
}
\end{lstlisting}

	\section{Hopcroft Karp求二分图最大匹配$O(EV^{0.5})$}
		\begin{lstlisting}
// hint :: 全部都是0base
// 用的时候,建好边,左边n个点,右边m个点,直接调用maxMatch即可
 
const int N = 3333;
 
vector<int> e[N];
int pairx[N], pairy[N], level[N];
int n, m;
 
bool dfs(int x) {
	for(int i = 0; i < (int)e[x].size(); i++) {
		int y = e[x][i];
		int w = pairy[y];
		if (w == -1 || level[x] + 1 == level[w] && dfs(w)) {
			pairx[x] = y;
			pairy[y] = x;
			return true;
		}
	}
	level[x] = -1;
	return false;
}
 
int maxMatch() {
	fill(pairx, pairx + n, -1);
	fill(pairy, pairy + m, -1);
 
	for(int answer = 0; ; ) {
		vector<int> queue;
		for(int i = 0; i < n; i++) {
			if (pairx[i] == -1) {
				level[i] = 0;
				queue.push_back(i);
			} else {
				level[i] = -1;
			}
		}
 
		for(int head = 0; head < (int)queue.size(); head++) {
			int x = queue[head];
			for(int i = 0; i < (int)e[x].size(); i++) {
				int y = e[x][i];
				int w = pairy[y];
				if (w != -1 && level[w] < 0) {
					level[w] = level[x] + 1;
					queue.push_back(w);
				}
			}
		}
 
		int delta = 0;
		for(int i = 0; i < n; i++) {
			if (pairx[i] == -1 && dfs(i)) {
				delta++;
			}
		}
		if (delta == 0) {
			return answer;
		} else {
			answer += delta;
		}
	}
}
 
int solve() {
	int timing;
	scanf("%d", &timing);
 
	static int x[N], y[N], s[N];
	scanf("%d", &n);
	for(int i = 0; i < n; i++) {
		scanf("%d %d %d", &x[i], &y[i], &s[i]);
		e[i].clear();
	}	
 
	scanf("%d", &m);
	for(int i = 0; i < m; i++) {
		int xx, yy;
		scanf("%d %d", &xx, &yy);
		for(int j = 0; j < n; j++) {
			if (timing * timing * s[j] * s[j] >= (xx - x[j]) * (xx - x[j]) + (yy - y[j]) * (yy - y[j])) {
				e[j].push_back(i);
			}
		}
	}
 
	return maxMatch();
}
 
int main() {
	freopen("input.txt", "r", stdin);
	int test;
	scanf("%d", &test);
	while(test--) {
		static int testCount = 0;
		printf("Scenario #%d:\n", ++testCount);
		printf("%d\n", solve());
		puts("");
	}
	return 0;
}
\end{lstlisting}

	\section{最小树形图}
		
	\begin{lstlisting}
const int maxn=1100;

int n,m , g[maxn][maxn] , used[maxn] , pass[maxn] , eg[maxn] , more , queue[maxn]; 	 

void combine (int id , int &sum ) {
	int tot = 0 , from , i , j , k ; 		 
	for ( ; id!=0 && !pass[ id ] ; id=eg[id] ) { 
		queue[tot++]=id ; pass[id]=1;
	} 
	for ( from=0; from<tot && queue[from]!=id ; from++);
	if  ( from==tot ) return ; 
	more = 1 ;
	for ( i=from ; i<tot ; i++) { 
		sum+=g[eg[queue[i]]][queue[i]] ;
		if ( i!=from ) { 
			used[queue[i]]=1;
			for ( j = 1 ; j <= n ; j++) if ( !used[j] )
				if ( g[queue[i]][j]<g[id][j] ) g[id][j]=g[queue[i]][j] ;
		}
	}
	for ( i=1; i<=n ; i++) if ( !used[i] && i!=id ) { 
		for ( j=from ; j<tot ; j++){ 
			k=queue[j];
			if ( g[i][id]>g[i][k]-g[eg[k]][k] ) g[i][id]=g[i][k]-g[eg[k]][k];
		}
	}
}
	
int mdst( int root ) { // return the total length of MDST 
	int i , j , k , sum = 0 ; 	 
	memset ( used , 0 , sizeof ( used ) ) ; 	 
	for ( more =1; more ; ) { 
		more = 0 ; 
		memset (eg,0,sizeof(eg)) ; 	 
		for ( i=1 ; i <= n ; i ++) if ( !used[i] && i!=root ) {
			for ( j=1 , k=0 ; j <= n ; j ++) if ( !used[j] && i!=j ) 	 
				if ( k==0 || g[j][i] < g[k][i] ) k=j ; 
			eg[i] = k ; 	 
		} 
		memset(pass,0,sizeof(pass)); 
		for ( i=1; i<=n ; i++) if ( !used[i] && !pass[i] && i!= root ) combine ( i , sum ) ; 	 
	}
	for ( i =1; i<=n ; i ++) if ( !used[i] && i!= root ) sum+=g[eg[i]][i];
	return sum ; 	 
} 	 


int main(){
   freopen("input.txt","r",stdin);
   freopen("output.txt","w",stdout);
   int i,j,k,test,cases;
   cases=0;
   scanf("%d",&test);
   while (test){
		test--;
		//if (n==0) break;
		scanf("%d%d",&n,&m);
//		memset(g,60,sizeof(g));
		foru(i,1,n)
		  foru(j,1,n) g[i][j]=1000001;
		foru(i,1,m) {
			scanf("%d%d",&j,&k);
			j++;k++;
			scanf("%d",&g[j][k]);	
		}
		cases++;
		printf("Case #%d: ",cases);
		k=mdst(1);
		if (k>1000000) printf("Possums!\n");  //===no
		else printf("%d\n",k);
	}
   
   return 0;
}
	\end{lstlisting}

	\section{KM}
		\begin{lstlisting}
#include <cstdio>
#include <cstdlib>
#include <algorithm>
#include <vector>
#include <cstring>
#include <string>
#include <iostream>

#define foreach(e, x) for(__typeof(x.begin()) e = x.begin(); e != x.end(); ++e)

using namespace std;

const int N = 333;
const int INF = (1 << 30);

int mat[N][N], lx[N], ly[N], vx[N], vy[N], slack[N];
int n, match[N];

bool find(int x) {
	vx[x] = 1;
	for(int i = 1; i <= n; i++) {
		if (vy[i]) {
			continue;
		}
		int temp = lx[x] + ly[i] - mat[x][i];
		if (temp == 0) {
			vy[i] = 1;
			if (match[i] == -1 || find(match[i])) {
				match[i] = x;
				return true;
			}
		} else {
			slack[i] = min(slack[i], temp);
		}
	}
	return false;
}

int KM() {
	for(int i = 1; i <= n; i++) {
		lx[i] = -INF;
		ly[i] = 0;
		match[i] = -1;
		for(int j = 1; j <= n; j++) {
			lx[i] = max(lx[i], mat[i][j]);
		}
	}
	for(int i = 1; i <= n; i++) {
		for(int j = 1; j <= n; j++) {
			slack[j] = INF;
		}
		for(; ;) {
			memset(vx, 0, sizeof(vx));
			memset(vy, 0, sizeof(vy));
			for(int j = 1; j <= n; j++) {
				slack[j] = INF;
			}
			if (find(i)) {
				break;
			}
			int delta = INF;
			for(int j = 1; j <= n; j++) {
				if (!vy[j]) {
					delta = min(delta, slack[j]);
				}
			}
			for(int j = 1; j <= n; j++) {
				if (vx[j]) {
					lx[j] -= delta;
				}
				if (vy[j]) {
					ly[j] += delta;
				} else {
					slack[j] -= delta;
				}
			}
		}
	}
	int answer = 0;
	for(int i = 1; i <= n; i++) {
		answer += mat[match[i]][i];
	}
	return answer;
}

int main() {
	while(scanf("%d", &n) != EOF) {
		for(int i = 1; i <= n; i++) {
			for(int j = 1; j <= n; j++) {
				scanf("%d", &mat[i][j]);
			}
		}
		printf("%d\n", KM());
	}
	return 0;
}
\end{lstlisting}

	\section{扩展KM}
		\begin{lstlisting}
#include <cstdio>
#include <cstdlib>
#include <algorithm>
#include <iostream>
#include <cstring>
using namespace std;

const int N = 205;
const int inf = 100000000;

int a[N], b[N], c[N][N], vx[N], vy[N], w[N][N], dx[N], dy[N];
int ans, m, n, slack[N], lk[N], next[N];

bool hungary(int x) {
	vx[x] = 1;
	for(int i = 1; i <= n; i++) {
		if (vy[i])
			continue;
		int delta = dx[x] + dy[i] - w[x][i];
		if (delta == 0) {
			vy[i] = 1;
			if (b[i]) {
				lk[x] = i;
				next[x] = 0;
				return true;
			}
			for(int j = 1; j <= m; j++) {
				if (vx[j])
					continue;
				if (c[j][i] && hungary(j)) {
					lk[x] = i;
					next[x] = j;
					return true;
				}
			}
		} else {
			slack[i] = min(slack[i], delta);
		}
	}
	return false;
}

void travel(int x) {
	int flow = a[x];
	for(int i = x; i; i = next[i]) {
		if (next[i])
			flow = min(flow, c[next[i]][lk[i]]);
		else 
			flow = min(flow, b[lk[i]]);
	}
	a[x] -= flow;
	for(int i = x; i; i = next[i]) {
		if (next[i])
			c[next[i]][lk[i]] -= flow;
		else
			b[lk[i]] -= flow;
		c[i][lk[i]] += flow;
	}
}

int Main() {
	scanf("%d %d", &m, &n);
	for(int i = 1; i <= m; i++)
		scanf("%d", &a[i]);
	for(int i = 1; i <= n; i++)
		scanf("%d", &b[i]);	
	for(int i = 1; i <= m; i++)
		for(int j = 1; j <= n; j++) {
			scanf("%d", &w[i][j]);
			w[i][j] *= -1;
			c[i][j] = 0;
	}
	memset(dy, 0, sizeof(dy));
	for(int i = 1; i <= m; i++) {
		dx[i] = -inf;
		for(int j = 1; j <= n; j++)
			dx[i] = max(dx[i], w[i][j]);
	}
	for(int i = 1; i <= m; i++) {
		while(1) {
			for(int j = 1; j <= n; j++)
				slack[j] = inf;
			while (a[i]) {
				fill(vx + 1, vx + m + 1, 0);
				fill(vy + 1, vy + n + 1, 0);
				if (hungary(i))
					travel(i);
				else
					break;
			}
			if (!a[i])
				break;
			int delta = inf;
			for(int j = 1; j <= n; j++)
				if (!vy[j])
					delta = min(delta, slack[j]);
			for(int j = 1; j <= m; j++)
				if (vx[j])
					dx[j] -= delta;
			for(int j = 1; j <= n; j++)
				if (vy[j])
					dy[j] += delta;
		}
	}
	long long ans = 0;
	for(int i = 1; i <= m; i++)
		for(int j = 1; j <= n; j++) {
			ans += (long long)c[i][j] * w[i][j];
		}
	cout << -ans << endl;
	return 0;
} 

int main() {
	int testCount;
	scanf("%d", &testCount);
	while(testCount--) {
		Main();
	}
	return 0;
}
	\end{lstlisting}

	\section{度限制生成树}
		\begin{lstlisting}
const int N = 55, M = 1010, INF = 1e8;
int n, m, S, K, ans, cnt, Best[N], fa[N], FE[N];
int f[N], p[M], t[M], c[M], o, Cost[N];
bool u[M], d[M];
pair<int, int> MinCost[N];
struct Edge {
	int a, b, c;
	bool operator < (const Edge & E) const { return c < E.c; }
}E[M];
vector<int> SE;
inline int F(int x) { return fa[x] == x ? x : fa[x] = F(fa[x]); }
inline void AddEdge(int a, int b, int C) {
	p[++o] = b; c[o] = C;
	t[o] = f[a]; f[a] = o;
}
void dfs(int i, int father) {
	fa[i] = father;
	if (father == S) Best[i] = -1;
	else {
		Best[i] = i;
		if (Cost[Best[father]] > Cost[i]) Best[i] = Best[father];
	}
	for (int j = f[i]; j; j = t[j])
	if (!d[j] && p[j] != father) {
		Cost[p[j]] = c[j];
		FE[p[j]] = j;
		dfs(p[j], i);
	}
}
inline void Kruskal() {
	cnt = n - 1; ans = 0; o = 1;
	for (int i = 1; i <= n; i++) fa[i] = i, f[i] = 0;
	sort(E + 1, E + m + 1);
	for (int i = 1; i <= m; i++) {
		if (E[i].b == S) swap(E[i].a, E[i].b);
		if (E[i].a != S && F(E[i].a) != F(E[i].b)) {
			fa[F(E[i].a)] = F(E[i].b);
			ans += E[i].c;
			cnt--;
			u[i] = true;
			AddEdge(E[i].a, E[i].b, E[i].c);
			AddEdge(E[i].b, E[i].a, E[i].c);
		}
	}
	for (int i = 1; i <= n; i++) MinCost[i] = make_pair(INF, INF);
	for (int i = 1; i <= m; i++)
	if (E[i].a == S) {
		SE.push_back(i);
		MinCost[F(E[i].b)] = min(MinCost[F(E[i].b)], make_pair(E[i].c, i));
	}
	for (int i = 1; i <= n; i++)
	if (i != S && fa[i] == i) {
		dfs(E[MinCost[i].second].b, S);
		u[MinCost[i].second] = true;
		ans += MinCost[i].first;
	}
}
bool Solve() {
	Kruskal();
	for (int i = cnt + 1; i <= K && i <= n; i++) {
		int MinD = INF, MinID = -1;
		for (int j = (int) SE.size() - 1; j >= 0; j--)
		if (u[SE[j]])
			SE.erase(SE.begin() + j);
		for (int j = 0; j < (int) SE.size(); j++) {
			int tmp = E[SE[j]].c - Cost[Best[E[SE[j]].b]];
			if (tmp < MinD) {
				MinD = tmp;
				MinID= SE[j];
			}
		}
		if (MinID == -1) return false;
		if (MinD >= 0) break;
		ans += MinD;
		u[MinID] = true;
		d[FE[Best[E[MinID].b]]] = d[FE[Best[E[MinID].b]] ^ 1] = true;
		dfs(E[MinID].b, S);
	}
	return true;
}
\end{lstlisting}

	\section{一般图匹配}
		\begin{lstlisting}
const int N = 300;
int n, Next[N], f[N], mark[N], visited[N], Link[N], Q[N], head, tail;
vector <int> E[N];
int getf(int x) { return f[x] == x ? x : f[x] = getf(f[x]); }
void merge(int x, int y) { x = getf(x); y = getf(y); if (x != y) f[x] = y; }
int LCA(int x, int y) {
	static int flag = 0;
	flag++;
	for (; ; swap(x, y)) if (x != -1) {
		x = getf(x);
		if (visited[x] == flag) return x;
		visited[x] = flag;
		if (Link[x] != -1) x = Next[Link[x]];
		else x = -1;
	}
}
void go(int a, int p) {
	while (a != p) {
		int b = Link[a], c = Next[b];
		if (getf(c) != p) Next[c] = b;
		if (mark[b] == 2) mark[Q[tail++] = b] = 1;
		if (mark[c] == 2) mark[Q[tail++] = c] = 1;
		merge(a, b); merge(b, c); a = c;
	}
}
void find(int s) {
	for (int i = 0; i < n; i++) {
		Next[i] = -1; f[i] = i;
		mark[i] = 0; visited[i] = -1;
	}
	head = tail = 0; Q[tail++] = s; mark[s] = 1;
	for (; head < tail && Link[s] == -1; ) {
		for (int i = 0, x = Q[head++]; i < (int)E[x].size(); i++) {
			if (Link[x] != E[x][i] && getf(x) != getf(E[x][i]) && mark[E[x][i]] != 2) {
				int y = E[x][i];
				if (mark[y] == 1) {
					int p = LCA(x, y);
					if (getf(x) != p) Next[x] = y;
					if (getf(y) != p) Next[y] = x;
					go(x, p);
					go(y, p);
				}
				else if (Link[y] == -1) {
					Next[y] = x;
					for (int j = y; j != -1; ) {
						int k = Next[j];
						int tmp = Link[k];
						Link[j] = k;
						Link[k] = j;
						j = tmp;
					}
					break;
				}
				else {
					Next[y] = x;
					mark[Q[tail++] = Link[y]] = 1;
					mark[y] = 2;
				}
			}
		}
	}
}
int main() {
	for (int i = 0; i < n; i++) Link[i] = -1;
	for (int i = 0; i < n; i++) if (Link[i] == -1) {
		find(i);
	}
	int ans = 0;
	for (int i = 0; i < n; i++) ans += Link[i] != -1;
	return ans;
}
	\end{lstlisting}

	%带权的?!
	\section{无向图最小割}
		\begin{lstlisting}
const int V = 100;
#define typec int
const typec inf = 0x3f3f3f; // max of res
const typec maxw = 1000; // maximum edge weight
typec g[V][V], w[V]; //g[i][j] = g[j][i]
int a[V], v[V], na[V];
typec mincut(int n) {
	int i, j, pv, zj;
	typec best = maxw * n * n;
	for (i = 0; i < n; i++) v[i] = i; // vertex: 0 ~ n-1
	while (n > 1) {
		for (a[v[0]] = 1, i = 1; i < n; i++) {
			a[v[i]] = 0; na[i - 1] = i;
			w[i] = g[v[0]][v[i]];
		}
		for (pv = v[0], i = 1; i < n; i++) {
			for (zj = -1, j = 1; j < n; j++)
				if (!a[v[j]] && (zj < 0 || w[j] > w[zj]))
					zj = j;
			a[v[zj]] = 1;
			if (i == n - 1) {
				if (best > w[zj]) best = w[zj];
				for (i = 0; i < n; i++)
					g[v[i]][pv] = g[pv][v[i]] +=
						g[v[zj]][v[i]];
				v[zj] = v[--n];
				break;
			}
			pv = v[zj];
			for (j = 1; j < n; j++)
				if(!a[v[j]])
					w[j] += g[v[zj]][v[j]];
		}
	}
	return best;
}
\end{lstlisting}

	\section{Hamilton回路}
		\begin{lstlisting}
bool graph[N][N];
int n, l[N], r[N], next[N], last[N], s, t;
char buf[10010];
void cover(int x) { l[r[x]] = l[x]; r[l[x]] = r[x]; }
int adjacent(int x) {
	for (int i = r[0]; i <= n; i = r[i]) if (graph[x][i]) return i;
	return 0;
}
int main() {
	scanf("%d\n", &n);
	for (int i = 1; i <= n; ++i) {
		gets(buf);
		string str = buf;
		istringstream sin(str);
		int x;
		while (sin >> x) {
			graph[i][x] = true;
		}
		l[i] = i - 1;
		r[i] = i + 1;
	}
	for (int i = 2; i <= n; ++i)
		if (graph[1][i]) {
			s = 1;
			t = i;
			cover(s);
			cover(t);
			next[s] = t;
			break;
		}
	while (true) {
		int x;
		while (x = adjacent(s)) {
			next[x] = s;
			s = x;
			cover(s);
		}
		while (x = adjacent(t)) {
			next[t] = x;
			t = x;
			cover(t);
		}
		if (!graph[s][t]) {
			for (int i = s, j; i != t; i = next[i])
				if (graph[s][next[i]] && graph[t][i]) {
					for (j = s; j != i; j = next[j])
						last[next[j]] = j;
					j = next[s];
					next[s] = next[i];
					next[t] = i;
					t = j;
					for (j = i; j != s; j = last[j])
						next[j] = last[j];
					break;
				}
		}
		next[t] = s;
		if (r[0] > n)
			break;
		for (int i = s; i != t; i = next[i])
			if (adjacent(i)) {
				s = next[i];
				t = i;
				next[t] = 0;
				break;
			}
	}
	for (int i = s; ; i = next[i]) {
		if (i == 1) {
			printf("%d", i);
			for (int j = next[i]; j != i; j = next[j])
				printf(" %d", j);
			printf(" %d\n", i);
			break;
		}
		if (i == t)
			break;
	}
}
\end{lstlisting}

	\section{弦图判定}
		\begin{lstlisting}
int n, m, first[1001], l, next[2000001], where[2000001],f[1001], a[1001], c[1001], L[1001], R[1001],
v[1001], idx[1001], pos[1001];
bool b[1001][1001];

int read(){
    char ch;
    for (ch = getchar(); ch < '0' || ch > '9'; ch = getchar());
    int cnt = 0;
    for (; ch >= '0' && ch <= '9'; ch = getchar()) cnt = cnt * 10 + ch - '0';
    return(cnt);
}

inline void makelist(int x, int y){
    where[++l] = y;
    next[l] = first[x];
    first[x] = l;
}

bool cmp(const int &x, const int &y){
    return(idx[x] < idx[y]);
}

int main(){
   //freopen("1015.in", "r", stdin);
   // freopen("1015.out", "w", stdout);
    for (;;)
    {
        n = read(); m = read();
        if (!n && !m) return 0;
        memset(first, 0, sizeof(first)); l = 0;
        memset(b, false, sizeof(b));
        for (int i = 1; i <= m; i++) 
        {
            int x = read(), y = read();
            if (x != y && !b[x][y])
            {
               b[x][y] = true; b[y][x] = true;
               makelist(x, y); makelist(y, x);
            }
        }
        memset(f, 0, sizeof(f));
        memset(L, 0, sizeof(L));
        memset(R, 255, sizeof(R));
        L[0] = 1; R[0] = n;
        for (int i = 1; i <= n; i++) c[i] = i, pos[i] = i;
        memset(idx, 0, sizeof(idx));
        memset(v, 0, sizeof(v));
        for (int i = n; i; --i)
        {
            int now = c[i];
            R[f[now]]--;
            if (R[f[now]] < L[f[now]]) R[f[now]] = -1;
            idx[now] = i; v[i] = now;
            for (int x = first[now]; x; x = next[x])
                if (!idx[where[x]]) 
                {
                   swap(c[pos[where[x]]], c[R[f[where[x]]]]);
                   pos[c[pos[where[x]]]] = pos[where[x]];
                   pos[where[x]] = R[f[where[x]]];
                   L[f[where[x]] + 1] = R[f[where[x]]]--;
                   if (R[f[where[x]]] < L[f[where[x]]]) R[f[where[x]]] = -1;
                   if (R[f[where[x]] + 1] == -1)
                       R[f[where[x]] + 1] = L[f[where[x]] + 1];
                   ++f[where[x]];
                }
        }
        bool ok = true;
        //v是完美消除序列.
        for (int i = 1; i <= n && ok; i++)
        {
            int cnt = 0;
            for (int x = first[v[i]]; x; x = next[x]) 
                if (idx[where[x]] > i) c[++cnt] = where[x];
            sort(c + 1, c + cnt + 1, cmp);
            bool can = true;
            for (int j = 2; j <= cnt; j++)
                if (!b[c[1]][c[j]])
                {
                    ok = false;
                    break;
                }
        }
        if (ok) printf("Perfect\n");
        else printf("Imperfect\n");
        printf("\n");
    }
}
\end{lstlisting}

	\section{弦图求团数}
		\begin{lstlisting}
int n, m, first[100001], next[2000001], where[2000001], l, L[100001], R[100001], c[100001], f[100001],
pos[100001], idx[100001], v[100001], ans;

inline void makelist(int x, int y){
    where[++l] = y;
    next[l] = first[x];
    first[x] = l;
}

int read(){
    char ch;
    for (ch = getchar(); ch < '0' || ch > '9'; ch = getchar());
    int cnt = 0;
    for (; ch >= '0' && ch <= '9'; ch = getchar()) cnt = cnt * 10 + ch - '0';
    return(cnt);
}

int main(){
    freopen("1006.in", "r", stdin);
    freopen("1006.out", "w", stdout);
    memset(first, 0, sizeof(first)); l = 0;
    n = read(); m = read();
    for (int i = 1; i <= m; i++)
    {
        int x, y;
        x = read(); y = read();
        makelist(x, y); makelist(y, x);
    }
    memset(L, 0, sizeof(L));
    memset(R, 255, sizeof(R));
    memset(f, 0, sizeof(f));
    memset(idx, 0, sizeof(idx));
    for (int i = 1; i <= n; i++) c[i] = i, pos[i] = i;
    L[0] = 1; R[0] = n; ans = 0;
    for (int i = n; i; --i)
    {
        int now = c[i], cnt = 1;
        idx[now] = i; v[i] = now;
        if (--R[f[now]] < L[f[now]]) R[f[now]] = -1;
        for (int x = first[now]; x; x = next[x])
            if (!idx[where[x]])
            {
                swap(c[pos[where[x]]], c[R[f[where[x]]]]);
                pos[c[pos[where[x]]]] = pos[where[x]];
                pos[where[x]] = R[f[where[x]]];
                L[f[where[x]] + 1] = R[f[where[x]]]--;
                if (R[f[where[x]]] < L[f[where[x]]]) R[f[where[x]]] = -1;
                if (R[f[where[x]] + 1] == -1) R[f[where[x]] + 1] = L[f[where[x]] + 1];
                ++f[where[x]];
            }
            else ++cnt;
        ans = max(ans, cnt);
    }
    printf("%d\n", ans);
}
\end{lstlisting}

	\section{有根树的同构}
		\begin{lstlisting}
//http://acm.sdut.edu.cn/judgeonline/showproblem?problem_id=1861 �и�����ͬ�� 
#include <cstdio>
#include <cstdlib>
#include <cstring>
#include <ctime>

using namespace std;

const int mm=1051697,p=4773737;
int m,n,first[101],where[10001],next[10001],l,hash[10001],size[10001],pos[10001];
long long f[10001],rt[10001];
bool in[10001];


inline void makelist(int x,int y){
    where[++l]=y;
    next[l]=first[x];
    first[x]=l;
}


inline void hashwork(int now){
    int a[1001],v[1001],tot=0;
    size[now]=1;
    for (int x=first[now];x;x=next[x])
    {
        hashwork(where[x]);
        a[++tot]=f[where[x]];
        v[tot]=size[where[x]];
        size[now]+=size[where[x]];
    }
    a[++tot]=size[now];
    v[tot]=1;
    int len=0;
    for (int i=1;i<=tot;i++) 
       for (int j=i+1;j<=tot;j++)
          if (a[j]<a[i])
          {
             int u=a[i];a[i]=a[j];a[j]=u;
             u=v[i];v[i]=v[j];v[j]=u;
          }
    f[now]=1;
    for (int i=1;i<=tot;i++)
       {
             f[now]=((f[now]*a[i])%p*rt[len])%p;
             len+=v[i];
       }
}

int main(){
    //freopen("1.txt","r",stdin);
    //freopen("2.txt","w",stdout);
    scanf("%d%d",&n,&m);
    rt[0]=1;
    for (int i=1;i<=100;i++)
        rt[i]=(rt[i-1]*mm)%p; 
    for (int i=1;i<=n;i++)
    {
        memset(first,0,sizeof(first));
        memset(in,false,sizeof(in));
        l=0;
        for (int j=1;j<m;j++)
        {
            int x,y;
            scanf("%d%d",&x,&y);
            makelist(x,y);
            in[y]=true;
        }
        int root=0;
        for (int j=1;j<=m;j++)
        if (!in[j]) 
        {
            root=j;
            break;
        }
        memset(size,0,sizeof(size));
        memset(f,0,sizeof(f));
        hashwork(root);
        hash[i]=f[root];
    }
    for (int i=1;i<=n;i++) pos[i]=i;
    memset(in,false,sizeof(in));
    for (int i=1;i<=n;i++)
     if (!in[i])
     {
                printf("%d",i);
                for (int j=i+1;j<=n;j++)
                if (hash[j]==hash[i])
                {
                    in[j]=true;
                    printf("=%d",j);
                }       
                printf("\n");
     }
}
\end{lstlisting}           

	\section{zkw费用流}
		\begin{lstlisting}
#include <cstdio>
#include <cstdlib>
#include <algorithm>
#include <cstring>
#include <cmath>
using namespace std;

const int N = 105 << 2, M = 205 * 205 * 2;
const int inf = 1000000000;

struct eglist {
	int other[M], succ[M], last[N], cap[M], cost[M], sum;
	void clear() {
		memset(last, -1, sizeof(last));
		sum = 0;
	}
	void _addEdge(int a, int b, int c, int d) {
		other[sum] = b, succ[sum] = last[a], last[a] = sum, cost[sum] = d, cap[sum++] = c;
	}
	void addEdge(int a, int b, int c, int d) {
		_addEdge(a, b, c, d);
		_addEdge(b, a, 0, -d);
	}
}e;

int n, m, S, T, tot, totFlow, totCost;
int dis[N], slack[N], visit[N], cur[N];

int modlable() {
	int delta = inf;
	for(int i = 1; i <= T; i++) {
		if (!visit[i] && slack[i] < delta)
			delta = slack[i];
		slack[i] = inf;
		cur[i] = e.last[i];
	}
	if (delta == inf)
		return 1;
	for(int i = 1; i <= T; i++)
		if (visit[i])
			dis[i] += delta;
	return 0;
}

int dfs(int x, int flow) {
	if (x == T) {
		totFlow += flow;
		totCost += flow * (dis[S] - dis[T]);
		return flow;
	}
	visit[x] = 1;
	int left = flow;
	for(int &i = cur[x]; ~i; i = e.succ[i])
		if (e.cap[i] > 0 && !visit[e.other[i]]) {
			int y = e.other[i];
			if (dis[y] + e.cost[i] == dis[x]) {
				int delta = dfs(y, min(left, e.cap[i]));
				e.cap[i] -= delta;
				e.cap[i ^ 1] += delta;
				left -= delta;
				if (!left)
					return flow;
			} else {
				slack[y] = min(slack[y], dis[y] + e.cost[i] - dis[x]);
			}
		}
	return flow - left;
}

pair<int, int> minCost() {
	totFlow = 0, totCost = 0;
	fill(dis + 1, dis + T + 1, 0);
	for(int i = 1; i <= T; i++)
		cur[i] = e.last[i];
	do {
		do {
			fill(visit + 1, visit + T + 1, 0);
		} while(dfs(S, inf));
	} while(!modlable());
	return make_pair(totFlow, totCost);
}

void run() {
	scanf("%d %d", &m, &n);
	e.clear();
	S = m + n + 1, T = m + n + 2;
	tot = 0;
	for(int i = 1; i <= m; i++) {
		int times;
		scanf("%d", &times);
		e.addEdge(S, i, times, 0);
	}
	for(int i = 1; i <= n; i++) {
		int times;
		scanf("%d", &times);
		e.addEdge(i + m, T, times, 0);
	}
	for(int i = 1; i <= m; i++)
		for(int j = 1; j <= n; j++) {
			int cost;
			scanf("%d", &cost);
			e.addEdge(i, j + m, inf, cost);
		}
	pair<int, int> tmp = minCost();
	printf("%d\n", tmp.second);
}

int main() {
	int Test;
	scanf("%d", &Test);
	for(; Test--; run());
	return 0;
}
\end{lstlisting}

	%\section{图同构哈希}
	%!!!
\chapter{字符串}
	\section{扩展KMP}
		传入字符串s和长度N,next[i]=LCP(s, s[i..N-1])
\begin{lstlisting}
void z(char *s, int *next, int N)
{
	int j = 0, k = 1;
	while (j + 1 < N && s[j] == s[j + 1]) ++ j;
	next[0] = N - 1; next[1] = j;
	for(int i = 2; i < N; ++ i) {
		int far = k + next[k] - 1, L = next[i - k];
		if (L < far - i + 1) next[i] = L;
		else {
			j = max(0, far - i + 1);
			while (i + j < N && s[j] == s[i + j]) ++ j;
			next[i] = j; k = i;
		}
	}
}
\end{lstlisting}

	\section{后缀数组}
		字符串后面会自动加上一个最小字符\texttt{$\backslash$0}.
	\begin{lstlisting}
const int N = 4 * int(1e5) + 10;

int n, m;
int sa[N], ta[N], tb[N], *rank = ta, *tmp = tb;
int height[N], myLog[N], f[N][20];
int str[N];

bool cmp(int i, int j, int l) {
	return tmp[i] == tmp[j] && tmp[i + l] == tmp[j + l];
}

void radixSort() {
	static int w[N];
	fill(w, w + m, 0);
	for (int i = 0; i < n; i++) {
		w[rank[i]]++;
	}
	for (int i = 1; i < m; i++) {
		w[i] += w[i - 1];
	}
	for (int i = n - 1; i >= 0; i--) {
		sa[--w[rank[tmp[i]]]] = tmp[i];
	}
}

void suffixArray() {
	for (int i = 0; i < n; i++) {
		rank[i] = str[i];
		tmp[i] = i;
	}
	radixSort();
	for (int j = 1, i, p; j < n; j <<= 1, m = p) {
		for (i = n - j, p = 0; i < n; i++) {
			tmp[p++] = i;
		}
		for (i = 0; i < n; i++) {
			if (sa[i] >= j) {
				tmp[p++] = sa[i] - j;
			}
		}
		radixSort();
		for (swap(tmp, rank), rank[sa[0]] = 0, i = p = 1; i < n; i++) {
			rank[sa[i]] = cmp(sa[i - 1], sa[i], j) ? p - 1 : p++;
		}
	}
	for (int i = 0, j, k = 0; i < n; ++i, k = max(k - 1, 0)) {
		if (rank[i]) {
			j = sa[rank[i] - 1];
			for (; str[i + k] == str[j + k]; k++);
			height[rank[i]] = k;
		}
	}
	for (int i = 2; i <= n; i++) {
		myLog[i] = myLog[i >> 1] + 1;
	}
	for (int i = 1; i < n; i++) {
		f[i][0] = height[i];
	}
	for (int j = 1; 1 << j <= n; j++) {
		for (int i = 1; i + (1 << j) <= n; i++) {
			f[i][j] = min(f[i][j - 1], f[i + (1 << j - 1)][j - 1]);
		}
	}
}

int lcp(int l, int r) {
	if (l > r) {
		return 0;
	}
	int len = myLog[r - l + 1];
	return min(f[l][len], f[r - (1 << len) + 1][len]);
}

int nBase, mBase;
int cnt[N];
char buf[N];

int pos(int x) {
	return x / (mBase << 1 | 1 );
}

int main() {
	n = 0;
	m = 256;
	scanf("%d%d", &nBase, &mBase);
	for (int i = 0; i < nBase; i++) {
		scanf("%s", buf);
		for (int j = 0; j < mBase; j++) {
			str[n++] = buf[j];
		}
		for (int j = 0; j < mBase; j++) {
			str[n++] = buf[j];
		}
		str[n++] = i < nBase - 1 ? m++ : 0;
	}
	suffixArray();
	int result = 0, total = 0;
	for (int i = 0, j = 0; i < n; i++) {
		for (; j < n && total < nBase; j++) {
			int p = pos(sa[j]);
			total += cnt[p]++ == 0;
		}
		if (total == nBase) {
			result = max(result, lcp(i + 1, j - 1));
		}
		int p = pos(sa[i]);
		total -= --cnt[p] == 0;
	}
	result = min(result, mBase);
	printf("%d\n", result);
	vector <int> ans(n);
	total = 0;
	memset(cnt, 0, sizeof(cnt));
	for (int i = 0, j = 0; i < n; i++) {
		for (; j < n && total < nBase; j++) {
			int p = pos(sa[j]);
			total += cnt[p]++ == 0;
		}
		if (total == nBase && lcp(i + 1, j - 1) >= result) {
			for (int k = i; k < j; k++) {
				int p = pos(sa[k]);
				ans[p] = sa[k] % (mBase << 1 | 1);
			}
			break;
		}
		int p = pos(sa[i]);
		total -= --cnt[p] == 0;
	}
	for (int i = 0; i < nBase; i++) {
		printf("%d\n", ans[i] % mBase + 1);
	}
}
	\end{lstlisting}

	\section{DC3}
		\begin{lstlisting}
//`DC3 待排序的字符串放在r 数组中,从r[0]到r[n-1],长度为n,且最大值小于m.`
//`约定除r[n-1]外所有的r[i]都大于0, r[n-1]=0。`
//`函数结束后,结果放在sa 数组中,从sa[0]到sa[n-1]。`
//`r必须开长度乘3`
#define maxn 10000
#define F(x) ((x)/3+((x)%3==1?0:tb))
#define G(x) ((x)<tb?(x)*3+1:((x)-tb)*3+2)

int wa[maxn],wb[maxn],wv[maxn],wss[maxn];
int s[maxn*3],sa[maxn*3];
int c0(int *r,int a,int b)
{
	return r[a]==r[b]&&r[a+1]==r[b+1]&&r[a+2]==r[b+2];
}
int c12(int k,int *r,int a,int b)
{
	if(k==2) return r[a]<r[b]||r[a]==r[b]&&c12(1,r,a+1,b+1);
	else return r[a]<r[b]||r[a]==r[b]&&wv[a+1]<wv[b+1];
}
void sort(int *r,int *a,int *b,int n,int m)
{
	int i;
	for(i=0;i<n;i++) wv[i]=r[a[i]];
	for(i=0;i<m;i++) wss[i]=0;
	for(i=0;i<n;i++) wss[wv[i]]++;
	for(i=1;i<m;i++) wss[i]+=wss[i-1];
	for(i=n-1;i>=0;i--) b[--wss[wv[i]]]=a[i];
}
void dc3(int *r,int *sa,int n,int m)
{
	int i,j,*rn=r+n,*san=sa+n,ta=0,tb=(n+1)/3,tbc=0,p;
	r[n]=r[n+1]=0;
	for(i=0;i<n;i++)
		if(i%3!=0) wa[tbc++]=i;
	sort(r+2,wa,wb,tbc,m);
	sort(r+1,wb,wa,tbc,m);
	sort(r,wa,wb,tbc,m);
	for(p=1,rn[F(wb[0])]=0,i=1;i<tbc;i++)
		rn[F(wb[i])]=c0(r,wb[i-1],wb[i])?p-1:p++;
	if (p<tbc) dc3(rn,san,tbc,p);
	else for (i=0;i<tbc;i++) san[rn[i]]=i;
	for (i=0;i<tbc;i++)
		if(san[i]<tb) wb[ta++]=san[i]*3;
	if(n%3==1) wb[ta++]=n-1;
	sort(r,wb,wa,ta,m);
	for(i=0;i<tbc;i++)
		wv[wb[i]=G(san[i])]=i;
	for(i=0,j=0,p=0;i<ta && j<tbc;p++)
		sa[p]=c12(wb[j]%3,r,wa[i],wb[j])?wa[i++]:wb[j++];
	for(;i<ta;p++) sa[p]=wa[i++];
	for(;j<tbc;p++) sa[p]=wb[j++];
}

int main(){
	int n,m=0;
	scanf("%d",&n);
	for (int i=0;i<n;i++) scanf("%d",&s[i]),s[i]++,m=max(s[i]+1,m);
	printf("%d\n",m);
	s[n++]=0;
	dc3(s,sa,n,m);
	for (int i=0;i<n;i++) printf("%d ",sa[i]);printf("\n");
}
	\end{lstlisting}

	\section{AC自动机}
		\begin{lstlisting}
namespace aho_corasick_automation {
	int const N = ;
	struct node {
		node *next[N], *fail;
		int count;
		inline node() {
			memset(next, 0, sizeof(next));
			fail = 0;
			count = 0;
		}
	};
	
	node *root;
	
	inline int idx(char x) {
		return x - 'a';
	}
	
	inline void insert(node *x, char *str) {
		int len = (int)strlen(str);
		for (int i = 0; i < len; ++i) {
			int c = idx(str[i]);
			if (!x->next[c]) {
				x->next[c] = new node();
			}
			x = x->next[c];
		}
		x->count++;
	}
	
	inline void build() {
		vector<node*> queue;
		queue.push_back(root->fail = root);
		for (int head = 0; head < (int)queue.size(); ++head) {
			node* x = queue[head];
			for (int i = 0; i < N; ++i) {
				if (x->next[i]) {
					x->next[i]->fail = (x == root) ? root : x->fail->next[i];
					x->next[i]->count += x->next[i]->fail->count;
					queue.push_back(x->next[i]);
				} else {
					x->next[i] = (x == root) ? root : x->fail->next[i];
				}
			}
		}
	}
	
	inline void prepare() {
		root = new node();
	}
}
\end{lstlisting}

	\section{极长回文子串}
		\begin{lstlisting}
//CF17 - E
typedef long long int64;
const int N = 4 * int(1e6) + 111;
const int mod = 51123987;
int n;
int input[N];
int start[N], finish[N];
int f[N];
int64 ans;
void prepare() {
	int k = 0;
	for (int i = 0; i < n; ++i) {
		if (k + f[k] < i) {
			int &l = f[i] = 0;
			for (; i - l - 1 >= 0 && i + l + 1 < n && input[i - l - 1] ==
					 input[i + l + 1]; l++);
			k = i;
		} else {
			int &l = f[i] = f[k - (i - k)];
			if (i + l >= k + f[k]) {
				l = min(l, k + f[k] - i);
				for (; i - l - 1 >= 0 && i + l + 1 < n && input[i - l - 1] ==
						 input[i + l + 1]; l++);
				k = i;
			}
		}
		int l = i - f[i], r = i + f[i];
		l += l & 1;
		r -= r & 1;
		if (l <= r) {
			l /= 2;
			r /= 2;
			int mid1 = l + r >> 1;
			int mid2 = mid1 + ((l + r) & 1);
			start[l]++;
			start[mid1 + 1]--;
			finish[mid2]++;
			finish[r + 1]--;
			ans = (ans + (r - l) / 2 + 1) % mod;
		}
	}
}
int main() {
	scanf("%d ", &n);
	for (int i = 0; i < n; ++i) {
		input[i << 1] = getchar();
		if (i < n - 1)
			input[i << 1 | 1] = '*';
	}
	n = n * 2 - 1;
	prepare();
	ans = ans * (ans - 1) / 2 % mod;
	n = (n + 1) / 2;
	int sum = 0;
	for (int i = 0; i < n; ++i) {
		if (i) {
			start[i] = (start[i] + start[i - 1]) % mod;
			finish[i] = (finish[i] + finish[i - 1]) % mod;
		}
		ans = (ans - (int64)start[i] * sum % mod) % mod;
		sum = (sum + finish[i]) % mod;
	}
	cout << (ans + mod) % mod << endl;
}
	\end{lstlisting}

	\section{后缀自动机--多个串的最长公共子串}
		\begin{lstlisting}

const int N = 255555;
const int C = 36;

struct Node {
	Node *next[C], *fail;
	int count, len, dp, dp2;
	void clear() {
		for(int i = 0; i < C; i++)
			next[i] = NULL;
		len = count = 0;
		fail = NULL;
	}
};

Node *tail, *q[N * 2], pool[N * 2], *head;
int used = 0, top = 0;
char bufer[N * 2];

Node *newNode() {
	pool[used++].clear();
	return &pool[used - 1];
}

void add(int x) {
	Node *np = newNode(), *p = tail;
	tail = np;
	np->len = p->len + 1;
	for(; p && !p->next[x]; p = p->fail)
		p->next[x] = np;
	if (!p)
		np->fail = head;
	else if (p->len + 1 == p->next[x]->len)
		np->fail = p->next[x];
	else {
		Node *q = p->next[x], *nq = newNode();
		*nq = *q; 
		nq->len = p->len + 1;
		q->fail = np->fail = nq;
		for(; p && p->next[x] == q; p = p->fail)
			p->next[x] = nq;
	}
}

int main() {
	scanf("%s", bufer);
	int length = strlen(bufer);
	head = tail = newNode();
	for(int i = 0; i < length; i++)
		add(bufer[i] - 'a');
	for(int i = 0; i < used; i++)
		pool[i].count = 0, pool[i].dp = pool[i].len;
	int number = 0;
	while(scanf("%s", bufer) == 1) {
		number++;
		length = strlen(bufer);
		Node *iter = head;
		int cur = 0;
		top = 0;
		for(int i = 0; i < length; i++) {
			int x = bufer[i] - 'a';
			while(iter != head && !iter->next[x])
				iter = iter->fail, cur = iter->len;
			if (iter->next[x]) {
				cur++;
				iter = iter->next[x];
			}
			q[top++] = iter; 
			if (iter->count == number - 1) {
				iter->count = number;
				iter->dp2 = cur;
			} else if (iter->count == number) {
				iter->dp2 = max(iter->dp2, cur);
			} else {
				top--;
			}
		}	
		for(int i = 0; i < top; i++) {
			q[i]->dp = min(q[i]->dp, q[i]->dp2);
		}
	}	
	int ans = 0;
	for(int i = 0; i < used; i++)
		if (pool[i].count == number)
			ans = max(ans, pool[i].dp);
	printf("%d\n", ans);
	return 0;
}
	\end{lstlisting}

	\section{后缀自动机--多次询问串在母串中的出现次数}
		\begin{lstlisting}

const int N = 255555;
const int C = 36;

struct Node {
	Node *next[C], *fail;
	int count, len;
	void clear() {
		for(int i = 0; i < C; i++)
			next[i] = NULL;
		len = count = 0;
		fail = NULL;
	}
};

Node *tail, *q[N * 2], pool[N * 2], *head;
int used = 0;
char bufer[N * 2];
int buc[N * 2], f[N * 2];

Node *newNode() {
	pool[used++].clear();
	return &pool[used - 1];
}

void add(int x) {
	Node *np = newNode(), *p = tail;
	tail = np;
	np->len = p->len + 1;
	for(; p && !p->next[x]; p = p->fail)
		p->next[x] = np;
	if (!p)
		np->fail = head;
	else if (p->len + 1 == p->next[x]->len)
		np->fail = p->next[x];
	else {
		Node *q = p->next[x], *nq = newNode();
		*nq = *q; 
		nq->len = p->len + 1;
		q->fail = np->fail = nq;
		for(; p && p->next[x] == q; p = p->fail)
			p->next[x] = nq;
	}
}

int main() {
	scanf("%s\n", bufer);
	int length = strlen(bufer);
	head = tail = newNode();
	for(int i = 0; i < length; i++)
		add(bufer[i] - 'a');
	for(int i = 0; i < used; ++i) 
		++buc[pool[i].len];
	for(int i = 1; i <= length; i++)
		buc[i] += buc[i - 1];
	for(int i = used - 1; i >= 0; i--)
		q[--buc[pool[i].len]] = &pool[i];
	Node *iter = head;
	for(int i = 0; i < length; ++i)
		(iter = iter->next[bufer[i] - 'a'])->count++;
	for(int i = used - 1; i > 0; --i) {
		f[q[i]->len] = max(f[q[i]->len], q[i]->count);
		q[i]->fail->count += q[i]->count;
	}
	for(int i = length - 1; i > 0; --i) {
		f[i] = max(f[i + 1], f[i]);
	}
	for(int i = 1; i <= length; i++)
		printf("%d\n", f[i]);
	return 0;
}
	\end{lstlisting}

	\section{循环串的最小表示}
		\begin{lstlisting}
struct cyc_string
{
	int n, offset;
	char str[max_length];
	char & operator [] (int x)
	{return str[((offset + x) % n)];}
	cyc_string(){offset = 0;}
};
void minimum_circular_representation(cyc_string & a)
{
	int i = 0, j = 1, dlt = 0, n = a.n;
	while(i < n and j < n and dlt < n)
	{
	  if(a[i + dlt] == a[j + dlt]) dlt++;
	  else
	  {
	    if(a[i + dlt] > a[j + dlt]) i += dlt + 1; else j += dlt + 1;
	    dlt = 0;
	  }
	}
	a.offset = min(i, j);
}
int main()
{return 0;}
\end{lstlisting}

\chapter{Others}
	\section{快速求逆}
		\begin{lstlisting}
int inverse(int x, int modulo) {
	if(x == 1)
		return 1;
	return (long long)(modulo - modulo / x) * inverse(modulo % x, modulo) % modulo;
}
\end{lstlisting}

	\section{求某年某月某日星期几}
		\begin{lstlisting}
int whatday(int d, int m, int y)
{
	int ans;
	if (m == 1 || m == 2) {
		m += 12; y --;
	}
	if ((y < 1752) || (y == 1752 && m < 9) || (y == 1752 && m == 9 && d < 3))
		ans = (d + 2 * m + 3 * (m + 1) / 5 + y + y / 4 + 5) % 7;
	else ans = (d + 2 * m + 3 * (m + 1) / 5 + y + y / 4 - y / 100 + y / 400) % 7;
	return ans;
}
\end{lstlisting}

	\section{LL*LL\%LL} 
		\begin{lstlisting}
LL multiplyMod(LL a, LL b, LL P) { // `需要保证 a 和 b 非负`
	LL t = (a * b - LL((long double)a / P * b + 1e-3) * P) % P;
	return t < 0 : t + P : t;
}
\end{lstlisting}

	\section{next\_nCk}
		\begin{lstlisting}
void nCk(int n, int k) {
	for (int comb = (1 << k) - 1; comb < (1 << n); ) {
		// ...
		{
			int x = comb & -comb, y = comb + x;
			comb = (((comb & ~y) / x) >> 1) | y;
		}
	}
}
\end{lstlisting}

	\section{单纯形}
		test on uva 12567
\begin{lstlisting}
const double eps = 1e-8;
// max{c * x | Ax <= b, x >= 0}的解, 无解返回空的vector, 否则就是解. 
vector<double> simplex(vector<vector<double> > &A, vector<double> b, vector<double> c) {
	int n = A.size(), m = A[0].size() + 1, r = n, s = m - 1;
	vector<vector<double> > D(n + 2, vector<double>(m + 1));
	vector<int> ix(n + m);
	for(int i = 0; i < n + m; i++) ix[i] = i;
	for(int i = 0; i < n; i++) {
		for(int j = 0; j < m - 1; j++)
			D[i][j] = -A[i][j];
		D[i][m - 1] = 1; D[i][m] = b[i];
		if (D[r][m] > D[i][m])
			r = i;
	}
	for(int j = 0; j < m - 1; j++) D[n][j] = c[j];
	D[n + 1][m - 1] = -1;
	for(double d; ;) {
		if (r < n) {
			swap(ix[s], ix[r + m]);
			D[r][s] = 1. / D[r][s];
			for(int j = 0; j <= m; j++)
				if (j != s) D[r][j] *= -D[r][s];
			for(int i = 0; i <= n + 1; i++)
				if (i != r) {
					for(int j = 0; j <= m; j++)
						if (j != s) D[i][j] += D[r][j] * D[i][s];
					D[i][s] *= D[r][s];
				}
		}
		r = -1, s = -1;
		for(int j = 0; j < m; j++)
			if (s < 0 || ix[s] > ix[j])
				if (D[n + 1][j] > eps || D[n + 1][j] > -eps && D[n][j] > eps)
					s = j;
		if (s < 0) break;
		for(int i = 0; i < n; i++)
			if (D[i][s] < -eps)
				if (r < 0 || (d = D[r][m] / D[r][s] - D[i][m] / D[i][s]) < -eps || d < eps && ix[r + m] > ix[i + m])
					r = i;
		if (r < 0) return vector<double> ();
	}
	if (D[n + 1][m] < -eps) return vector<double> ();
	vector<double> x(m - 1);
	for(int i = m; i < n + m; i++)
		if (ix[i] < m - 1)
			x[ix[i]] = D[i - m][m];
	return x;
}
\end{lstlisting}
	\section{曼哈顿最小生成树}
		\begin{lstlisting}
/*
`只需要考虑每个点的 pi/4*k -- pi/4*(k+1)的区间内的第一个点,这样只有4n条无向边。`
*/
const int maxn = 100000+5;
const int Inf = 1000000005;
struct TreeEdge
{
	int x,y,z;
	void make( int _x,int _y,int _z ) { x=_x; y=_y; z=_z; }
} data[maxn*4];

inline bool operator < ( const TreeEdge& x,const TreeEdge& y ){
	return x.z<y.z;
}

int x[maxn],y[maxn],px[maxn],py[maxn],id[maxn],tree[maxn],node[maxn],val[maxn],fa[maxn];
int n;
inline bool compare1( const int a,const int b ) { return x[a]<x[b]; }
inline bool compare2( const int a,const int b ) { return y[a]<y[b]; }
inline bool compare3( const int a,const int b ) { return (y[a]-x[a]<y[b]-x[b] || y[a]-x[a]==y[b]-x[b] && y[a]>y[b]); }
inline bool compare4( const int a,const int b ) { return (y[a]-x[a]>y[b]-x[b] || y[a]-x[a]==y[b]-x[b] && x[a]>x[b]); }
inline bool compare5( const int a,const int b ) { return (x[a]+y[a]>x[b]+y[b] || x[a]+y[a]==x[b]+y[b] && x[a]<x[b]); }
inline bool compare6( const int a,const int b ) { return (x[a]+y[a]<x[b]+y[b] || x[a]+y[a]==x[b]+y[b] && y[a]>y[b]); }
void Change_X()
{
	for(int i=0;i<n;++i) val[i]=x[i];
	for(int i=0;i<n;++i) id[i]=i;
	sort(id,id+n,compare1);
	int cntM=1, last=val[id[0]]; px[id[0]]=1;
	for(int i=1;i<n;++i)
	{
		if(val[id[i]]>last) ++cntM,last=val[id[i]];
		px[id[i]]=cntM;
	}
}
void Change_Y()
{
	for(int i=0;i<n;++i) val[i]=y[i];
	for(int i=0;i<n;++i) id[i]=i;
	sort(id,id+n,compare2);
	int cntM=1, last=val[id[0]]; py[id[0]]=1;
	for(int i=1;i<n;++i)
	{
		if(val[id[i]]>last) ++cntM,last=val[id[i]];
		py[id[i]]=cntM;
	}
}
inline int absValue( int x ) { return (x<0)?-x:x; }
inline int Cost( int a,int b ) { return absValue(x[a]-x[b])+absValue(y[a]-y[b]); }
int find( int x ) { return (fa[x]==x)?x:(fa[x]=find(fa[x])); }
int main()
{
//	freopen("input.txt", "r", stdin);
//	freopen("output.txt", "w", stdout);
	
	int test=0;
	while( scanf("%d",&n)!=EOF && n )
	{
		for(int i=0;i<n;++i) scanf("%d%d",x+i,y+i);
		Change_X();
		Change_Y();
		
		int cntE = 0;
		for(int i=0;i<n;++i) id[i]=i;
		sort(id,id+n,compare3);
		for(int i=1;i<=n;++i) tree[i]=Inf,node[i]=-1;
		for(int i=0;i<n;++i)
		{
			int Min=Inf, Tnode=-1;
			for(int k=py[id[i]];k<=n;k+=k&(-k)) if(tree[k]<Min) Min=tree[k],Tnode=node[k];
			if(Tnode>=0) data[cntE++].make(id[i],Tnode,Cost(id[i],Tnode));
			int tmp=x[id[i]]+y[id[i]];
			for(int k=py[id[i]];k;k-=k&(-k)) if(tmp<tree[k]) tree[k]=tmp,node[k]=id[i];
		}
		sort(id,id+n,compare4);
		for(int i=1;i<=n;++i) tree[i]=Inf,node[i]=-1;
		for(int i=0;i<n;++i)
		{
			int Min=Inf, Tnode=-1;
			for(int k=px[id[i]];k<=n;k+=k&(-k)) if(tree[k]<Min) Min=tree[k],Tnode=node[k];
			if(Tnode>=0) data[cntE++].make(id[i],Tnode,Cost(id[i],Tnode));
			int tmp=x[id[i]]+y[id[i]];
			for(int k=px[id[i]];k;k-=k&(-k)) if(tmp<tree[k]) tree[k]=tmp,node[k]=id[i];			
		}
		sort(id,id+n,compare5);
		for(int i=1;i<=n;++i) tree[i]=Inf,node[i]=-1;
		for(int i=0;i<n;++i)
		{
			int Min=Inf, Tnode=-1;
			for(int k=px[id[i]];k;k-=k&(-k)) if(tree[k]<Min) Min=tree[k],Tnode=node[k];
			if(Tnode>=0) data[cntE++].make(id[i],Tnode,Cost(id[i],Tnode));
			int tmp=-x[id[i]]+y[id[i]];
			for(int k=px[id[i]];k<=n;k+=k&(-k)) if(tmp<tree[k]) tree[k]=tmp,node[k]=id[i];	
		}
		sort(id,id+n,compare6);
		for(int i=1;i<=n;++i) tree[i]=Inf,node[i]=-1;
		for(int i=0;i<n;++i)
		{
			int Min=Inf, Tnode=-1;
			for(int k=py[id[i]];k<=n;k+=k&(-k)) if(tree[k]<Min) Min=tree[k],Tnode=node[k];
			if(Tnode>=0) data[cntE++].make(id[i],Tnode,Cost(id[i],Tnode));
			int tmp=-x[id[i]]+y[id[i]];
			for(int k=py[id[i]];k;k-=k&(-k)) if(tmp<tree[k]) tree[k]=tmp,node[k]=id[i];
		}
		
		long long Ans = 0;
		sort(data,data+cntE);
		for(int i=0;i<n;++i) fa[i]=i;
		for(int i=0;i<cntE;++i) if(find(data[i].x)!=find(data[i].y))
		{
			Ans += data[i].z;
			fa[fa[data[i].x]]=fa[data[i].y];
		}
		
		cout<<"Case "<<++test<<": "<<"Total Weight = "<<Ans<<endl;
	}
	return 0;
}
	\end{lstlisting}

	\section{最长公共子序列}
		\subsection{最长公共子序列} %{{{
	\begin{lstlisting}
const int dx[]={0,-1,0,1};
const int dy[]={1,0,-1,0};
const string ds="ENWS";
char G[52][52];
char A[22222], B[22222], buf[22222];
int n, m;

typedef unsigned long long ll;

const int M = 62;
const int maxn = 20010;
const int maxt = 130;
const int maxl = maxn / M + 10;
const ll Top = ((ll) 1 << (M));
const ll Topless = Top - 1;
const ll underTop = ((ll) 1 << (M - 1));
typedef ll bitarr[maxl];
bitarr comp[maxt], row[2], X;

void get(char *S){
	int L,x,y,sz=0;
	scanf("%d%d%d",&L,&x,&y),x--,y--;
	//scanf(" %s",buf);
	S[sz++]=G[x][y];
	for(int i=0;i<L;i++){
		char ch;
		scanf(" %c", &ch);
		int pos=ds.find(ch);
		x+=dx[pos],y+=dy[pos];
		if (x < 0 || y < 0 || x >= n || y >= m) for(;;);
		S[sz++]=G[x][y];
	}
	S[sz]=0;
}

bool calc[maxt];

void prepare() {
	
	int u, p;	
	memset(calc, 0, sizeof(calc));	
	for (int i = 0; i < m; i++) {
		u = B[i];
		if (calc[u]) continue;	//======仅对所有字符集 ,每次一次 
		calc[u] = 1;
		memset(comp[u], 0, sizeof(comp[u]));
		for (p = 0; p < n; p++) if (u == A[p]) comp[u][p / M] ^= ((ll) 1 << (p % M));			
	}		
}

void solve() {
	prepare();
	memset(row, 0, sizeof(row));
	int prev, curt;
	int i, u, p, c, cc;
	int Ln = (n / M) + 1;
	prev = 0;
	for (i = 0; i < m; i++) {
		curt = 1 - prev; u = B[i];
		for (p = 0; p < Ln; p++) X[p] = row[prev][p] | comp[u][p];
		c = 0;
		for (p = 0; p < Ln; p++) {
			cc = (row[prev][p] & underTop) > 0;
			row[prev][p] = ((row[prev][p] & (underTop - 1)) << 1) + c;
			c = cc;
		}
		for (p = 0; p < Ln; p++) {
			if (row[prev][p] != Topless) {
				row[prev][p]++;
				break;
			}
			row[prev][p] = 0;
		}
		c = 0;
		for (p = 0; p < Ln; p++) {
			if (X[p] >= row[prev][p] + c) 
				row[prev][p] = X[p] - (row[prev][p] + c), c = 0;
			else
				row[prev][p] = Top + X[p] - (row[prev][p] + c), c = 1;
		}
		for (p = 0; p < Ln; p++) 
			row[curt][p] = X[p] & (row[prev][p] ^ X[p]);
		prev = curt;
	}
	int ret = 0;
	for (i = 0; i < n; i++) 
		if (row[prev][i / M] & ((ll) 1 << (i % M))) ret++;	
//	printf("%d %d %d\n", n, m, ret);
//=========ret 就是最长公共子序列。 
	printf("%d %d\n", n - ret, m - ret);
}

int main(){
	int tests=0,T;
	scanf("%d",&T);
	while(T--){
		scanf("%d%d",&n,&m);
		for(int i=0;i<n;i++)
			for (int j = 0; j < m; j++) 
				scanf(" %c",&G[i][j]);
		get(A),get(B);
		
		printf("Case %d: ", ++tests);	
//		printf("A = %s\n, B = %s\n", A, B);
		n = strlen(A), m = strlen(B);
		//n = 20000; m = 20000;
		//for (int i = 0; i < m; i++) A[i] = B[i] = 'A'; 
		//A[m] = B[m] = 0;
		solve();		
	}
}
	\end{lstlisting}

	\section{环状最长公共子序列}
		\begin{lstlisting}
const int N = 2222;

int a[N], b[N];
int n, dp[N][N], from[N][N];

int run() {
	scanf("%d", &n);
	for(int i = 1; i <= n; i++) {
		scanf("%d", &a[i]);
		a[i + n] = a[i];
		b[n - i + 1] = a[i];
	}
	memset(from, 0, sizeof(from));
	int ans = 0;
	for(int i = 1; i <= 2 * n; i++) {
		from[i][0] = 2;
		int upleft = 0, up = 0, left = 0;
		for(int j = 1; j <= n; j++) {
			upleft = up;
			if (a[i] == b[j]) {
				upleft++;
			} else {
				upleft = INT_MIN;
			}
			if (from[i - 1][j])
				up++;
			int mm = max(up, max(left, upleft));
			if (mm == left) {
				from[i][j] = 0;
			} else if (mm == upleft)
				from[i][j] = 1;
			else
				from[i][j] = 2;
			left = mm;
		}
		if (i >= n) {
			int count = 0;
			for(int x = i, y = n; y; ) {
				if (from[x][y] == 1) {
					x--; y--;
					count++;
				} else if (from[x][y] == 0)
					y--;
				else
					x--;
			}
			ans = max(ans, count);
			int x = i - n + 1;
			from[x][0] = 0;
			int y = 0;
			for(; y <= n && from[x][y] == 0; y++);
			for(; x <= i; x++) {
				from[x][y] = 0;
				if (x == i) {
					break;
				}
				for(; y <= n; ++y) {
					if (from[x + 1][y] == 2) {
						break;
					}
					if (y + 1 <= n && from[x + 1][y + 1] == 1) {
						y++;
						break;
					}
				}
			}
		}
	}
	if (n)
		printf("%d\n", ans);
	return n;
}

int main() {
	for(; run(); );
	return 0;
} 
	\end{lstlisting}

	\section{长方体表面两点最近距离}
		\begin{lstlisting}
int r;
void turn(int i, int j, int x, int y, int z,int x0, int y0, int L, int W, int H) {
	if (z==0) {
		int R = x*x+y*y;
		if (R<r) r=R;
	}
	else{
		if(i>=0 && i< 2)
			turn(i+1, j, x0+L+z, y, x0+L-x, x0+L, y0, H, W, L);
		if(j>=0 && j< 2)
			turn(i, j+1, x, y0+W+z, y0+W-y, x0, y0+W, L, H, W);
		if(i<=0 && i>-2)
			turn(i-1, j, x0-z, y, x-x0, x0-H, y0, H, W, L);
		if(j<=0 && j>-2)
			turn(i, j-1, x, y0-z, y-y0, x0, y0-H, L, H, W);
	}
}
int main(){
	int L, H, W, x1, y1, z1, x2, y2, z2;
	cin >> L >> W >> H >> x1 >> y1 >> z1 >> x2 >> y2 >> z2;
	if (z1!=0 && z1!=H)
	if (y1==0 || y1==W)
		swap(y1,z1), std::swap(y2,z2), std::swap(W,H);
	else
		swap(x1,z1), std::swap(x2,z2), std::swap(L,H);
	if (z1==H) z1=0, z2=H-z2;
	r=0x3fffffff; turn(0,0,x2-x1,y2-y1,z2,-x1,-y1,L,W,H);
	cout<<r<<endl;
	return 0;
}
	\end{lstlisting}

	\section{插头DP}
		\begin{lstlisting}
#include <cstdio>
#include <cstdlib>
#include <algorithm>
#include <vector>
#include <iostream>
using namespace std;

typedef long long int64;
typedef pair<int, long long> State;
const int MAXN = 8;

char map[MAXN + 10][MAXN + 10];
int n, m, lastx, lasty;
int64 ans;
vector<State> vec[2];


void mergy(int cur) {
	sort(vec[cur].begin(), vec[cur].end());
	int size = 0;
	for(int i = 0, j = 0; i < vec[cur].size(); i = j) {
		vec[cur][size] = vec[cur][i];
		j = i + 1;
		while(j < vec[cur].size() && vec[cur][j].first == vec[cur][size].first) 
			vec[cur][size].second += vec[cur][j].second, j++;
		size++;
	}
	vec[cur].resize(size);
}

void next_line(int cur) {
	int size = 0;
	for(int i = 0; i < vec[cur].size(); i++) {
		int sta = vec[cur][i].first;
		if ((sta >> (m << 1)) == 0) {
			vec[cur][size] = vec[cur][i];
			vec[cur][size].first <<= 2;
			size++;
		}
	}
	vec[cur].resize(size);
}

inline int replace(int sta, int pos, int v) {
	return (sta & (~(3 << (pos << 1)))) | (v << (pos << 1));
}

inline int replace(int &sta, int pos, int v1, int v2) {
	int res = replace(sta, pos, v1);
	res = replace(res, pos + 1, v2);
	return res;
}

int Trans(int sta, int pos) {
	int cnt = 1, v = (sta >> (pos << 1) & 3);
	if (v == 1) {
		sta = replace(sta, pos, 0, 0);
		for(int i = pos + 2; ; i++) {
			if ((sta >> (i << 1) & 3) == 1)
				cnt++;
			else if ((sta >> (i << 1) & 3) == 2)
				cnt--;
			if (cnt == 0)
				return replace(sta, i, 1);
		}
	} else {
		sta = replace(sta, pos, 0, 0);
		for(int i = pos - 1; ; i--) {
			if ((sta >> (i << 1) & 3) == 1) 
				cnt--;
			else if ((sta >> (i << 1) & 3) == 2)
				cnt++;
			if (cnt == 0)
				return replace(sta, i, 2);
		}
	}
}

void dp_block(int i, int j, int cur) {
	for(int s = 0; s < vec[cur].size(); s++) {
		int sta = vec[cur][s].first;
		int64 val = vec[cur][s].second;
		int left = (sta >> (j << 1)) & 3, up = (sta >> ((j + 1) << 1)) & 3;
		if (left == 0 && up == 0) {
			vec[cur ^ 1].push_back(State(sta, val));
		}
	}
}

void dp_blank(int i, int j, int cur) {
	for(int s = 0; s < vec[cur].size(); s++) {
		int sta = vec[cur][s].first;
		int64 val = vec[cur][s].second;
		int left = (sta >> (j << 1)) & 3, up = (sta >> ((j + 1) << 1)) & 3, ns = 0;
		if (left && up) {
			if (left == 2 && up == 1) {
				vec[cur ^ 1].push_back(State(replace(sta, j, 0, 0), val));
			} else if (left == 1 && up == 2) {
				if (replace(sta, j, 0, 0) == 0 && i == lastx && j == lasty)
					ans += val;
			} else if (left == 1 && up == 1) {
				vec[cur ^ 1].push_back(State(Trans(sta, j), val));
			} else if (left == 2 && up == 2) {
				vec[cur ^ 1].push_back(State(Trans(sta, j), val));
			}
		} else if (left || up) {
			vec[cur ^ 1].push_back(State(sta, val));
			vec[cur ^ 1].push_back(State(replace(sta, j, up, left), val));
		} else {
			vec[cur ^ 1].push_back(State(replace(sta, j, 1, 2), val));
		}
	}
}

void show(int cur) {
	for(int i = 0; i < vec[cur].size(); i++)
		printf("%d %I64d\n", vec[cur][i].first, vec[cur][i].second);
	printf("step\n");
}

int main() {
	freopen("input.txt", "r", stdin);
	while(scanf("%d %d", &n, &m) == 2) {
		ans = 0;
		lastx = lasty = -1;
		gets(map[0]);
		for(int i = 0; i < n; i++) {
			scanf("%s", map[i]);
			for(int j = 0; j < m; j++) {
				if (map[i][j] == '.') {
					lastx = i, lasty = j;
				}
			}
		}
		if (lastx == -1) {
			printf("0\n");
			continue;
		}
		int cur = 0;
		vec[cur].clear();
		vec[cur].push_back(State(0, 1));
		for(int i = 0; i < n; i++) {
			for(int j = 0; j < m; j++) {
				vec[cur ^ 1].clear();
				if (map[i][j] == '.') 
					dp_blank(i, j, cur);
				else
					dp_block(i, j, cur);
				cur ^= 1;
				mergy(cur);
				//show(cur);
			}
			next_line(cur);
		}
		cout << ans << endl;
	}
	return 0;
}
	\end{lstlisting}

	\section{最大团搜索}
		Int g[][]为图的邻接矩阵。
	MC(V)表示点集V的最大团
	令Si={vi, vi+1, ..., vn}, mc[i]表示MC(Si)
	倒着算mc[i],那么显然MC(V)=mc[1]
	此外有mc[i]=mc[i+1] or mc[i]=mc[i+1]+1
	\begin{lstlisting}
void init(){
	int i, j;
	for (i=1; i<=n; ++i) for (j=1; j<=n; ++j) scanf("%d", &g[i][j]);
}
void dfs(int size){
	int i, j, k;
	if (len[size]==0) {
		if (size>ans) {
			ans=size; found=true;
		}
		return;
	}
	for (k=0; k<len[size] && !found; ++k) {
		if (size+len[size]-k<=ans) break;
		i=list[size][k];
		if (size+mc[i]<=ans) break;
		for (j=k+1, len[size+1]=0; j<len[size]; ++j)
		if (g[i][list[size][j]]) list[size+1][len[size+1]++]=list[size][j];
		dfs(size+1);
	}
}
void work(){
	int i, j;
	mc[n]=ans=1;
	for (i=n-1; i; --i) {
		found=false;
		len[1]=0;
		for (j=i+1; j<=n; ++j) if (g[i][j]) list[1][len[1]++]=j;
		dfs(1);
		mc[i]=ans;
	}
}
void print(){
	printf("%d\n", ans);
}
	\end{lstlisting}

	\section{Dancing Links}
		\begin{lstlisting}
namespace dancing_links {
	int const N = , M = , G = ;
 
	struct node {
		int col, row, left, right, up, down;
		inline void clear() {
			col = row = left = right = up = down = 0;
		}
	} grid[G];
 
	int n, m, tot;
	int cnt[M], head[N], tail[N];
 
	inline void prepare() {
		tot = m + 1;
		for (int i = 1; i <= n; ++i) {
			head[i] = tail[i] = 0;
		}
		for (int i = 1; i <= m + 1; ++i) {
			grid[i].col = i;
			grid[i].left = i - 1;
			grid[i].right = i + 1;
			grid[i].up = i;
			grid[i].down = i;
			cnt[i] = 0;
		}
		grid[1].left = m + 1;
		grid[m + 1].right = 1;
	}
 
	inline void remove(int x) {
		grid[grid[x].right].left = grid[x].left;
		grid[grid[x].left].right = grid[x].right;
		for (int y = grid[x].down; y != x; y = grid[y].down) {
			for (int z = grid[y].right; z != y; z = grid[z].right) {
				cnt[grid[z].col]--;
				grid[grid[z].down].up = grid[z].up;
				grid[grid[z].up].down = grid[z].down;
			}
		}
	}
 
	inline void resume(int x) {
		for (int y = grid[x].up; y != x; y = grid[y].up) {
			for (int z = grid[y].left; z != y; z = grid[z].left) {
				cnt[grid[z].col]++;
				grid[grid[z].up].down = z;
				grid[grid[z].down].up = z;
			}
		}
		grid[grid[x].right].left = x;
		grid[grid[x].left].right = x;
	}
 
	inline void add(int x, int y) {
		tot++;
		cnt[y]++;
		if (!head[x]) {
			head[x] = tot;
		}
		if (!tail[x]) {
			tail[x] = tot;
		}
		grid[tot].row = x; grid[tot].col = y;
		grid[tot].up = grid[y].up; grid[grid[y].up].down = tot;
		grid[tot].down = y; grid[y].up = tot;
		grid[tot].left = tail[x]; grid[tail[x]].right = tot;
		grid[tot].right = head[x]; grid[head[x]].left = tot;
		tail[x] = tot;
	}
 
	inline bool dfs(int dep) {
		if (grid[m + 1].right == m + 1) {
			return true;
		}
		int x = grid[m + 1].right;
		for (int i = x; i != m + 1; i = grid[i].right) {
			if (cnt[i] < cnt[x]) {
				x = i;
			}
		}
		if (!cnt[x]) {
			return false;
		}
		remove(x);
		for (int i = grid[x].down; i != x; i = grid[i].down) {
			for (int j = grid[i].right; j != i; j = grid[j].right) {
				remove(grid[j].col);
			}
			if (dfs(dep + 1)) {
				return true;
			}
			for (int j = grid[i].left; j != i; j = grid[j].left) {
				resume(grid[j].col);
			}
		}
		resume(x);
		return false;
	}
 
	inline void clear() {
		for (int i = 1; i <= tot; ++i) {
			grid[i].clear();
		}
	}
}
\end{lstlisting}

	\section{极大团计数}
		Bool g[][]为图的邻接矩阵,图点的标号由1至n。
	\begin{lstlisting}
void dfs(int size){
	int i, j, k, t, cnt, best = 0;
	bool bb;
	if (ne[size]==ce[size]){
		if (ce[size]==0) ++ans;
		return;
	}
	for (t=0, i=1; i<=ne[size]; ++i) {
		for (cnt=0, j=ne[size]+1; j<=ce[size]; ++j)
		if (!g[list[size][i]][list[size][j]]) ++cnt;
		if (t==0 || cnt<best) t=i, best=cnt;
	}
	if (t && best<=0) return;
	for (k=ne[size]+1; k<=ce[size]; ++k) {
		if (t>0){
			for (i=k; i<=ce[size]; ++i) if (!g[list[size][t]][list[size][i]]) break;
			swap(list[size][k], list[size][i]);
		}
		i=list[size][k];
		ne[size+1]=ce[size+1]=0;
		for (j=1; j<k; ++j)if (g[i][list[size][j]]) list[size+1][++ne[size+1]]=list[size][j];
		for (ce[size+1]=ne[size+1], j=k+1; j<=ce[size]; ++j)
		if (g[i][list[size][j]]) list[size+1][++ce[size+1]]=list[size][j];
		dfs(size+1);
		++ne[size];
		--best;
		for (j=k+1, cnt=0; j<=ce[size]; ++j) if (!g[i][list[size][j]]) ++cnt;
		if (t==0 || cnt<best) t=k, best=cnt;
		if (t && best<=0) break;
	}
}
void work(){
	int i;
	ne[0]=0; ce[0]=0;
	for (i=1; i<=n; ++i) list[0][++ce[0]]=i;
	ans=0;
	dfs(0);
}
	\end{lstlisting}

\chapter{Hints}
	\section{积分表}
		\[\arcsin x \to \frac{1}{\sqrt{1-x^2}}				   \]
\[\arccos x \to -\frac{1}{\sqrt{1-x^2}}				  \]
\[\arctan x \to \frac{1}{1+x^2}						  \]
\[a^x \to \frac{a^x}{\ln a}							  \]
\[\sin x \to -\cos x									 \]
\[\cos x \to \sin x									  \]
\[\tan x \to -\ln\cos x								  \]
\[\sec x \to \ln\tan(\frac{x}{2}+\frac{\pi}{4})		  \]
\[\tan^2 x \to \tan x - x								\]
\[\csc x \to \ln\tan\frac{x}{2}						  \]
\[\sin^2 x \to \frac{x}{2} - \frac{1}{2}\sin x\cos x	 \]
\[\cos^2 x \to \frac{x}{2} + \frac{1}{2}\sin x\cos x	 \]
\[\sec^2 x \to \tan x									\]
\[\frac{1}{\sqrt{a^2-x^2}} \to \arcsin\frac{x}{a}		\]
\[\csc^2 x \to -\cot x								   \]
\[\frac{1}{a^2-x^2}(|x|<|a|) \to \frac{1}{2a}\ln\frac{a+x}{a-x}  \]
\[\frac{1}{x^2-a^2}(|x|>|a|) \to \frac{1}{2a}\ln\frac{x-a}{x+a}  \]
\[\sqrt{a^2-x^2} \to \frac{x}{2}\sqrt{a^2-x^2}+\frac{a^2}{2}\arcsin\frac{x}{a}   \]
\[\frac{1}{\sqrt{x^2+a^2}} \to \ln(x+\sqrt{a^2+x^2}) \]
\[\sqrt{a^2+x^2} \to \frac{x}{2}\sqrt{a^2+x^2}+\frac{a^2}{2}\ln(x+\sqrt{a^2+x^2})\]
\[\frac{1}{\sqrt{x^2-a^2}} \to \ln(x+\sqrt{x^2-a^2})\]
\[\sqrt{x^2-a^2} \to \frac{x}{2}\sqrt{x^2-a^2}-\frac{a^2}{2}\ln(x+\sqrt{x^2-a^2})\]
\[\frac{1}{x\sqrt{a^2-x^2}} \to -\frac{1}{a}\ln\frac{a+\sqrt{a^2-x^2}}{x}\]
\[\frac{1}{x\sqrt{x^2-a^2}} \to \frac{1}{a}\arccos\frac{a}{x}\]
\[\frac{1}{x\sqrt{a^2+x^2}} \to -\frac{1}{a}\ln\frac{a+\sqrt{a^2+x^2}}{x}\]
\[\frac{1}{\sqrt{2ax-x^2}} \to \arccos(1-\frac{x}{a})\]
\[\frac{x}{ax+b} \to \frac{x}{a}-\frac{b}{a^2}\ln(ax+b)\]
\[\sqrt{2ax-x^2} \to \frac{x-a}{2}\sqrt{2ax-x^2}+\frac{a^2}{2}\arcsin(\frac{x}{a}-1)\]
\[\frac{1}{x\sqrt{ax+b}}(b<0) \to \frac{2}{\sqrt{-b}}\arctan\sqrt{\frac{ax+b}{-b}}\]
\[x\sqrt{ax+b} \to \frac{2(3ax-2b)}{15a^2}(ax+b)^{\frac{3}{2}}\]
\[\frac{1}{x\sqrt{ax+b}}(b>0) \to \frac{1}{\sqrt{b}}\ln\frac{\sqrt{ax+b}-\sqrt{b}}{\sqrt{ax+b}+\sqrt{b}}\]
\[\frac{x}{\sqrt{ax+b}} \to \frac{2(ax-2b)}{3a^2}\sqrt{ax+b}\]
\[\frac{1}{x^2 \sqrt{ax+b}} \to -\frac{\sqrt{ax+b}}{bx}-\frac{a}{2b}\int\frac{\ud x}{x\sqrt{ax+b}}\]
\[\frac{\sqrt{ax+b}}{x} \to 2\sqrt{ax+b}+b\int\frac{\ud x}{x\sqrt{ax+b}}\]
\[\frac{1}{\sqrt{(ax+b)^n}}(n>2) \to \frac{-2}{a(n-2)}\cdot\frac{1}{\sqrt{(ax+b)^{n-2} }}\]
\[\frac{1}{ax^2+c}(a>0,c>0) \to \frac{1}{\sqrt{ac}}\arctan{(x\sqrt{\frac{a}{c}})}\]
\[\frac{x}{ax^2+c} \to \frac{1}{2a}\ln(ax^2+c)\]
\[\frac{1}{ax^2+c}(a+,c-) \to \frac{1}{2\sqrt{-ac}}\ln\frac{x\sqrt{a}-\sqrt{-c}}{x\sqrt{a}+\sqrt{-c}}\]
\[\frac{1}{x(ax^2+c)} \to \frac{1}{2c}\ln\frac{x^2}{ax^2+c}\]
\[\frac{1}{ax^2+c}(a-,c+) \to \frac{1}{2\sqrt{-ac}}\ln\frac{\sqrt{c}+x\sqrt{-a}}{\sqrt{c}-x\sqrt{-a}}\]
\[x{\sqrt{ax^2+c}} \to \frac{1}{3a}\sqrt{(ax^2+c)^3}\]
\[\frac{1}{(ax^2+c)^n}(n>1) \to \frac{x}{2c(n-1)(ax^2+c)^{n-1}}+\frac{2n-3}{2c(n-1)}\int\frac{\ud x}{(ax^2+c)^{n-1}}\]
\[\frac{x^n}{ax^2+c}(n\ne 1)\to \frac{x^{n-1}}{a(n-1)}-\frac{c}{a}\int\frac{x^{n-2}}{ax^2+c}\ud x\]
\[\frac{1}{x^2(ax^2+c)} \to \frac{-1}{cx}-\frac{a}{c}\int\frac{\ud x}{ax^2+c}\]
\[\frac{1}{x^2(ax^2+c)^n}(n\ge 2) \to \frac{1}{c}\int\frac{\ud x}{x^2(ax^2+c)^{n-1}}-\frac{a}{c}\int\frac{\ud x}{(ax^2+c)^n}\]
\[\sqrt{ax^2+c}(a>0) \to \frac{x}{2}\sqrt{ax^2+c}+\frac{c}{2\sqrt{a}}\ln(x\sqrt{a}+\sqrt{ax^2+c})\]
\[\sqrt{ax^2+c}(a<0) \to \frac{x}{2}\sqrt{ax^2+c}+\frac{c}{2\sqrt{-a}}\arcsin(x\sqrt{\frac{-a}{c}})\]
\[\frac{1}{\sqrt{ax^2+c}}(a>0) \to \frac{1}{\sqrt{a}}\ln(x\sqrt{a}+\sqrt{ax^2+c})\]
\[\frac{1}{\sqrt{ax^2+c}}(a<0) \to \frac{1}{\sqrt{-a}}\arcsin(x\sqrt{-\frac{a}{c}})\]
\[\sin^2 ax \to \frac{x}{2}-\frac{1}{4a}\sin 2ax\]
\[\cos^2 ax \to \frac{x}{2}+\frac{1}{4a}\sin 2ax\]
\[\frac{1}{\sin ax} \to \frac{1}{a}\ln\tan\frac{ax}{2}\]
\[\frac{1}{\cos^2 ax} \to \frac{1}{a}\tan ax\]
\[\frac{1}{\cos ax} \to \frac{1}{a}\ln \tan(\frac{\pi}{4}+\frac{ax}{2})\]
\[\ln(ax)\to x\ln(ax)-x\]
\[\sin^3 ax \to \frac{-1}{a}\cos ax+\frac{1}{3a}\cos^3 ax\]
\[\cos^3 ax \to \frac{1}{a}\sin ax - \frac{1}{3a}\sin^3 ax\]
\[\frac{1}{\sin^2 ax}\to -\frac{1}{a}\cot ax\]
\[x\ln(ax)\to \frac{x^2}{2}\ln(ax)-\frac{x^2}{4}\]
\[\cos ax\to \frac{1}{a}\sin ax\]
\[x^2 e^{ax} \to \frac{e^{ax}}{a^3}(a^2x^2-2ax+2)\]
\[(\ln(ax))^2 \to x(\ln(ax))^2-2x\ln(ax)+2x\]
\[x^2\ln(ax) \to \frac{x^3}{3}\ln(ax)-\frac{x^3}{9}\]
\[x^n\ln(ax) \to \frac{x^{n+1}}{n+1}\ln(ax)-\frac{x^{n+1}}{(n+1)^2}\]
\[\sin(\ln ax) \to \frac{x}{2}[\sin(\ln ax) - \cos(\ln ax)]\]
\[\cos(\ln ax) \to \frac{x}{2}[\sin(\ln ax) + \cos(\ln ax)]\]

	\section{数学公式}
		\paragraph{组合公式}
\begin{itemize}
\item
fibonacci\\
$f_0=0, f_1=1$\\
$f_{n+2}f_n-f_{n+1}^2=(-1)^{n+1}$\\
$f_{-n}=(-1)^{n-1}f_n$\\
$f_{n+k}=f_kf_{n+1}+f_{k-1}f_n$\\
$gcd(f_m, f_n)=f_{gcd(m, n)}$\\
$f_m|f_n^2\Leftrightarrow nf_n|m$\\

\item
  $\sum_{k=1}^{n}(2k-1)^2 = \frac{n(4n^2-1)}{3}	$
\item
  $\sum_{k=1}^{n}k^3 = (\frac{n(n+1)}{2})^2	$
\item
  $\sum_{k=1}^{n}(2k-1)^3 = n^2(2n^2-1)	$
\item
  $\sum_{k=1}^{n}k^4 = \frac{n(n+1)(2n+1)(3n^2+3n-1)}{30}  $
\item
  $\sum_{k=1}^{n}k^5 = \frac{n^2(n+1)^2(2n^2+2n-1)}{12}	$
\item
  $\sum_{k=1}^{n}k(k+1) = \frac{n(n+1)(n+2)}{3}	$
\item
  $\sum_{k=1}^{n}k(k+1)(k+2) = \frac{n(n+1)(n+2)(n+3)}{4} $
\item
  $\sum_{k=1}^{n}k(k+1)(k+2)(k+3) = \frac{n(n+1)(n+2)(n+3)(n+4)}{5} $
\item
  $\mbox{错排:}D_n = n!(1-\frac{1}{1!}+\frac{1}{2!}-\frac{1}{3!}+\ldots+\frac{(-1)^n}{n!}) = (n-1)(D_{n-2}-D_{n-1})$
\end{itemize}


	\section{平面几何公式}
		\subsection*{三角形}
\begin{enumerate}
	\item 半周长
			$P=(a+b+c)/2$
	\item 面积
			$S=aH_a/2=ab\sin(C)/2=\sqrt{P(P-a)(P-b)(P-c)}$
	\item 中线
		$M_a=\sqrt{2(b^2+c^2)-a^2)}/2=\sqrt{b^2+c^2+2bc\cos(A)}/2$
	\item 角平分线 
		$T_a=\sqrt{bc((b+c)^2-a^2)}/(b+c)=2bc\cos(A/2)/(b+c)$
	\item 高线
		$H_a=b\sin(C)=c\sin(B)=\sqrt{b^2-((a^2+b^2-c^2)/(2a))^2}$
	\item 内切圆半径
		\begin{align*}
			r&=S/P=\arcsin(B/2)\sin(C/2)/\sin((B+C)/2)=4R\sin(A/2)\sin(B/2)\sin(C/2)\\
			&=\sqrt{(P-a)(P-b)(P-c)/P}=P\tan(A/2)\tan(B/2)\tan(C/2)
		\end{align*}
	\item 外接圆半径
		$R=abc/(4S)=a/(2\sin(A))=b/(2\sin(B))=c/(2\sin(C))$
\end{enumerate}

\subsection*{四边形}
	$D1,D2$为对角线,$M$对角线中点连线,$A$为对角线夹角
	\begin{enumerate}
		\item $a^2+b^2+c^2+d^2=D1^2+D2^2+4M^2$
		\item $S=D1D2\sin(A)/2$
		\item 圆内接四边形 $ac+bd=D1D2$
		\item 圆内接四边形,$P$为半周长 $S=\sqrt{(P-a)(P-b)(P-c)(P-d)}$
	\end{enumerate}

\subsection*{正n边形}
	$R$为外接圆半径,$r$为内切圆半径
	\begin{enumerate}
		\item 中心角 $A=2\pi/n$
		\item 内角 $C=(n-2)\pi/n$
		\item 边长 $a=2\sqrt{R^2-r^2}=2R\sin(A/2)=2r\tan(A/2)$
		\item 面积 $S=nar/2=nr^2\tan(A/2)=nR^2\sin(A)/2=na^2/(4\tan(A/2))$
	\end{enumerate}

\subsection*{圆}
	\begin{enumerate}
		\item 弧长 $l=rA$
		\item 弦长 $a=2\sqrt{2hr-h^2}=2r\sin(A/2)$
		\item 弓形高 $h=r-\sqrt{r^2-a^2/4}=r(1-\cos(A/2))=\arctan(A/4)/2$
		\item 扇形面积 $S1=rl/2=r^2A/2$
		\item 弓形面积 $S2=(rl-a(r-h))/2=r^2(A-\sin(A))/2$
	\end{enumerate}
	
\subsection*{棱柱}
	\begin{enumerate}
		\item 体积$V=Ah$,$A$为底面积,$h$为高
		\item 侧面积 $S=lp$,$l$为棱长,$p$为直截面周长
		\item 全面积 $T=S+2A$
	\end{enumerate}

\subsection*{棱锥}
	\begin{enumerate}
		\item 体积$V=Ah$,$A$为底面积,$h$为高
		\item 正棱锥侧面积 $S=lp$,$l$为棱长,$p$为直截面周长
		\item 正棱锥全面积 $T=S+2A$
	\end{enumerate}

\subsection*{棱台}
	\begin{enumerate}
		\item 体积 $V=(A1+A2+\sqrt{A1A2})h/3$, $A1,A2$为上下底面积,$h$为高
		\item 正棱台侧面积 $S=(p1+p2)l/2$,$p1,p2$为上下底面周长,$l$为斜高
		\item 正棱台全面积 $T=S+A1+A2$
	\end{enumerate}

\subsection*{圆柱}
	\begin{enumerate}
		\item 侧面积 $S=2\pi rh$
		\item 全面积 $T=2\pi r(h+r)$
		\item 体积 $V=\pi r^2h$
	\end{enumerate}

\subsection*{圆锥}
	\begin{enumerate}
		\item 母线 $l=\sqrt{h^2+r^2}$
		\item 侧面积 $S=\pi rl$
		\item 全面积 $T=\pi r(l+r)$
		\item 体积 $V=\pi r^2h/3$
	\end{enumerate}

\subsection*{圆台}
	\begin{enumerate}
		\item 母线 $l=\sqrt{h^2+(r1-r2)^2}$
		\item 侧面积 $S=\pi(r1+r2)l$
		\item 全面积 $T=\pi r1(l+r1)+\pi r2(l+r2)$
		\item 体积 $V=\pi(r1^2+r2^2+r1r2)h/3$
	\end{enumerate}

\subsection*{球}
	\begin{enumerate}
		\item 全面积 $T=4\pi r^2$
		\item 体积 $V=4\pi r^3/3$
	\end{enumerate}

\subsection*{球台}
	\begin{enumerate}
		\item 侧面积 $S=2\pi rh$
		\item 全面积 $T=\pi(2rh+r1^2+r2^2)$
		\item 体积 $V=\pi h(3(r1^2+r2^2)+h^2)/6$
	\end{enumerate}

\subsection*{球扇形}
	\begin{enumerate}
		\item 全面积 $T=\pi r(2h+r0)$,$h$为球冠高,$r0$为球冠底面半径
		\item 体积 $V=2\pi r^2h/3$
	\end{enumerate}


	\section{网络流Hints}
	下界:(u, v)下界为c:超级源到t建流量为c, s到超级汇建流量为c, (原来的汇到原来的源建无穷, 如果有), 流一遍超级源出边满了就存在可行流.\\
	下界最大流(有源汇): 上面的搞完从原来的源到原来的汇流一遍\\
	下界最小流(有源汇): 上面的搞完从原来的汇到原来的源流一遍\\
	\section{2-SATHints}
	每对点都选择强连通时color较小的\\
	\section{二分图相关Hints}
		二分图最小覆盖集: 从右边的所有没有匹配过的点出发走增广路, 右边所有没有打上记号的点, 加上左边已经有记号的点.\\
最小覆盖数=最大匹配数.

	\section{java\_hints}
		旧\\
		\begin{lstlisting}
import java.io.*;
import java.util.*;
import java.math.*;

class InputReader {
	BufferedReader buff;
	StringTokenizer tokenizer;

	InputReader(InputStream stream) {
		buff = new BufferedReader(new InputStreamReader(stream));
		tokenizer = null;
	}
	boolean hasNext() {
		while (tokenizer == null || !tokenizer.hasMoreTokens())
			try {
				tokenizer = new StringTokenizer(buff.readLine());
			}
			catch (Exception e) {
				return false;
			}
		return true;
	}
	String next() {
		if (!hasNext())
			throw new RuntimeException();
		return tokenizer.nextToken();
	}
	int nextInt() { return Integer.parseInt(next()); }
	long nextLong() { return Long.parseLong(next()); }
}

class Node implements Comparable<Node> {
	int key;
	public int compareTo(Node o) {
		if (key != o.key)
			return key < o.key ? -1 : 1;
		return 0;
	}
	public boolean equals(Object o) { return false; }
	public String toString() { return ""; }
	public int hashCode() { return key; }
}

class MyComparator implements Comparator<Node> {
	public int compare(Node a, Node b) {
		if (a.key != b.key)
			return a.key < b.key ? -1 : 1;
		return 0;
	}
}

public class Main {
	public static void main(String[] args) {
		new Main().run();
	}
	void run() {
		PriorityQueue<Integer> Q = new PriorityQueue<Integer>();
		Q.offer(1); Q.poll(); Q.peek(); Q.size();

		HashMap<Node, Integer> dict = new HashMap<Node, Integer>();
		dict.entrySet(); dict.put(new Node(), 0); dict.containsKey(new Node());
		//Map.Entry e = (Map.Entry)it.next(); e.getValue(); e.getKey();

		HashSet<Node> h = new HashSet<Node>();
		h.contains(new Node()); h.add(new Node()); h.remove(new Node());

		Random rand = new Random();
		rand.nextInt(); rand.nextDouble();

		int temp = 0;
		BigInteger a = BigInteger.ZERO, b = new BigInteger("1"), c =
				 BigInteger.valueOf(2);
		a.remainder(b); a.modPow(b, c); a.pow(temp); a.intValue();
		a.isProbablePrime(temp); // 1 - 1 / 2 ^ certainty
		a.nextProbablePrime();

		Arrays.asList(array);
		Arrays.sort(array, fromIndex, toIndex, comparator);
		Arrays.fill(array, fromIndex, toIndex, value);
		Arrays.binarySearch(array, key, comparator); // found ? index : -
				(insertPoint) - 1
		Arrays.equals(array, array2);
		Collection.toArray(arrayType[]);

		Collections.copy(dest, src);
		Collections.fill(collection, value);
		Collections.max(collection, comparator);
		Collections.replaceAll(list, oldValue, newValue);
		Collections.reverse(list);
		Collections.reverseOrder();
		Collections.rotate(list, distance); //  --------->
		Collections.shuffle(list); // random_shuffle
	}
}
\end{lstlisting}

		新\\
		\begin{lstlisting}
import java.io.*;
import java.util.*;
import java.math.*;

public class Main {
	public static void main(String[] args) {
		InputStream inputStream = System.in;
		OutputStream outputStream = System.out;
		InputReader in = new InputReader(inputStream);
		PrintWriter out = new PrintWriter(outputStream);
		Task solver = new Task();
		solver.solve(1, in, out);
		out.close();
	}
}

class Task {
	public void solve(int testNumber, InputReader in, PrintWriter out) {
		
	}
}

class InputReader {
	public BufferedReader reader;
	public StringTokenizer tokenizer;
	
	public InputReader(InputStream stream) {
		reader = new BufferedReader(new InputStreamReader(stream), 32768);
		tokenizer = null;
	}
	
	public String next() {
		while (tokenizer == null || !tokenizer.hasMoreTokens()) {
			try {
				tokenizer = new StringTokenizer(reader.readLine());
			} catch (IOException e) {
				throw new RuntimeException(e);
			}
		}
		return tokenizer.nextToken();
	}
	
	public int nextInt() {
		return Integer.parseInt(next());
	}
	
	public long nextLong() {
		return Long.parseLong(next());
	}
}
\end{lstlisting}

	\section{Usage\_of\_Rope}
		\begin{lstlisting}
#include <ext/rope>
using __gnu_cxx::crope; using __gnu_cxx::rope;
a = b.substr(from, len);        // [from, from + len)
a = b.substr(from);             // [from, from]
b.c_str();                      // might lead to memory leaks
b.delete_c_str();               // delete the c_str that created before
a.insert(p, str);               // insert str before position p
a.erase(i, n);                  // erase [i, i + n)
\end{lstlisting}

\end{document}
