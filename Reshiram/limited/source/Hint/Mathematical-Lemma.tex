\begin{spacing}{0.1}
\subsection*{求和公式}
	\begin{enumerate}
		\item $\sum_{k=1}^{n}(2k-1)^2 = \frac{n(4n^2-1)}{3}	$
		\item $\sum_{k=1}^{n}k^3 = [\frac{n(n+1)}{2}]^2	$
		\item $\sum_{k=1}^{n}(2k-1)^3 = n^2(2n^2-1)	$
		\item $\sum_{k=1}^{n}k^4 = \frac{n(n+1)(2n+1)(3n^2+3n-1)}{30}  $
		\item $\sum_{k=1}^{n}k^5 = \frac{n^2(n+1)^2(2n^2+2n-1)}{12}	$
		\item $\sum_{k=1}^{n}k(k+1) = \frac{n(n+1)(n+2)}{3}	$
		\item $\sum_{k=1}^{n}k(k+1)(k+2) = \frac{n(n+1)(n+2)(n+3)}{4} $
		\item $\sum_{k=1}^{n}k(k+1)(k+2)(k+3) = \frac{n(n+1)(n+2)(n+3)(n+4)}{5} $
		\item $\displaystyle \frac{1}{(1 - x)^{n + 1}} = \sum_{i = 0}^{n}\binom{i + n}{i}x^i$
		\item $\displaystyle \frac{1}{\sqrt{1 - 4x}} = \sum_{i = 0}^{n}\binom{2i}{i}x^i$
	\end{enumerate}
\subsection*{斐波那契数列}
	\begin{enumerate}
		\item $fib_0=0, fib_1=1, fib_n=fib_{n-1}+fib_{n-2}$
		\item $fib_{n+2} \cdot fib_n-fib_{n+1}^2=(-1)^{n+1}$
		\item $fib_{-n}=(-1)^{n-1}fib_n$
		\item $fib_{n+k}=fib_k \cdot fib_{n+1}+fib_{k-1} \cdot fib_n$
		\item $gcd(fib_m, fib_n)=fib_{gcd(m, n)}$
		\item $fib_m|fib_n^2\Leftrightarrow nfib_n|m$
	\end{enumerate}
\subsection*{错排公式}
	\[D_n = (n-1)(D_{n-2}-D_{n-1})= = n! \cdot (1-\frac{1}{1!}+\frac{1}{2!}-\frac{1}{3!}+\ldots+\frac{(-1)^n}{n!})\]
\subsection*{莫比乌斯函数}
	$$\mu(n) = \begin{cases}
		1 & \text{若}n=1\\
		(-1)^k & \text{若}n\text{无平方数因子,且}n = p_1p_2\dots p_k\\
		0 & \text{若}n\text{有大于}1\text{的平方数因数}
	\end{cases}$$
	$$\sum_{d|n}{\mu(d)} = \begin{cases}
		1 & \text{若}n=1\\
		0 & \text{其他情况}
	\end{cases}$$
	$$g(n) = \sum_{d|n}{f(d)} \Leftrightarrow f(n) = \sum_{d|n}{\mu(d)g(\frac{n}{d})}$$
	$$g(x) = \sum_{n=1}^{[x]}f(\frac{x}{n}) \Leftrightarrow f(x) = \sum_{n=1}^{[x]}{\mu(n)g(\frac{x}{n})}$$
\subsection*{Burnside引理}
	设$G$是一个有限群,作用在集合$X$上。对每个$g$属于$G$,令$X^g$表示$X$中在$g$作用下的不动元素,轨道数(记作$|X/G|$)为$\displaystyle |X/G| = \frac{1}{|G|}\sum_{g \in G}|X^g|.\,$
\subsection*{五边形数定理}
	设$p(n)$是$n$的拆分数,有$\displaystyle p(n) = \sum_{k \in \mathbb{Z} \setminus \{0\}} (-1)^{k - 1} p\left(n - \frac{k(3k - 1)}{2}\right)$
\subsection*{树的计数}
	\begin{enumerate}
		\item 有根树计数:$n+1$个结点的有根树的个数为$\displaystyle a_{n+1} = \frac{\sum_{j=1}^{n}{j \cdot a_j \cdot{S_{n, j}}}}{n}$,其中,$\displaystyle S_{n, j} = \sum_{i=1}^{n/j}{a_{n+1-ij}} = S_{n-j, j} + a_{n+1-j}$
		\item 无根树计数:当$n$为奇数时,$n$个结点的无根树的个数为$a_n-\sum_{i=1}^{n/2}{a_ia_{n-i}}$,当$n$为偶数时,$n$个结点的无根树的个数为$a_n-\sum_{i=1}^{n/2}{a_ia_{n-i}}+\frac{1}{2}a_{\frac{n}{2}}(a_{\frac{n}{2}}+1)$
		\item $n$个结点的完全图的生成树个数为:$\displaystyle n^{n-2}$
		\item 矩阵-树定理:
		图$G$由$n$个结点构成,设$\bm{A}[G]$为图$G$的邻接矩阵、$\bm{D}[G]$为图$G$的度数矩阵,
		则图$G$的不同生成树的个数为$\bm{C}[G] = \bm{D}[G] - \bm{A}[G]$的任意一个$n-1$阶主子式的行列式值。
	\end{enumerate}
\subsection*{欧拉公式}
	平面图的顶点个数、边数和面的个数有如下关系:$\displaystyle V - E + F = C+ 1$\par
	其中,$V$是顶点的数目,$E$是边的数目,$F$是面的数目,$C$是组成图形的连通部分的数目。当图是单连通图的时候,公式简化为:$\displaystyle V - E + F = 2$
\subsection*{皮克定理}
	给定顶点坐标均是整点(或正方形格点)的简单多边形,其面积$A$和内部格点数目$i$、边上格点数目$b$的关系:$A = i + \frac{b}{2} - 1$
\subsection*{牛顿恒等式}
	设$$\prod_{i = 1}^n{(x - x_i)} = a_n + a_{n - 1} x + \dots + a_1 x^{n - 1} + a_0 x^n$$
	$$p_k = \sum_{i = 1}^n{x_i^k}$$
	则$$a_0 p_k + a_1 p_{k - 1} + \cdots + a_{k - 1} p_1 + k a_k = 0$$\par
	特别地,对于$$|\bm{A} - \lambda \bm{E}| = (-1)^n(a_n + a_{n - 1} \lambda + \cdots + a_1 \lambda^{n - 1} + a_0 \lambda^n)$$
	有$$p_k = \mathrm{Tr}(\bm{A}^k)$$
%\section{数论公式}
\section{平面几何公式}
\subsection*{三角形}
\begin{spacing}{0.1}
	\begin{enumerate}
		\item 面积:$\displaystyle S=\frac{a \cdot H_a}{2}=\frac{ab \cdot sinC}{2}=\sqrt{p(p-a)(p-b)(p-c)}\left(\frac{a + b + c}{2}\right)$
		\item 中线:$\displaystyle M_a=\frac{\sqrt{2(b^2+c^2)-a^2}}{2}=\frac{\sqrt{b^2+c^2+2bc \cdot cosA}}{2}$
		\item 角平分线:$\displaystyle T_a=\frac{\sqrt{bc \cdot [(b+c)^2-a^2]}}{b+c}=\frac{2bc}{b+c}cos\frac{A}{2}$
		\item 高线:$\displaystyle H_a=bsinC=csinB=\sqrt{b^2-(\frac{a^2+b^2-c^2}{2a})^2}$
		\item 内切圆半径
			\begin{align*}
				r&=\frac{S}{p}=\frac{arcsin\frac{B}{2} \cdot sin\frac{C}{2}}{sin\frac{B+C}{2}}=4R \cdot sin\frac{A}{2}sin\frac{B}{2}sin\frac{C}{2}\\
				&=\sqrt{\frac{(p-a)(p-b)(p-c)}{p}}=p \cdot tan\frac{A}{2}tan\frac{B}{2}tan\frac{C}{2}
			\end{align*}
		\item 外接圆半径:$\displaystyle R=\frac{abc}{4S}=\frac{a}{2sinA}=\frac{b}{2sinB}=\frac{c}{2sinC}$
	\end{enumerate}
\end{spacing}
\subsection*{四边形}
\begin{spacing}{0.1}
	$D_1, D_2$为对角线,$M$对角线中点连线,$A$为对角线夹角,$p$为半周长
	\begin{enumerate}
		\item $a^2+b^2+c^2+d^2=D_1^2+D_2^2+4M^2$
		\item $S=\frac{1}{2}D_1D_2sinA$
		\item 对于圆内接四边形:$\displaystyle ac+bd=D_1D_2$
		\item 对于圆内接四边形:$\displaystyle S=\sqrt{(p-a)(p-b)(p-c)(p-d)}$
	\end{enumerate}
\end{spacing}
\subsection*{正$n$边形}
\begin{spacing}{0.1}
	$R$为外接圆半径,$r$为内切圆半径
	\begin{enumerate}
		\item 中心角:$\displaystyle A=\frac{2\pi}{n}$
		\item 内角:$\displaystyle C=\frac{n-2}{n}\pi$
		\item 边长:$\displaystyle a=2\sqrt{R^2-r^2}=2R \cdot sin\frac{A}{2}=2r \cdot tan\frac{A}{2}$
		\item 面积:$\displaystyle S=\frac{nar}{2}=nr^2 \cdot tan\frac{A}{2}=\frac{nR^2}{2} \cdot sinA=\frac{na^2}{4 \cdot tan\frac{A}{2}}$
	\end{enumerate}
\end{spacing}
\subsection*{圆}
\begin{spacing}{0.1}
	\begin{enumerate}
		\item 弧长:$\displaystyle l=rA$
		\item 弦长:$\displaystyle a=2\sqrt{2hr-h^2}=2r\cdot sin\frac{A}{2}$
		\item 弓形高:$\displaystyle h=r-\sqrt{r^2-\frac{a^2}{4}}=r(1-cos\frac{A}{2})=\frac{1}{2} \cdot arctan\frac{A}{4}$
		\item 扇形面积:$\displaystyle S_1=\frac{rl}{2}=\frac{r^2A}{2}$
		\item 弓形面积:$\displaystyle S_2=\frac{rl-a(r-h)}{2}=\frac{r^2}{2}(A-sinA)$
	\end{enumerate}
\end{spacing}
\subsection*{棱柱}
\begin{spacing}{0.1}
	\begin{enumerate}
		\item 体积($A$为底面积,$h$为高):$\displaystyle V=Ah$
		\item 侧面积($l$为棱长,$p$为直截面周长):$\displaystyle S=lp$
		\item 全面积:$\displaystyle T=S+2A$
	\end{enumerate}
\end{spacing}
\subsection*{棱锥}
\begin{spacing}{0.1}
	\begin{enumerate}
		\item 体积($A$为底面积,$h$为高):$\displaystyle V=Ah$
		\item 正棱锥侧面积($l$为棱长,$p$为直截面周长):$\displaystyle S=lp$
		\item 正棱锥全面积:$\displaystyle T=S+2A$
	\end{enumerate}
\end{spacing}
\subsection*{棱台}
\begin{spacing}{0.1}
	\begin{enumerate}
		\item 体积($A_1,A_2$为上下底面积,$h$为高):$\displaystyle V=(A_1+A_2+\sqrt{A_1A_2}) \cdot \frac{h}{3}$
		\item 正棱台侧面积($p_1,p_2$为上下底面周长,$l$为斜高):$\displaystyle S=\frac{p_1+p_2}{2}l$
		\item 正棱台全面积:$\displaystyle T=S+A_1+A_2$
	\end{enumerate}
\end{spacing}
\subsection*{圆柱}
\begin{spacing}{0.1}
	\begin{enumerate}
		\item 侧面积:$\displaystyle S=2\pi rh$
		\item 全面积:$\displaystyle T=2\pi r(h+r)$
		\item 体积:$\displaystyle V=\pi r^2h$
	\end{enumerate}
\end{spacing}
\subsection*{圆锥}
\begin{spacing}{0.1}
	\begin{enumerate}
		\item 母线:$\displaystyle l=\sqrt{h^2+r^2}$
		\item 侧面积:$\displaystyle S=\pi rl$
		\item 全面积:$\displaystyle T=\pi r(l+r)$
		\item 体积:$\displaystyle V=\frac{\pi}{3} r^2h$
	\end{enumerate}
\end{spacing}
\subsection*{圆台}
\begin{spacing}{0.1}
	\begin{enumerate}
		\item 母线:$\displaystyle l=\sqrt{h^2+(r_1-r_2)^2}$
		\item 侧面积:$\displaystyle S=\pi(r_1+r_2)l$
		\item 全面积:$\displaystyle T=\pi r_1(l+r_1)+\pi r_2(l+r_2)$
		\item 体积:$\displaystyle V=\frac{\pi}{3}(r_1^2+r_2^2+r_1r_2)h$
	\end{enumerate}
\end{spacing}
\subsection*{球}
\begin{spacing}{0.1}
	\begin{enumerate}
		\item 全面积:$\displaystyle T=4\pi r^2$
		\item 体积:$\displaystyle V=\frac{4}{3}\pi r^3$
	\end{enumerate}
\end{spacing}
\subsection*{球台}
\begin{spacing}{0.1}
	\begin{enumerate}
		\item 侧面积:$\displaystyle S=2\pi rh$
		\item 全面积:$\displaystyle T=\pi(2rh+r_1^2+r_2^2)$
		\item 体积:$\displaystyle V=\frac{\pi h[3(r_1^2+r_2^2)+h^2]}{6}$
	\end{enumerate}
\end{spacing}
\subsection*{球扇形}
\begin{spacing}{0.1}
	\begin{enumerate}
		\item 全面积($h$为球冠高,$r_0$为球冠底面半径):$\displaystyle T=\pi r(2h+r_0)$
		\item 体积:$\displaystyle V=\frac{2}{3}\pi r^2h$
	\end{enumerate}
\end{spacing}
\section{立体几何公式}
\subsection*{球面三角公式}
	设$a, b, c$是边长,$A, B, C$是所对的二面角,
	有余弦定理$$cos a = cos b \cdot cos c + sin b \cdot sin c \cdot cos A$$
	正弦定理$$\frac{sin A}{sin a} = \frac{sin B}{sin b} = \frac{sin C}{sin c}$$
	三角形面积是$A + B + C - \pi$
\subsection*{四面体体积公式}
	$U, V, W, u, v, w$是四面体的$6$条棱,$U, V, W$构成三角形,$(U, u), (V, v), (W, w)$互为对棱,
	则$$V = \frac{\sqrt{(s - 2a)(s - 2b)(s - 2c)(s - 2d)}}{192 uvw}$$
	其中$$\left\{\begin{array}{lll}
			a & = & \sqrt{xYZ}, \\
			b & = & \sqrt{yZX}, \\
			c & = & \sqrt{zXY}, \\
			d & = & \sqrt{xyz}, \\
			s & = & a + b + c + d, \\
			X & = & (w - U + v)(U + v + w), \\
			x & = & (U - v + w)(v - w + U), \\
			Y & = & (u - V + w)(V + w + u), \\
			y & = & (V - w + u)(w - u + V), \\
			Z & = & (v - W + u)(W + u + v), \\
			z & = & (W - u + v)(u - v + W)
		\end{array}\right.$$
\end{spacing}
