%!TEX program = xelatex
\documentclass[landscape, oneside, a4paper, cs4size]{book}

\def\marginset#1#2{                      % 页边设置 \marginset{left}{top}
\setlength{\oddsidemargin}{#1}         % 左边(书内侧)装订预留空白距离
\iffalse                   % 如果考虑左侧(书内侧)的边注区则改为\iftrue
\reversemarginpar
\addtolength{\oddsidemargin}{\marginparsep}
\addtolength{\oddsidemargin}{\marginparwidth}
\fi
\setlength{\evensidemargin}{0mm}       % 置0
\iffalse                   % 如果考虑右侧(书外侧)的边注区则改为\iftrue
\addtolength{\evensidemargin}{\marginparsep}
\addtolength{\evensidemargin}{\marginparwidth}
\fi
% \paperwidth = h + \oddsidemargin+\textwidth+\evensidemargin + h
\setlength{\hoffset}{\paperwidth}
\addtolength{\hoffset}{-\oddsidemargin}
\addtolength{\hoffset}{-\textwidth}
\addtolength{\hoffset}{-\evensidemargin}
\setlength{\hoffset}{0.5\hoffset}
\addtolength{\hoffset}{-1in}           % h = \hoffset + 1in
%\setlength{\voffset}{-1in}             % 0 = \voffset + 1in
\setlength{\topmargin}{\paperheight}
\addtolength{\topmargin}{-\headheight}
\addtolength{\topmargin}{-\headsep}
\addtolength{\topmargin}{-\textheight}
\addtolength{\topmargin}{-\footskip}
\addtolength{\topmargin}{#2}           % 上边预留装订空白距离
\setlength{\topmargin}{0.5\topmargin}
}
% 调整页边空白使内容居中,两参数分别为纸的左边和上边预留装订空白距离
\marginset{125mm}{200mm}


%\usepackage{ctex}
\usepackage{bm}
%\usepackage[fleqn]{amsmath}
\usepackage{harpoon}
\usepackage{fontspec}
\usepackage{listings}
\usepackage[left=1cm,right=1cm,top=1.2cm,bottom=1cm,columnsep=1cm,dvipdfm]{geometry}
\usepackage{setspace}
\usepackage{bm}
\usepackage{cmap}
\usepackage{cite}
\usepackage{float}
\usepackage{xeCJK}
\usepackage{amsthm}
\usepackage{amsmath}
\usepackage{amssymb}
\usepackage{multirow}
\usepackage{multicol}
\usepackage{setspace}
\usepackage{enumerate}
\usepackage{indentfirst}
\usepackage{adjmulticol}
\usepackage{titlesec}
\usepackage{color,minted}
\usepackage{xeCJK}
\allowdisplaybreaks
%\setlength{\parindent}{0em}
%\setlength{\mathindent}{0pt}
\lstset{breaklines}
\let\cleardoublepage\relax
\titleformat{\chapter}{\normalfont\normalsize\sffamily}{\thechapter}{10pt}{}
\titleformat{\section}{\normalfont\footnotesize\sffamily}{\thesection}{1em}{}
\titleformat{\subsection}{\normalfont\footnotesize\sffamily}{\thesubsection}{1em}{}
\titleformat{\subsubsection}{\normalfont\footnotesize\sffamily}{\thesubsubsection}{1em}{}
\titlespacing*{\chapter} {0pt}{5pt}{5pt}
\titlespacing*{\section} {0pt}{0pt}{0pt}
\titlespacing*{\subsection} {0pt}{0pt}{0pt}
\titlespacing*{\subsubsection}{0pt}{0pt}{0pt}
%configure fonts
\setmonofont{LMMono10-Regular}[Scale=0.8]
%\setmonofont{FiraCode-Retina}[Scale=0.8]
\setCJKmainfont{FandolSong-Regular}
\setCJKsansfont{SourceHanSans-Medium}
\setCJKmonofont[Scale=0.8]{STXihei}
\usepackage{yfonts}

\usepackage{fancyhdr}

\renewcommand{\theFancyVerbLine}{\sffamily \textcolor[rgb]{0.5,0.5,0.5}{\scriptsize {\arabic{FancyVerbLine}}}}

\usemintedstyle{tango}

\setminted[cpp]{
	style=xcode,
	mathescape,
	linenos,
	autogobble,
	baselinestretch=0.8,
	tabsize=2,
	fontsize=\normalsize,
	%bgcolor=Gray,
	frame=single,
	framesep=1mm,
	framerule=0.3pt,
	numbersep=1mm,
	breaklines=true,
	breaksymbolsepleft=2pt,
	%breaksymbolleft=\raisebox{0.8ex}{ \small\reflectbox{\carriagereturn}}, %not moe!
	%breaksymbolright=\small\carriagereturn,
	breakbytoken=false,
}
\setminted[java]{
	style=xcode,
	mathescape,
	linenos,
	autogobble,
	baselinestretch=0.8,
	tabsize=2,
	fontsize=\normalsize,
	%bgcolor=Gray,
	frame=single,
	framesep=1mm,
	framerule=0.3pt,
	numbersep=1mm,
	breaklines=true,
	breaksymbolsepleft=2pt,
	%breaksymbolleft=\raisebox{0.8ex}{ \small\reflectbox{\carriagereturn}}, %not moe!
	%breaksymbolright=\small\carriagereturn,
	breakbytoken=false,
}
\setminted[text]{
	style=xcode,
	mathescape,
	linenos,
	autogobble,
	baselinestretch=0.8,
	tabsize=4,
	fontsize=\normalsize,
	%bgcolor=Gray,
	frame=single,
	framesep=1mm,
	framerule=0.3pt,
	numbersep=1mm,
	breaklines=true,
	breaksymbolsepleft=2pt,
	%breaksymbolleft=\raisebox{0.8ex}{ \small\reflectbox{\carriagereturn}}, %not moe!
	%breaksymbolright=\small\carriagereturn,
	breakbytoken=false,
}

\usepackage{lastpage}
\pagestyle{fancy}
\fancypagestyle{plain}{}
\fancyhf{}
\lhead{Shanghai Jiao Tong University × Arondight}
\chead{\leftmark}
\rhead{\thepage/\pageref{LastPage}}
\setlength{\headsep}{1pt}

\usepackage{tocloft}
\makeatletter
\renewcommand{\@cftmaketoctitle}{}
\makeatother

\usepackage{punk}

\begin{document}\scriptsize
	\renewcommand{\thefootnote}{\fnsymbol{footnote}}
	\title{\Huge{{\punkfamily Arondight's Standard Code Library}}\thanks{https://www.github.com/footoredo/Arondight}}
	\author{\emph{Shanghai Jiao Tong University}}
	\date{Dated: \today}
	\maketitle
	\clearpage
	\begin{multicols}{2}
		\tableofcontents
		\clearpage
		\begin{spacing}{0.8}
			\def \source {../source}
\cleardoublepage
\chapter{代数}
%\cleardoublepage
\section{$O(n^2\log n)$求线性递推数列第n项}
Given $a_0, a_1, \cdots , a_{m-1} \\
\indent a_n = c_0 * a_{n-m} + \cdots + c_{m-1} * a_0 \\
\indent a_0 \ is \ the \ nth \ element, \cdots, a_{m-1} \ is \ the \ n+m-1th \ element
$
\inputminted{cpp}{\source/algebra/linear-recursion.cpp}
\section{任意模数快速傅里叶变换}
\inputminted{cpp}{\source/algebra/FFT_1e9+7.cpp}
\section{快速傅里叶变换}
\inputminted{cpp}{\source/algebra/FFT.cpp}
\section{闪电数论变换与魔力CRT}
\inputminted{cpp}{\source/algebra/NTT+CRT.cpp}
\section{多项式求逆}
Given polynomial a and n, b is the polynomial such that $a * b \equiv 1 (\mod x^n) $
\inputminted{cpp}{\source/algebra/polynomial-inverse.cpp}
\section{多项式除法}
d is quotient and r is remainder
\inputminted{cpp}{\source/algebra/polynomial-divide.cpp}
\section{多项式取指数取对数}
Given polynomial a and n, b is the polynomial such that $b \equiv e^a (\mod x^n)$ or $b \equiv \ln a (\mod x^n)$
\inputminted{cpp}{\source/algebra/polynomial-expandln.cpp}
\section{快速沃尔什变换}
\inputminted{cpp}{\source/algebra/FWT.cpp}
\section{单纯形}
\inputminted{cpp}{\source/algebra/simplex.cpp}
\cleardoublepage
\chapter{数论}
\section{大整数相乘取模}
\inputminted{cpp}{\source/number-theory/biginteger-multiply.cpp}
\section{EX-GCD}
\inputminted{cpp}{\source/number-theory/extended-euclid.cpp}
\section{Miller-rabin}
\inputminted{cpp}{\source/number-theory/miller-rabin.cpp}
\section{Pollard-rho.cpp}
\inputminted{cpp}{\source/number-theory/pollard-rho.cpp}
\section{非互质CRT}
first is remainder, second is module
\inputminted{cpp}{\source/number-theory/CRT.cpp}
\section{非互质CRT -zky}
\inputminted{cpp}{\source/number-theory/CRT-zky.cpp}
\section{Pell方程}
\inputminted{cpp}{\source/number-theory/Pell.cpp}
\section{Simpson}
\inputminted{cpp}{\source/number-theory/Simpson.cpp}
\section{解一元三次方程}
听说极端情况精度不够
\inputminted{cpp}{\source/number-theory/解一元三次方程.cpp}
\section{线段下整点}
solve for $\sum_{i=0}^{n-1} \lfloor \frac{a+bi}{m}\rfloor$, $n,m,a,b>0$
\inputminted{cpp}{\source/number-theory/integer-lattice-under-segment.cpp}
\section{线性同余不等式}
\inputminted{cpp}{\source/number-theory/线性同余不等式.cpp}
\section{EX-BSGS -zzq}
\inputminted{cpp}{\source/number-theory/EX-BSGS-zzq.cpp}
\section{EX-BSGS -zky}
\inputminted{cpp}{\source/number-theory/EX-BSGS-zky.cpp}
\section{分治乘法}
\inputminted{cpp}{\source/number-theory/DAC-multiply}
\section{组合数模$p^k$}
\inputminted{cpp}{\source/number-theory/CnmmodP.cpp}
%\section{线性筛}
%\inputminted{cpp}{\source/number-theory/linear-sieve.cpp}

\cleardoublepage
\chapter{图论}
\section{基础}
\inputminted{cpp}{\source/graph-theory/basis.cpp}
\section{KM}
\inputminted{cpp}{\source/graph-theory/KM.cpp}
\section{点双连通分量}
\texttt{bcc.forest} is a set of connected tree whose vertices are chequered with cut-vertex and BCC.
\inputminted{cpp}{\source/graph-theory/biconnected-graph-vertex.cpp}
\section{边双连通分量}
\inputminted{cpp}{\source/graph-theory/biconnected-graph-edge.cpp}
\section{最小树形图}
\inputminted{cpp}{\source/graph-theory/optimum-branching.cpp}
\section{带花树}
\inputminted{cpp}{\source/graph-theory/blossom-algorithm.cpp}
\section{Dominator Tree}
\inputminted{cpp}{\source/graph-theory/dominator-tree.cpp}
\section{无向图最小割}
\inputminted{cpp}{\source/graph-theory/stoer-wagner-algorithm.cpp}
\section{重口味费用流}
\inputminted{cpp}{\source/graph-theory/zkw-cost-flow.cpp}
\section{2-SAT}
\inputminted{cpp}{\source/graph-theory/2-satisfiability.cpp}

\cleardoublepage
\chapter{数据结构}
\section{Kd-tree}
\inputminted{cpp}{\source/data-structure/Kd-tree.cpp}
\section{LCT}
\inputminted{cpp}{\source/data-structure/LCT.cpp}
\section{树状数组上二分第k大}
\inputminted{cpp}{\source/data-structure/fenwicktree_kth.cpp}
\section{Treap}
\inputminted{cpp}{\source/data-structure/Treap.cpp}
\section{FHQ-Treap}
\inputminted{cpp}{\source/data-structure/fhqTreap.cpp}
\section{真-FHQTreap}
\inputminted{cpp}{\source/data-structure/true.fhqtreap.cpp}
\section{带修改莫队上树}
\inputminted{cpp}{\source/data-structure/mo-team-on-tree.cpp}
\section{虚树}
\inputminted{cpp}{\source/data-structure/virtual-tree.cpp}
\cleardoublepage
\chapter{字符串}
\section{Manacher}
\inputminted{cpp}{\source/string/manacher.cpp}
\section{指针版回文自动机}
\inputminted{cpp}{\source/string/PalindromeAutomaton_pointer.cpp}
%\section{数组版后缀自动机}
%\inputminted{cpp}{\source/string/SuffixAutomaton_array.cpp}
%\section{指针版后缀自动机}
%\inputminted{cpp}{\source/string/SuffixAutomaton_pointer.cpp}
%\section{广义后缀自动机}
%\inputminted{cpp}{\source/string/EX_SuffixAutomaton_pointer.cpp}
\section{后缀数组}
\inputminted{cpp}{\source/string/SA.cpp}
\section{最小表示法}
\inputminted{cpp}{\source/string/min_express.cpp}
\cleardoublepage
\chapter{计算几何}

\inputminted{cpp}{\source/computational-geometry/2d/basis.cpp}
\section{凸包}
\inputminted{cpp}{\source/computational-geometry/2d/convex.cpp}
\section{三角形的心}
\inputminted{cpp}{\source/computational-geometry/2d/triangle.cpp}
\section{半平面交}
\inputminted{cpp}{\source/computational-geometry/2d/half-plane-intersection.cpp}
\section{圆交面积及重心}
\inputminted{cpp}{\source/computational-geometry/2d/circles-intersections.cpp}

\section{三维向量绕轴旋转}
\inputminted{cpp}{\source/computational-geometry/3d/basis.cpp}
\section{三维凸包}
\inputminted{cpp}{\source/computational-geometry/3d/convex.cpp}

\cleardoublepage
\chapter{提示}

\section{控制cout输出实数精度}
\inputminted{cpp}{\source/hints/control-cout-precision.cpp}
\section{vimrc}
\inputminted{text}{\source/hints/vimrc}
\section{让make支持c艹11}
In .bashrc or whatever:
\begin{verbatim}
export CXXFLAGS='-std=c++11 -Wall'
\end{verbatim}

\section{tuple相关}
\inputminted{cpp}{\source/hints/tuple.cpp}

\section{线性规划转对偶}

\begin{equation*}
\begin{aligned}
&\text{maximize }\mathbf{c}^{T}\mathbf{x}\\
&\text{subject to }\mathbf{A}\mathbf{x} \leq \mathbf{b}, \mathbf{x} \geq 0
\end{aligned}
\Longleftrightarrow
\begin{aligned}
&\text{minimize }\mathbf{y}^{T}\mathbf{b}\\
&\text{subject to }\mathbf{y}^{T}\mathbf{A} \geq \mathbf{c}^{T}, \mathbf{y} \geq 0
\end{aligned}
\end{equation*}

\section{32-bit/64-bit随机素数}
\begin{tabular}{|l|l|}
\hline
\texttt{32-bit} & \texttt{64-bit} \\
\hline
73550053 & 1249292846855685773 \\
\hline
148898719 & 1701750434419805569 \\
\hline
189560747 & 3605499878424114901 \\
\hline
459874703 & 5648316673387803781 \\
\hline
1202316001 & 6125342570814357977 \\
\hline
1431183547 & 6215155308775851301 \\
\hline
1438011109 & 6294606778040623451 \\
\hline
1538762023 & 6347330550446020547 \\
\hline
1557944263 & 7429632924303725207 \\
\hline
1981315913 & 8524720079480389849 \\
\hline
\end{tabular}

\section{NTT 素数及其原根}
\begin{tabular}{|l|l|}
\hline
\texttt{Prime} & \texttt{Primitive root} \\
\hline
1053818881 & 7 \\
\hline
1051721729 & 6 \\
\hline
1045430273 & 3 \\
\hline
1012924417 & 5 \\
\hline
1007681537 & 3 \\
\hline
\end{tabular}

\section {Java Hints}
\inputminted{java}{\source/hints/template.java}

\cleardoublepage
		\end{spacing}
	\end{multicols}
\end{document}
