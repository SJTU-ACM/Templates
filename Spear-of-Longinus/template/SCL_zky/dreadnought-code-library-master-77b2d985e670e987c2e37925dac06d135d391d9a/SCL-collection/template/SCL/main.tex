\documentclass[a4paper,10pt]{report}
\usepackage{amsmath}
\usepackage{amssymb}
\usepackage{fontspec}
\usepackage{listings} 
\usepackage[top=1cm, bottom=2cm, left=2cm, right=1cm]{geometry}
\usepackage[BoldFont]{xeCJK}
\setlength{\parindent}{0em}
\setCJKmainfont{微软雅黑}
\lstset {
    language=C++,
    numbers=left,
    tabsize=4,
    breaklines=tr,
    extendedchars=false
}

\title{\LARGE{Standard Code Library}}
\author{Ceno Liu}
\date{June, 2013}

\begin{document}
\maketitle

\tableofcontents

\chapter{Algorithms and Datastructures}

	\section{*High Precision in C Plus Plus}
	Comming Soon

	\section{*Fraction Class}
	Comming Soon

	\section{Splay Tree}
	注意初始化内存池和null节点,以及根据需要修改update和relax,区间必须是1-based
	\begin{lstlisting}
const int MAX_NODE = 50000 + 10;
const int INF = 2000000000;

struct Node *null;

struct Node
{
	int rev, add;
	int val, maxv, size;
	Node *ch[2], *p;
	
	void set(Node *t, int _d) {
		ch[_d] = t;
		t->p = this;
	}
	int dir() {
		return this == p->ch[1];
	}
	void update() {
		maxv = max(max(ch[0]->maxv, ch[1]->maxv), val);
		size = ch[0]->size + ch[1]->size + 1;
	}
	void relax() {
		if (add) {
			ch[0]->appAdd(add);
			ch[1]->appAdd(add);
			add = 0;
		}
		if (rev) {
			ch[0]->appRev();
			ch[1]->appRev();
			rev = false;
		}
	}
	void appAdd(int x) {
		if (this == null) return;
		add += x;
		val += x;
		maxv += x;
	}
	void appRev() {
		if (this == null) return;
		rev ^= true;
		swap(ch[0], ch[1]);
	}
};

Node nodePool[MAX_NODE], *curNode;

Node *newNode(int val = 0)
{
	Node *t = curNode ++;
	t->maxv = t->val = val;
	t->rev = t->add = 0;
	t->size = 1;
	t->ch[0] = t->ch[1] = t->p = null;
	return t;
}

struct Splay
{
	Node *root;
	
	Splay() {
		root = newNode();
		root->set(newNode(), 0);
		root->update();
	}
	
	Splay(int *a, int N) { //sequence is 1-based
		root = build(a, 0, N + 1);
	}
	
	Node* build(int *a, int l, int r) {
		if (l > r) return null;
		int mid = l + r >> 1;
		Node *t = newNode(a[mid]);
		t->set(build(a, l, mid - 1), 0);
		t->set(build(a, mid + 1, r), 1);
		t->update();
		return t;
	}
	
	void rot(Node *t)
	{
		Node *p = t->p; int d = t->dir();
		p->relax(); t->relax();
		if (p == root) root = t;
		p->set(t->ch[! d], d);
		p->p->set(t, p->dir());
		t->set(p, ! d);
		p->update();
	}
	
	void splay(Node *t, Node *f = null)
	{
		for(t->relax(); t->p != f; ) {
			if (t->p->p == f) rot(t);
			else t->dir() == t->p->dir() ? (rot(t->p), rot(t)) : (rot(t), rot(t));
		}
		t->update();
	}
	
	Node* getKth(int k) {
		Node *t = root;
		int tmp;
		for( ; ; ) {
			t->relax();
			tmp = t->ch[0]->size + 1;
			if (tmp == k) return t;
			if (tmp < k) {
				k -= tmp;
				t = t->ch[1];
			} else
				t = t->ch[0];
		}
	}
	
	//make range[l,r] root->ch[1]->ch[0]
	//make range[x+1,x] to add something after position x
	void getRng(int l, int r) { 
		r += 2;
		Node *p = getKth(l);
		Node *q = getKth(r);
		splay(p); splay(q, p);
	}
	
	void addRng(int l, int r, int x) {
		getRng(l, r);
		root->ch[1]->ch[0]->appAdd(x);
	}
	
	void revRng(int l, int r) {
		getRng(l, r);
		root->ch[1]->ch[0]->appRev();
	}
	
	int maxvRng(int l, int r) {
		getRng(l, r);
		return root->ch[1]->ch[0]->maxv;
	}
};

void initNull()
{
	curNode = nodePool;
	null = curNode ++;
	null->size = 0;
	null->maxv = - INF;
}
\end{lstlisting} 


	\section{Link/Cut Tree}
	根据需求修改Node中的relax和update函数,修改access,以及Node的构造函数,注意初始化内存池和null节点
	\begin{lstlisting}
struct Node
{
	Node *ch[2], *p;
	int isroot;
	bool dir();
	void set(Node*, bool);
	void update();
	void relax();
} *null;

void rot(Node *t)
{
	Node *p = t->p; bool d = t->dir();
	p->relax(); t->relax();
	p->set(t->ch[! d], d);
	if (p->isroot) t->p = p->p, swap(p->isroot, t->isroot);
	else p->p->set(t, p->dir());
	t->set(p, ! d);
	p->update();
}

void Splay(Node *t)
{
	for(t->relax(); ! t->isroot; ) {
		if (t->p->isroot) rot(t);
		else t->dir() == t->p->dir() ? (rot(t->p), rot(t)) : (rot(t), rot(t));
	}
	t->update();
}

void Access(Node *t)
{
	for(Node *s = null; t != null; s = t, t = t->p) {
		Splay(t);
		t->ch[1]->isroot = true;
		s->isroot = false;
		t->ch[1] = s;
		t->update();
	}
}
bool Node::dir()
{
	return this == p->ch[1];
}
void Node::set(Node *t, bool _d)
{
	ch[_d] = t; t->p = this;
}
void Node::Update()
{

}
void Node::Relax()
{
	if (this == Null) return;

}
\end{lstlisting}



	\section{Binary Heap}
	双射堆,ind[v]表示标号为v的节点在堆中的位置
	\begin{lstlisting}
const int MAX_V = 100000 + 10;
struct Heap
{
	int tot;
	int a[MAX_V], h[MAX_V], ind[MAX_V];
	void exchange(int i, int j) {
		swap(h[i], h[j]);
		swap(ind[h[i]], ind[h[j]]);
	}
	inline int val(int x) {
		return a[h[x]];
	}
	void fixUp(int x) {
		if (x / 2 && val(x / 2) < val(x)) 
			exchange(x, x / 2), fixUp(x / 2);
	}
	void fixDown(int x) {
		int p = x * 2; if (p > tot) return;
		if (p < tot && val(p + 1) > val(p)) ++ p;
		if (val(p) > val(x))
			exchange(p, x), fixDown(p);
	}
	void Update(int i, int x) {
		a[i] = x;
		fixUp(ind[i]);
		fixDown(ind[i]);
	}
	int top() {
		return h[1];
	}
	void pop() {
		exchange(1, tot);
		-- tot;
		fixDown(1);
	}
	void insert(int i, int x) {
		++ tot;
		h[tot] = i;
		ind[i] = tot;
		a[i] = x;
		fixUp(tot);
	}
} H;
\end{lstlisting}


	\section{Leftist Tree}
	没写delete操作,注意初始化内存池和null节点
	\begin{lstlisting}
struct Node
{
	int dis, val;
	Node *ch[2];
} *null;

Node* merge(Node *u, Node *v)
{
	if (u == null) return v;
	if (v == null) return u;
	if (u->val < v->val) swap(u, v);
	u->ch[1] = merge(u->ch[1], v);
	if (u->ch[1]->dis > u->ch[0]->dis)
		swap(u->ch[1], u->ch[0]);
	u->dis = u->ch[1]->dis + 1;
	return u;
}

Node* newNode(int w)
{
	Node *t = totNode ++;
	t->ch[0] = t->ch[1] = null;
	t->val = w; t->dis = 0;
	return t;
}
\end{lstlisting}


	\section{Treap}
	包含build, insert和erase,执行时注意初始化内存池和null节点
	\begin{lstlisting}
struct Node {
	Node *child[2]; int key; int size, count, aux;
	inline Node(int _aux) {
		child[0] = child[1] = 0; key = size = count = 0; aux = _aux;
	}
	inline void update() { size = count + child[0]->size + child[1]->size; }
};
Node *null;
inline void print(Node *&x) {
	if (x == null) return; print(x->child[0]); printf("%d ", x->key);
	print(x->child[1]);
}
inline Node* create(int key)
	Node *x = new Node(rand() % INT_MAX); x->key = key; x->count = x->size = 1;
	x->child[0] = x->child[1] = null; return x;
}
inline void rotate(Node *&x, int dir) {
	Node *y = x->child[!dir]; x->child[!dir] = y->child[dir]; y->child[dir] = x;
	x->update(); y->update(); x = y;
}
inline void insert(Node *&x, int key) {
	if (x == null) { x = create(key); return; }
	if (x->key == key) x->count++;
	else if (x->key > key) {
		insert(x->child[0], key); if (x->child[0]->aux < x->aux) rotate(x, 1);
	} else {
		insert(x->child[1], key); if (x->child[1]->aux < x->aux) rotate(x, 0);
	} x->update();
}
inline void erase(Node *&x, int key) {
	if (x == null) return;
	if (x->key == key) {
		if (x->count > 1) x->count--;
		else if (x->child[0] == null && x->child[1] == null) {
			delete(x); x = null; return;
		} else if (x->child[0]->aux < x->child[1]->aux)
			rotate(x, 1), erase(x->child[1], key);
		else rotate(x, 0), erase(x->child[0], key);
	} else if (x->key > key) erase(x->child[0], key);
	else erase(x->child[1], key);
	x->update();
}
inline void prepare() { null = new Node(INT_MAX); }
\end{lstlisting}

	\section{Segment Tree}
	包含建树和区间操作样例,没有写具体操作
	\begin{lstlisting}
struct Tree
{
	int l, r;
	Tree *ch[2];
	Tree() {}
	Tree(int _l, int _r, int *sqn) {
		l = _l; r = _r;
		if (l + 1 == r)
			return;
		int mid = l + r >> 1;
		ch[0] = new Tree(l, mid, sqn);
		ch[1] = new Tree(mid, r, sqn);
	}
	
	void insert(int p, int x) {
		if (p < l || p >= r)
			return;
		//some operations
		if (l + 1 == r)
			return;
		ch[0]->insert(p, x);
		ch[1]->insert(p, x);
	}
	
	int query(int _l, int _r, int x) {
		if (_r <= l || _l >= r)
			return 0;
		if (_l <= l && _r >= r)
			// return information in [l, r)
		//merge ch[0]->query(_l, _r, x), ch[1]->query(_l, _r, x) and return
	}
};
\end{lstlisting}


	\section{Heavy-Light Decomposition}
	包含BFS剖分过程,线段树部分视题目而定
	\begin{lstlisting}
struct Tree()
{
	
};

int father[MAX_N], size[MAX_N], depth[MAX_N];
int bfsOrd[MAX_N], pathId[MAX_N], ordInPath[MAX_N], sqn[MAX_N];
Tree *root[MAX_N];

void doBfs(int s)
{
	int *que = bfsOrd;
	int qh = 0, qt = 0;
	father[s] = -1; depth[s] = 0;
	
	for(que[qt ++] = s; qh < qt; ) {
		int u = que[qh ++];
		foreach(iter, adj[u]) {
			int v = *iter;
			if (v == father[u])
				continue;
			father[v] = u;
			depth[v] = depth[u] + 1;
			que[qt ++] = v;
		}
	}
}

void doSplit()
{
	for(int i = N - 1; i >= 0; -- i) {
		int u = bfsOrd[i];
		size[u] = 1;
		foreach(iter, adj[u]) {
			int v = *iter;
			if (v == father[u])
				continue;
			size[u] += size[v];
		}
	}
	
	memset(pathId, -1, sizeof pathId);
	for(int i = 0; i < N; ++ i) {
		int top = bfsOrd[i];
		if (pathId[top] != -1)
			continue;
		
		int cnt = 0;
		for(int u = top; u != -1; ) {
			sqn[cnt] = val[u];
			ordInPath[u] = cnt;
			pathId[u] = top;
			++ cnt;
			
			int next = -1;
			foreach(iter, adj[u]) {
				int v = *iter;
				if (v == father[u])
					continue;
				if (next < 0 || size[next] < size[v])
					next = v;
			}
			u = next;
		}
		
		root[top] = new Tree(0, cnt, sqn);
	}
}

void prepare()
{
	doBfs(0);
	doSplit();
}

\end{lstlisting}


	
	\section{KD Tree}
	读入N个点,输出距离每个点的最近点。
	\begin{lstlisting}
const int MAX_N = 100000 + 10;
const int MAX_NODE = 200000 + 10;
const LL INF = 2000000000000000020LL;

int N;

struct Point
{
    int x, y, id;
};

LL dis(const Point &a, const Point &b)
{
    return 1LL * (a.x - b.x) * (a.x - b.x) + 1LL * (a.y - b.y) * (a.y - b.y);
}

struct Node
{
    Point p;
    int maxX, minX, maxY, minY;
    int l, r, d;
    Node *ch[2];
};

LL ret;
LL ans[MAX_N];
Node *root;
Point p[MAX_N], queryPoint;
Node *totNode, nodePool[MAX_NODE];

int cmpx(const Point &a, const Point &b)
{
    return a.x < b.x;
}
int cmpy(const Point &a, const Point &b)
{
    return a.y < b.y;
}

Node* newNode(int l, int r, Point p, int deep)
{
    Node *t = totNode ++;
    t->l = l; t->r = r;
    t->p = p; t->d = deep;
    t->maxX = t->minX = p.x;
    t->maxY = t->minY = p.y;
    return t;
}

void updateInfo(Node *t, Node *p)
{
    t->maxX = max(t->maxX, p->maxX);
    t->maxY = max(t->maxY, p->maxY);
    t->minX = min(t->minX, p->minX);
    t->minY = min(t->minY, p->minY);
}

Node* build(int l, int r, int deep)
{
    if (l == r) return NULL;
    if (deep & 1) sort(p + l, p + r, cmpx);
    else sort(p + l, p + r, cmpy);
    int mid = (l + r) >> 1;
    Node *t = newNode(l, r, p[mid], deep & 1);
    if (l + 1 == r) return t;
    t->ch[0] = build(l, mid, deep + 1);
    t->ch[1] = build(mid + 1, r, deep + 1);
    if (t->ch[0]) updateInfo(t, t->ch[0]);
    if (t->ch[1]) updateInfo(t, t->ch[1]);
    return t;
}

void updateAns(Point p)
{
    ret = min(ret, dis(p, queryPoint));
}

LL calc(Node *t, LL d)
{
    LL tmp;
    if (d) {
        if (queryPoint.x >= t->minX && queryPoint.x <= t->maxX) tmp = 0;
        else tmp = min(abs(queryPoint.x - t->maxX), abs(queryPoint.x - t->minX));
    } else {
        if (queryPoint.y >= t->minY && queryPoint.y <= t->maxY) tmp = 0;
        else tmp = min(abs(queryPoint.y - t->maxY), abs(queryPoint.y - t->minY));
    }
    return tmp * tmp;
}

void query(Node *t)
{
    if (t == NULL) return;
    if (t->p.id != queryPoint.id) updateAns(t->p);
    if (t->l + 1 == t->r) return;
    LL dl = t->ch[0] ? calc(t->ch[0], t->d) : INF;
    LL dr = t->ch[1] ? calc(t->ch[1], t->d) : INF;
    if (dl < dr) {
        query(t->ch[0]);
        if (ret > dr) query(t->ch[1]);
    } else {
        query(t->ch[1]);
        if (ret > dl) query(t->ch[0]);
    }
}

void solve()
{
    scanf("%d", &N);
    for(int i = 0; i < N; ++ i) {
        scanf("%d%d", &p[i].x, &p[i].y);
        p[i].id = i;
    }
    totNode = nodePool;
    root = build(0, N, 1);
    
    for(int i = 0; i < N; ++ i) {
        queryPoint = p[i];
        ret = INF;
        query(root);
        ans[p[i].id] = ret;
    }
    for(int i = 0; i < N; ++ i)
        printf("%I64d\n", ans[i]);
}

int main()
{
    int T; scanf("%d", &T);
    for( ; T --; )
        solve();
    return 0;
}
\end{lstlisting}

	\section{Manacher}
	len[i] means the max palindrome length centered i/2\\
	eg: cs = "abbacabbaddabbaae"\\
	len = 1 0 1 4 1 0 1 0 9 0 1 0 1 4 1 0 1 0 1 10 1 0 1 0 1 4 1 0 1 2 1 0 1 0
	\begin{lstlisting}
void palindrome(char cs[], int len[], int n) { 
	for (int i = 0; i < n * 2; ++i) {
		len[i] = 0;
	}
	for (int i = 0, j = 0, k; i < n * 2; i += k, j = max(j - k, 0)) {
		while (i - j >= 0 && i + j + 1 < n * 2 && cs[(i - j) / 2] == cs[(i + j + 1) / 2])
			j++;
		len[i] = j;
		for (k = 1; i - k >= 0 && j - k >= 0 && len[i - k] != j - k; k++) {
			len[i + k] = min(len[i - k], j - k);
		}
	}
}
\end{lstlisting}

	
	\section{Z Algorithm}
	传入字符串s和长度N,next[i]=LCP(s, s[i..N-1])
	\begin{lstlisting}
void z(char *s, int *next, int N)
{
	int j = 0, k = 1;
	while (j + 1 < N && s[j] == s[j + 1]) ++ j;
	next[0] = N - 1; next[1] = j;
	for(int i = 2; i < N; ++ i) {
		int far = k + next[k] - 1, L = next[i - k];
		if (L < far - i + 1) next[i] = L;
		else {
			j = max(0, far - i + 1);
			while (i + j < N && s[j] == s[i + j]) ++ j;
			next[i] = j; k = i;
		}
	}
}
\end{lstlisting}


	\section{Aho-Corasick Automaton}
	包含建trie和构造自动机的过程
	\begin{lstlisting}

struct acNode
{
    int id;
    acNode *ch[26], *fail;
} *totNode, *root, nodePool[MAX_V];

acNode* newNode()
{
    acNode *now = totNode ++;
    now->id = 0; now->fail = 0;
    memset(now->ch, 0, sizeof now->ch);
    return now;
}

void acInsert(char *c, int id)
{
    acNode *cur = root;
    while (*c) {
        int p = *c - 'A'; //change the index
        if (! cur->ch[p]) cur->ch[p] = newNode();
        cur = cur->ch[p];
        ++ c;
    }
    cur->id = id;
}

void getFail()
{
    acNode *cur;
    queue<acNode*> Q;
    for(int i = 0; i < 26; ++ i)
        if (root->ch[i]) {
            root->ch[i]->fail = root;
            Q.push(root->ch[i]);
        } else root->ch[i] = root;
    while (! Q.empty()) {
        cur = Q.front(); Q.pop();
        for(int i = 0; i < 26; ++ i)
            if (cur->ch[i]) {
                cur->ch[i]->fail = cur->fail->ch[i];
                Q.push(cur->ch[i]);
            } else cur->ch[i] = cur->fail->ch[i];
    }
}
\end{lstlisting}


	\section{Suffix Array}
	对于串a求SA,长度为N,M为元素值范围,height[i]=LCP(suf[rank[i]],suf[rank[i]-1])
	\begin{lstlisting}
const int MAX_N = 1000000 + 10;

int rank[MAX_N], height[MAX_N];

int cmp(int *x, int a, int b, int d)
{
	return x[a] == x[b] && x[a + d] == x[b + d];
}

void doubling(int *a, int N, int M)
{
	static int sRank[MAX_N], tmpA[MAX_N], tmpB[MAX_N];
	int *x = tmpA, *y = tmpB;
	for(int i = 0; i < M; ++ i) sRank[i] = 0;
	for(int i = 0; i < N; ++ i) ++ sRank[x[i] = a[i]];
	for(int i = 1; i < M; ++ i) sRank[i] += sRank[i - 1];
	for(int i = N - 1; i >= 0; -- i) sa[-- sRank[x[i]]] = i;
	
	for(int d = 1, p = 0; p < N; M = p, d <<= 1) {
		p = 0; for(int i = N - d; i < N; ++ i) y[p ++] = i;
		for(int i = 0; i < N; ++ i)
			if (sa[i] >= d) y[p ++] = sa[i] - d;
		for(int i = 0; i < M; ++ i) sRank[i] = 0;
		for(int i = 0; i < N; ++ i) ++ sRank[x[i]];
		for(int i = 1; i < M; ++ i) sRank[i] += sRank[i - 1];
		for(int i = N - 1; i >= 0; -- i) sa[-- sRank[x[y[i]]]] = y[i];
		swap(x, y); x[sa[0]] = 0; p = 1;
		for(int i = 1; i < N; ++ i)
			x[sa[i]] = cmp(y, sa[i], sa[i - 1], d) ? p - 1 : p ++;
	}
}

void calcHeight()
{
	for(int i = 0; i < N; ++ i) rank[sa[i]] = i;
	int cur = 0;
	for(int i = 0; i < N; ++ i)
		if (rank[i]) {
			if (cur) cur --;
			for( ; a[i + cur] == a[sa[rank[i] - 1] + cur]; ++ cur);
			height[rank[i]] = cur;
		}
}
\end{lstlisting}


	\section{Suffix Automaton}
	……保重
	\begin{lstlisting}
struct State
{
	int val;
	State *suf, *go[26];
} *root, *last;

State statePool[MAX_N], *curState;

void extend(int w)
{
	State *p = last, *np = curState ++;
	np->val = p->val + 1;
	for( ; p && ! p->go[w]; p = p->suf)
		p->go[w] = np;
	if (! p)
		np->suf = root;
	else {
		State *q = p->go[w];
		if (q->val == p->val + 1)
			np->suf = q;
		else {
			State *nq = curState ++;
			nq->val = p->val + 1;
			memcpy(nq->go, q->go, sizeof q->go);
			nq->suf = q->suf;
			q->suf = np->suf = nq;
			for( ; p && p->go[w] == q; p = p->suf)
				p->go[w] = nq;
		}
	}
	last = np;
}
\end{lstlisting}


	
	\section{*Dancing Links}
	Comming Soon

\chapter{Graph Theory and Network Algorithms}

	\section{Dijkstra}
	求s到其他点的最短路
	\begin{lstlisting}
int used[MAX_N], dis[MAX_N];
void dijstra(int s) {
	fill(dis, dis + N, INF); dis[s] = 0;
	priority_queue<pair<int, int> > que;
	que.push(make_pair(-dis[s], s));
	while (!que.empty()) {
		int u = que.top().second; que.pop();
		if (used[u]) continue;
		used[u] = true;
		foreach(e, E[u])
			if (dis[u] + e->w < dis[e->t]) {
				dis[e->t] = dis[u] + e->w;
				que.push(make_pair(-dis[e->t], e->t));
			}
	}
}
\end{lstlisting}


	\section{*Minimum Directed Spanning Tree}
	Comming Soon

	\section{KuhnMunkres}
	求完备匹配的最大权匹配,建好的完全图用w[][]存储,点数为N
	\begin{lstlisting}
#include <cstdio>
#include <cstdlib>
#include <algorithm>
#include <vector>
#include <cstring>
#include <string>
#include <iostream>

#define foreach(e, x) for(__typeof(x.begin()) e = x.begin(); e != x.end(); ++e)

using namespace std;

const int N = 333;
const int INF = (1 << 30);

int mat[N][N], lx[N], ly[N], vx[N], vy[N], slack[N];
int n, match[N];

bool find(int x) {
	vx[x] = 1;
	for(int i = 1; i <= n; i++) {
		if (vy[i]) {
			continue;
		}
		int temp = lx[x] + ly[i] - mat[x][i];
		if (temp == 0) {
			vy[i] = 1;
			if (match[i] == -1 || find(match[i])) {
				match[i] = x;
				return true;
			}
		} else {
			slack[i] = min(slack[i], temp);
		}
	}
	return false;
}

int KM() {
	for(int i = 1; i <= n; i++) {
		lx[i] = -INF;
		ly[i] = 0;
		match[i] = -1;
		for(int j = 1; j <= n; j++) {
			lx[i] = max(lx[i], mat[i][j]);
		}
	}
	for(int i = 1; i <= n; i++) {
		for(int j = 1; j <= n; j++) {
			slack[j] = INF;
		}
		for(; ;) {
			memset(vx, 0, sizeof(vx));
			memset(vy, 0, sizeof(vy));
			for(int j = 1; j <= n; j++) {
				slack[j] = INF;
			}
			if (find(i)) {
				break;
			}
			int delta = INF;
			for(int j = 1; j <= n; j++) {
				if (!vy[j]) {
					delta = min(delta, slack[j]);
				}
			}
			for(int j = 1; j <= n; j++) {
				if (vx[j]) {
					lx[j] -= delta;
				}
				if (vy[j]) {
					ly[j] += delta;
				} else {
					slack[j] -= delta;
				}
			}
		}
	}
	int answer = 0;
	for(int i = 1; i <= n; i++) {
		answer += mat[match[i]][i];
	}
	return answer;
}

int main() {
	while(scanf("%d", &n) != EOF) {
		for(int i = 1; i <= n; i++) {
			for(int j = 1; j <= n; j++) {
				scanf("%d", &mat[i][j]);
			}
		}
		printf("%d\n", KM());
	}
	return 0;
}
\end{lstlisting}


	\section{Maximum Flow}
	iSAP算法求S到T的最大流,点数为cntN,边表存储在*E[]中
	\begin{lstlisting}
struct Edge
{
	int t, c;
	Edge *n, *r;
} *E[MAX_V], edges[MAX_M], *totEdge;

Edge* makeEdge(int s, int t, int c)
{
	Edge *e = totEdge ++;
	e->t = t; e->c = c; e->n = E[s]; 
	return E[s] = e;
}

void addEdge(int s, int t, int c)
{
	Edge *p = makeEdge(s, t, c), *q = makeEdge(t, s, 0);
	p->r = q; q->r = p;
}

int maxflow()
{
	static int	cnt		[MAX_V];
	static int	h		[MAX_V];
	static int	que		[MAX_V];
	static int	aug		[MAX_V];
	static Edge	*cur	[MAX_V];
	static Edge	*prev	[MAX_V];
	fill(h, h + cntN, cntN);
	memset(cnt, 0, sizeof cnt);
	int qt = 0, qh = 0; h[T] = 0;
	for(que[qt ++] = T; qh < qt; ) {
		int u = que[qh ++];
		++ cnt[h[u]];
		for(Edge *e = E[u]; e; e = e->n) 
			if (e->r->c && h[e->t] == cntN) {
				h[e->t] = h[u] + 1;
				que[qt ++] = e->t;
			}
	}
	memcpy(cur, E, sizeof E);
	aug[S] = INF; Edge *e;
	int flow = 0, u = S;
	while (h[S] < cntN) {
		for(e = cur[u]; e; e = e->n)
			if (e->c && h[e->t] + 1 == h[u])
				break;
		if (e) {
			int v = e->t;
			cur[u] = prev[v] = e;
			aug[v] = min(aug[u], e->c);
			if ((u = v) == T) {
				int by = aug[T];
				while (u != S) {
					Edge *p = prev[u];
					p->c -= by;
					p->r->c += by;
					u = p->r->t;
				}
				flow += by;
			}
		} else {
			if (!-- cnt[h[u]]) return flow;
			h[u] = cntN;
			for(e = E[u]; e; e = e->n)
				if (e->c && h[u] > h[e->t] + 1)
					h[u] = h[e->t] + 1, cur[u] = e;
			++ cnt[h[u]];
			if (u != S) u = prev[u]->r->t;
		}
	}
	return flow;
}
\end{lstlisting}


	\section{Minimum Cost Maximum Flow}
	注意图的初始化,费用和流的类型依题目而定
	\begin{lstlisting}
int flow, cost;

struct Edge
{
	int t, c, w;
	Edge *n, *r;
} *totEdge, edges[MAX_M], *E[MAX_V];

Edge* makeEdge(int s, int t, int c, int w)
{
	Edge *e = totEdge ++;
	e->t = t; e->c = c; e->w = w; e->n = E[s];
	return E[s] = e;
}

void addEdge(int s, int t, int c, int w)
{
	Edge *st = makeEdge(s, t, c, w), *ts = makeEdge(t, s, 0, -w);
	st->r = ts; ts->r = st;
}

int SPFA()
{
	static int que[MAX_V];
	static int aug[MAX_V];
	static int in[MAX_V];
	static int dist[MAX_V];
	static Edge *prev[MAX_V];
	int qh = 0, qt = 0;
	
	int u, v;
	fill(dist, dist + cntN, INF); dist[S] = 0;
	fill(in, in + cntN, 0); in[S] = true;
	que[qt ++] = S; aug[S] = INF;
	for( ; qh != qt; ) {
		u = que[qh]; qh = (qh + 1) % MAX_N;
		for(Edge *e = E[u]; e; e = e->n) {
			if (! e->c) continue;
			v = e->t;
			if (dist[v] > dist[u] + e->w) {
				dist[v] = dist[u] + e->w;
				aug[v] = min(aug[u], e->c);
				prev[v] = e;
				if (! in[v]) {
					in[v] = true;
					if (qh != qt && dist[v] <= dist[que[qh]]) {
						qh = (qh - 1 + MAX_N) % MAX_N;
						que[qh] = v;
					} else {
						que[qt] = v;
						qt = (qt + 1) % MAX_N;
					}
				}
			}
		}
		in[u] = false;
	}
	
	if (dist[T] == INF) return false;
	cost += dist[T] * aug[T];
	flow += aug[T];
	for(u = T; u != S; ) {
		prev[u]->c -= aug[T];
		prev[u]->r->c += aug[T];
		u = prev[u]->r->t;
	}
	return true;
}

int minCostFlow()
{
	flow = cost = 0;
	while(SPFA());
	return cost;
}
\end{lstlisting}


	\section{Strongly Connected Component}
	N个点的图求SCC,totID为时间标记,top为栈顶,totCol为强联通分量个数,注意初始化
	\begin{lstlisting}
int totID, totCol;
int col[MAX_N], low[MAX_N], dfn[MAX_N];
int top, stack[MAX_N], instack[MAX_N];

int tarjan(int u)
{
	low[u] = dfn[u] = ++ totID;
	instack[u] = true; stack[++ top] = u;
	
	int v;
	foreach(it, adj[u]) {
		v = it->first;
		if (dfn[v] == -1)
			low[u] = min(low[u], tarjan(v));
		else if (instack[v])
			low[u] = min(low[u], dfn[v]);
	}
	
	if (low[u] == dfn[u]) {
		do {
			v = stack[top --];
			instack[v] = false;
			col[v] = totCol;
		} while(v != u);
		++ totCol;
	}
	return low[u];
}

void solve()
{
	totID = totCol = top = 0;
	fill(dfn, dfn + N, 0);
	for(int i = 0; i < N; ++ i)
		if (! dfn[i])
			tarjan(i);
}
\end{lstlisting}

	
	\section{*2-SAT}
	Comming Soon

\chapter{Number Theory}
	\section{Chinese Remainder Theorem}
	包括扩展欧几里得,求逆元,和保证除数互质条件下的CRT
	包括扩展欧几里得,求逆元,和保证除数互质条件下的CRT
\begin{lstlisting}
LL x, y;
void exGcd(LL a, LL b)
{
	if (b == 0) {
		x = 1;
		y = 0;
		return;
	}
	exGcd(b, a % b);
	LL k = y;
	y = x - a / b * y;
	x = k;
}

LL inversion(LL a, LL b)
{
	exGcd(a, b);
	return (x % b + b) % b;
}

LL CRT(vector<LL> m, vector<LL> a)
{
	int N = m.size();
	LL M = 1, ret = 0;
	for(int i = 0; i < N; ++ i)
		M *= m[i];
	
	for(int i = 0; i < N; ++ i) {
		ret = (ret + (M / m[i]) * a[i] % M * inversion(M / m[i], m[i])) % M;
	}
	return ret;
}
\end{lstlisting}



	\section{Pollard's Rho and Miller-Rabbin}
	大数分解和素性判断
	\begin{lstlisting}
typedef long long LL;

LL modMul(LL a, LL b, LL P)
{
	LL ret = 0;
	for( ; a; a >>= 1) {
		if (a & 1) {
			ret += b;
			if (ret >= P) ret -= P;
		}
		b <<= 1;
		if (b >= P) b -= P;
	}
	return ret;
}

LL modPow(LL a, LL u, LL P)
{
	LL ret = 1;
	for( ; u; u >>= 1, a = modMul(a, a, P))
		if (u & 1) ret = modMul(ret, a, P);
	return ret;
}

int millerRabin(LL N)
{
	if (N == 2) return true;
	LL t = 0, u = N - 1, x, y, a;
	for( ; ! (u & 1); ++ t, u >>= 1) ;
	for(int k = 0; k < 10; ++ k) {
		a = rand() % (N - 2) + 2;
		x = modPow(a, u, N);
		for(int i = 0; i < t; ++ i, x = y) {
			y = modMul(x, x, N);
			if (y == 1 && x > 1 && x < N - 1) return false;
		}
		if (x != 1) return false;
	}
	return true;
}

LL gcd(LL a, LL b)
{
	return ! b ? a : gcd(b, a % b);
}

LL pollardRho(LL N)
{
	LL i = 1, x = rand() % N;
	LL y = x, k = 2, d = 1;
	do {
		d = gcd(x - y + N, N);
		if (d != 1 && d != N) return d;
		if (++ i == k) y = x, k <<= 1;
		x = (modMul(x, x, N) - 1 + N) % N;
	} while (y != x);
	return N;
}

void getFactor(LL N)
{
	if (N < 2) return;
	if (millerRabin(N)) {
		//do some operations
		return;
	}
	LL x = pollardRho(N);
	getFactor(x);
	getFactor(N / x);
}
\end{lstlisting}

	
	\section{*Baby-step Giant-step}
	Comming soon
	
\chapter{Algebraic Algorithms}

	\section{*Linear Equations in $Z_m$}
	Comming Soon
	
	\section{*Linear Equations in $R$}
	Comming Soon
	
	\section{*Fast Fourier Transform}
	Comming Soon

\chapter{Computational Geometry}
	
	\section{Basic Operations}
	平面几何基本操作,之后的几个都需要先敲它
	\begin{lstlisting} 
#include <cstdio>
#include <cstring>
#include <algorithm>
#include <iostream>
#include <climits>
#include <numeric>
#define foreach(e,x) for(__typeof(x.begin()) e=x.begin();e!=x.end();++e)
#define REP(i,n) for(int i=0;i<n;++i)
using namespace std;

const double EPS = 1e-8;
inline int sign(double a) {
	return a < -EPS ? -1 : a > EPS;
}

struct Point {
	double x, y;
	Point() {
	}
	Point(double _x, double _y) :
			x(_x), y(_y) {
	}
	Point operator+(const Point&p) const {
		return Point(x + p.x, y + p.y);
	}
	Point operator-(const Point&p) const {
		return Point(x - p.x, y - p.y);
	}
	Point operator*(double d) const {
		return Point(x * d, y * d);
	}
	Point operator/(double d) const {
		return Point(x / d, y / d);
	}
	bool operator<(const Point&p) const {
		int c = sign(x - p.x);
		if (c)
			return c == -1;
		return sign(y - p.y) == -1;
	}
	double dot(const Point&p) const {
		return x * p.x + y * p.y;
	}
	double det(const Point&p) const {
		return x * p.y - y * p.x;
	}
	double alpha() const {
		return atan2(y, x);
	}
	double distTo(const Point&p) const {
		double dx = x - p.x, dy = y - p.y;
		return hypot(dx, dy);
	}
	double alphaTo(const Point&p) const {
		double dx = x - p.x, dy = y - p.y;
		return atan2(dy, dx);
	}
	void read() {
		scanf("%lf%lf", &x, &y);
	}
	double abs() {
		return hypot(x, y);
	}
	double abs2() {
		return x * x + y * y;
	}
	void write() {
		cout << "(" << x << "," << y << ")" << endl;
	}
};

#define cross(p1,p2,p3) ((p2.x-p1.x)*(p3.y-p1.y)-(p3.x-p1.x)*(p2.y-p1.y))

#define crossOp(p1,p2,p3) sign(cross(p1,p2,p3))

Point isSS(Point p1, Point p2, Point q1, Point q2) {
	double a1 = cross(q1,q2,p1), a2 = -cross(q1,q2,p2);
	return (p1 * a2 + p2 * a1) / (a1 + a2);
}

double calcArea(const vector<Point>&ps) {
	int n = ps.size();
	double ret = 0;
	for (int i = 0; i < n; ++i) {
		ret += ps[i].det(ps[(i + 1) % n]);
	}
	return ret / 2; //maybe need abs(ret)
}
\end{lstlisting} 

	
	\section{Convex Hull}
	Montone Chain
	\begin{lstlisting} 
vector<Point> convexHull(vector<Point> ps) {
	int n = ps.size();
	if (n <= 1)
		return ps;
	sort(ps.begin(), ps.end());
	vector<Point> qs;
	for (int i = 0; i < n; qs.push_back(ps[i++])) {
		while (qs.size() > 1 && crossOp(qs[qs.size()-2],qs.back(),ps[i]) <= 0)
			qs.pop_back();
	}
	for (int i = n - 2, t = qs.size(); i >= 0; qs.push_back(ps[i--])) {
		while (qs.size() > t && crossOp(qs[qs.size()-2],qs.back(),ps[i]) <= 0)
			qs.pop_back();
	}
	qs.pop_back();
	return qs;
}
\end{lstlisting} 

	
	\section{Convex Diameter}
	\begin{lstlisting} 
double convexDiameter(const vector<Point>&ps) {
	int n = ps.size();
	int is = 0, js = 0;
	for (int i = 1; i < n; ++i) {
		if (ps[i].x > ps[is].x)
			is = i;
		if (ps[i].x < ps[js].x)
			js = i;
	}
	double maxd = ps[is].distTo(ps[js]);
	int i = is, j = js;
	do {
		if ((ps[(i + 1) % n] - ps[i]).det(ps[(j + 1) % n] - ps[j]) >= 0)
			(++j) %= n;
		else
			(++i) %= n;
		maxd = max(maxd, ps[i].distTo(ps[j]));
	} while (i != is || j != js);
	return maxd;
}
\end{lstlisting} 

	
	\section{Convex Cut}
	\begin{lstlisting} 
vector<Point> convexCut(const vector<Point>&ps, Point q1, Point q2) {
	vector<Point> qs;
	int n = ps.size();
	for (int i = 0; i < n; ++i) {
		Point p1 = ps[i], p2 = ps[(i + 1) % n];
		int d1 = crossOp(q1,q2,p1), d2 = crossOp(q1,q2,p2);
		if (d1 >= 0)
			qs.push_back(p1);
		if (d1 * d2 < 0)
			qs.push_back(isSS(p1, p2, q1, q2));
	}
	return qs;
}
\end{lstlisting} 


\chapter{Classic Problems}
	\section{Nim Game}
		对于N堆石子,每人轮流取\\
		1)每堆石子个数为1,根据奇偶性直接判胜负\\
		2)有一堆个数大于1,先手必胜(直接根据奇偶性调整留一个还是取光)\\
		3)Nim游戏:每堆个数任意,xor和为0则为必败态,否则必胜,证明略\\
		4)Moore's Nim K游戏:从最少1堆最多K堆中取任意数量的石子,结论:把所有堆的石子个数按二进制表示,如果任意一位一的个数总和都为K+1的倍数则为必败态,否则为必胜态,显然Nim是K=1的特殊情况。\\
		5)anti-nim游戏:取到最后一个石子为败,结论:必胜态当且仅当:1)所有堆石子数都为1且游戏的SG值为0,2)存在某堆石子数大于1且游戏的sg值不为0\\

\end{document}
