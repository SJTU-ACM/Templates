\documentclass[a4paper,10pt]{book}
\usepackage{amsmath}
\usepackage{amssymb}
\usepackage{fontspec}
\usepackage{listings} 
\usepackage{harpoon}
\usepackage[left=1.5cm, right=1.5cm]{geometry}
\usepackage[BoldFont]{xeCJK}
\oddsidemargin -0.1 true cm
\if@twoside
	\evensidemargin -0.1 true cm
\fi
\setlength{\parindent}{0em}
\setCJKmainfont{Microsoft YaHei}
\lstset{
	language=C++,
	numbers=left,
	tabsize=4,
	breaklines=tr,
	extendedchars=false
}

\title{\LARGE{Standard Code Library}}
\author{Tempest}
\date{October, 2014}
\begin{document}
\maketitle

\tableofcontents

\newpage

\chapter{数学}
	\section{平面几何公式}
		\subsection*{三角形}
\begin{enumerate}
	\item 半周长
			$P=(a+b+c)/2$
	\item 面积
			$S=aH_a/2=ab\sin(C)/2=\sqrt{P(P-a)(P-b)(P-c)}$
	\item 中线
		$M_a=\sqrt{2(b^2+c^2)-a^2)}/2=\sqrt{b^2+c^2+2bc\cos(A)}/2$
	\item 角平分线 
		$T_a=\sqrt{bc((b+c)^2-a^2)}/(b+c)=2bc\cos(A/2)/(b+c)$
	\item 高线
		$H_a=b\sin(C)=c\sin(B)=\sqrt{b^2-((a^2+b^2-c^2)/(2a))^2}$
	\item 内切圆半径
		\begin{align*}
			r&=S/P=\arcsin(B/2)\sin(C/2)/\sin((B+C)/2)=4R\sin(A/2)\sin(B/2)\sin(C/2)\\
			&=\sqrt{(P-a)(P-b)(P-c)/P}=P\tan(A/2)\tan(B/2)\tan(C/2)
		\end{align*}
	\item 外接圆半径
		$R=abc/(4S)=a/(2\sin(A))=b/(2\sin(B))=c/(2\sin(C))$
\end{enumerate}

\subsection*{四边形}
	$D1,D2$为对角线,$M$对角线中点连线,$A$为对角线夹角
	\begin{enumerate}
		\item $a^2+b^2+c^2+d^2=D1^2+D2^2+4M^2$
		\item $S=D1D2\sin(A)/2$
		\item 圆内接四边形 $ac+bd=D1D2$
		\item 圆内接四边形,$P$为半周长 $S=\sqrt{(P-a)(P-b)(P-c)(P-d)}$
	\end{enumerate}

\subsection*{正n边形}
	$R$为外接圆半径,$r$为内切圆半径
	\begin{enumerate}
		\item 中心角 $A=2\pi/n$
		\item 内角 $C=(n-2)\pi/n$
		\item 边长 $a=2\sqrt{R^2-r^2}=2R\sin(A/2)=2r\tan(A/2)$
		\item 面积 $S=nar/2=nr^2\tan(A/2)=nR^2\sin(A)/2=na^2/(4\tan(A/2))$
	\end{enumerate}

\subsection*{圆}
	\begin{enumerate}
		\item 弧长 $l=rA$
		\item 弦长 $a=2\sqrt{2hr-h^2}=2r\sin(A/2)$
		\item 弓形高 $h=r-\sqrt{r^2-a^2/4}=r(1-\cos(A/2))=\arctan(A/4)/2$
		\item 扇形面积 $S1=rl/2=r^2A/2$
		\item 弓形面积 $S2=(rl-a(r-h))/2=r^2(A-\sin(A))/2$
	\end{enumerate}
	
\subsection*{棱柱}
	\begin{enumerate}
		\item 体积$V=Ah$,$A$为底面积,$h$为高
		\item 侧面积 $S=lp$,$l$为棱长,$p$为直截面周长
		\item 全面积 $T=S+2A$
	\end{enumerate}

\subsection*{棱锥}
	\begin{enumerate}
		\item 体积$V=Ah$,$A$为底面积,$h$为高
		\item 正棱锥侧面积 $S=lp$,$l$为棱长,$p$为直截面周长
		\item 正棱锥全面积 $T=S+2A$
	\end{enumerate}

\subsection*{棱台}
	\begin{enumerate}
		\item 体积 $V=(A1+A2+\sqrt{A1A2})h/3$, $A1,A2$为上下底面积,$h$为高
		\item 正棱台侧面积 $S=(p1+p2)l/2$,$p1,p2$为上下底面周长,$l$为斜高
		\item 正棱台全面积 $T=S+A1+A2$
	\end{enumerate}

\subsection*{圆柱}
	\begin{enumerate}
		\item 侧面积 $S=2\pi rh$
		\item 全面积 $T=2\pi r(h+r)$
		\item 体积 $V=\pi r^2h$
	\end{enumerate}

\subsection*{圆锥}
	\begin{enumerate}
		\item 母线 $l=\sqrt{h^2+r^2}$
		\item 侧面积 $S=\pi rl$
		\item 全面积 $T=\pi r(l+r)$
		\item 体积 $V=\pi r^2h/3$
	\end{enumerate}

\subsection*{圆台}
	\begin{enumerate}
		\item 母线 $l=\sqrt{h^2+(r1-r2)^2}$
		\item 侧面积 $S=\pi(r1+r2)l$
		\item 全面积 $T=\pi r1(l+r1)+\pi r2(l+r2)$
		\item 体积 $V=\pi(r1^2+r2^2+r1r2)h/3$
	\end{enumerate}

\subsection*{球}
	\begin{enumerate}
		\item 全面积 $T=4\pi r^2$
		\item 体积 $V=4\pi r^3/3$
	\end{enumerate}

\subsection*{球台}
	\begin{enumerate}
		\item 侧面积 $S=2\pi rh$
		\item 全面积 $T=\pi(2rh+r1^2+r2^2)$
		\item 体积 $V=\pi h(3(r1^2+r2^2)+h^2)/6$
	\end{enumerate}

\subsection*{球扇形}
	\begin{enumerate}
		\item 全面积 $T=\pi r(2h+r0)$,$h$为球冠高,$r0$为球冠底面半径
		\item 体积 $V=2\pi r^2h/3$
	\end{enumerate}


	\section{NTT}
		\begin{lstlisting}
const int modulo(786433);
const int G(10);//原根
int pw[999999];
void FFT(int P[], int n, int oper) {
	for(int i(1), j(0); i < n - 1; i++) {
		for(int s(n); j ^= s >>= 1, ~j & s;);
		if (i < j) 
			swap(P[i], P[j]);
	}
	int unit_p0;
	for(int d(0); (1 << d) < n; d++) {
		int m(1 << d), m2(m * 2);
		unit_p0 = oper == 1?pw[(modulo - 1) / m2]:pw[modulo - 1 - (modulo - 1) / m2];
		for(int i = 0; i < n; i += m2) {
			int unit(1);
			for(int j(0); j < m; j++) {
				int &P1 = P[i + j + m], &P2 = P[i + j];
				int t = (long long)unit * P1 % modulo;
				P1 = (P2 - t + modulo) % modulo;
				P2 = (P2 + t) % modulo;
				unit = (long long)unit * unit_p0 % modulo;
			}
		}
	}
}

int nn;
int A[N], B[N], C[N];
//A * B = C;
//len = nn
void multiply() {
	FFT(A, nn, 1);
	FFT(B, nn, 1);
	for(int i(0); i < nn; i++) {
		C[i] = (long long)A[i] * B[i] % modulo;
	}
	FFT(C,  nn, -1);
}

int main() {
	pw[0] = 1;
	for(int i(1); i < modulo; i++) {
		pw[i] = (long long)pw[i - 1] * G % modulo;
	}
}
\end{lstlisting}

	\section{FFT}
		\begin{lstlisting}
void FFT(Complex P[], int n, int oper) {
	for (int i(1), j(0); i < n - 1; i++) {
		for (int s(n); j ^= s >>= 1, ~j & s;);
		if (i < j)
			swap(P[i], P[j]);
	}
	Complex unit_p0;
	for (int d(0); (1 << d) < n; d++) {
		int m(1 << d), m2(m * 2);
		double p0(pi / m * oper);
		unit_p0.imag(sin(p0));
		unit_p0.real(cos(p0));
		for (int i(0); i < n; i += m2) {
			Complex unit = 1;
			for (int j = 0; j < m; j++) {
				Complex &P1 = P[i + j + m], &P2 = P[i + j];
				Complex t = unit * P1;
				P1 = P2 - t;
				P2 = P2 + t;
				unit = unit * unit_p0;
			}
		}
	}
}
void multiply() {
	FFT(a, n, 1);
	FFT(b, n, 1);
	for(int i(0); i < n; i++) {
		c[i] = a[i] * b[i];
	}
	FFT(c, n, -1);
	for(int i(0); i < n; i++) {
		ans[i] += (int)(c[i].real() / n + 0.5);
	}
}
\end{lstlisting}
	
	\section{中国剩余定理}
		包括扩展欧几里得,求逆元,和保证除数互质条件下的CRT
		包括扩展欧几里得,求逆元,和保证除数互质条件下的CRT
\begin{lstlisting}
LL x, y;
void exGcd(LL a, LL b)
{
	if (b == 0) {
		x = 1;
		y = 0;
		return;
	}
	exGcd(b, a % b);
	LL k = y;
	y = x - a / b * y;
	x = k;
}

LL inversion(LL a, LL b)
{
	exGcd(a, b);
	return (x % b + b) % b;
}

LL CRT(vector<LL> m, vector<LL> a)
{
	int N = m.size();
	LL M = 1, ret = 0;
	for(int i = 0; i < N; ++ i)
		M *= m[i];
	
	for(int i = 0; i < N; ++ i) {
		ret = (ret + (M / m[i]) * a[i] % M * inversion(M / m[i], m[i])) % M;
	}
	return ret;
}
\end{lstlisting}


	\section{求某年某月某日星期几}
		\begin{lstlisting}
int whatday(int d, int m, int y)
{
	int ans;
	if (m == 1 || m == 2) {
		m += 12; y --;
	}
	if ((y < 1752) || (y == 1752 && m < 9) || (y == 1752 && m == 9 && d < 3))
		ans = (d + 2 * m + 3 * (m + 1) / 5 + y + y / 4 + 5) % 7;
	else ans = (d + 2 * m + 3 * (m + 1) / 5 + y + y / 4 - y / 100 + y / 400) % 7;
	return ans;
}
\end{lstlisting}

	\section{快速求逆}
		\begin{lstlisting}
int inverse(int x, int modulo) {
	if(x == 1)
		return 1;
	return (long long)(modulo - modulo / x) * inverse(modulo % x, modulo) % modulo;
}
\end{lstlisting}

	\section{Miller Rabin}	
		miller\_rabin\_32是针对32位以下整数的; miller\_rabin\_64是针对64位以下整数的.
		直接调用prime()函数, 当返回值是true时表示是素数, 否则不是质数.
		\begin{lstlisting}
int const n = 3; int const base[] = {2, 7, 61};
int const n = 9; int const base[] = {2, 3, 5, 7, 11, 13, 17, 19, 23};
inline long long power(const long long &x, const long long &k, const long long &modular) {
	long long ans = 1, num = x % modular;
	for (long long i = k; i > 0; i >>= 1) {
		if (i & 1) ans = multiply(ans, num, modular);
		num = multiply(num, num, modular);
	} return ans;
}
inline bool check(const long long &p, const long long &base) {
	long long n = p - 1; for (; !(n & 1); n >>= 1);
	long long m = power(base, n, p);
	for (; n != p - 1 && m != 1 && m != p - 1; )
		m = multiply(m, m, p), n <<= 1;
	return m == p - 1 || (n & 1) == 1;
}
inline bool prime(const long long &p) {
	for (int i = 0; i < n; ++i) if (base[i] == p) return true;
	if (p == 1 || !(p & 1)) return false;
	for (int i = 0; i < n; ++i) if (!check(p, base[i])) return false;
	return true;
}
\end{lstlisting}
\chapter{计算几何}
\chapter{数据结构}
\chapter{图论}
\chapter{字符串}
\chapter{别的}
\end{document}
